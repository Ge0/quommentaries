\documentclass{article}
\usepackage[utf8]{inputenc}
\usepackage[a4paper, total={7in, 10in}]{geometry}

\usepackage{amsmath}
\usepackage{amsfonts}
\usepackage{braket}
    \renewcommand{\braket}[2]{\mathop{\mathinner{
        \langle{#1}|{#2}\rangle}}}
    \newcommand{\ketbra}[2]{\mathop{\mathinner{
        |{#1}\rangle \langle{#2}|}}}
    \newcommand{\norm}[1]{||#1||}

\usepackage{hyperref}
    \hypersetup{
        colorlinks=true,
        linkcolor= blue,
        filecolor=magenta,      
        urlcolor=blue,
    }
    
\usepackage{titlesec}
    \setlength{\parindent}{0em}
    \setlength{\parskip}{1em}
    \titlespacing\subsection{1em}{0.5em}{-0.5em}
    \titlespacing\subsubsection{1em}{0.5em}{-0.5em}
    \titlespacing\section{0em}{1em}{0.5em}

\usepackage{fancyhdr}
    \pagestyle{fancy}
    %\renewcommand{\chaptermark}[1]{\markboth{#1}{}}
    \renewcommand{\sectionmark}[1]{\markright{\arabic{section}.\ #1}}
    \rhead[\nouppercase{\rightmark}]{\href{mailto:gbezerra@lncc.br}{gbezerra@lncc.br}}

\usepackage{comment}

\usepackage{graphicx, caption}
\usepackage{subcaption}
\usepackage{float}
\usepackage{enumitem}

\usepackage{array} %to center table content
\newcolumntype{P}[1]{>{\centering\arraybackslash}p{#1}} %to center table content

\renewcommand\labelitemii{$\circ$}
\renewcommand\labelitemiii{\textasteriskcentered}
\renewcommand{\mod}{\text{\ mod\ }}

\newcommand{\todo}[1]{ %adds flashy todo text
    { \color{red} \MakeUppercase{\textbf{TODO: #1}} }
}

\newcommand{\mean}[1]{\mathop{\mathinner{
    \langle {#1} \rangle}}}

\newcommand{\R}{\mathbb{R}}
\newcommand{\C}{\mathbb{C}}

\newcommand{\matrx}[1]{\left[\begin{matrix} #1 \end{matrix}\right]}

\allowdisplaybreaks %allows page break inside begin{align} block

%%%%%%%%%%%%%%%%%%%%%%%%%%%%%%%%%%%%%%%%%%%%%%%%%%%%%%%%%%%%%%%%%%%%%%%%%%%%%%%%%%%%%%%%%%%%
\title{Quommentaries}
\author{
    Gustavo Bezerra (Koruja)\\
    \href{mailto:gbezerra@lncc.br}{gbezerra@lncc.br}
}
%\date{}

\begin{document}
	\maketitle

	\tableofcontents
	\pagebreak

	\section{Leftoverture}
This repository is dedicated to solve exercises and comment on Quantum Computing. Most of the discussion is based on \href{https://www.cambridge.org/core/books/quantum-computation-and-quantum-information/01E10196D0A682A6AEFFEA52D53BE9AE}{Nielsen And Chuang's book "Quantum Computation and Quantum Information"}.
In addition, \href{https://books.google.com.br/books?hl=pt-BR&lr=&id=8jwVDAAAQBAJ&oi=fnd&pg=PR5&dq=an+introduction+quantum+computation+mosca&ots=1EgvgtQL_A&sig=YqURWZlJOHdatZHyly-cNPLhdxM#v=onepage&q=an\%20introduction\%20quantum\%20computation\%20mosca&f=false}{Kaye, Laflamme and Mosca's "An Introduction to Quantum Computing"}
is used as a complementary book, as well as \href{http://www.cambridge.org/gb/academic/subjects/computer-science/cryptography-cryptology-and-coding/quantum-computing-computer-scientists?format=HB}{Yanofsky and Mannucci's "Quantum Computing for Computer Scientists"}
- recommended by \href{http://vitorgreati.me/}{Greati}.

\subsection{Objective}
Although Nielsen and Chuang's book is very famous, some equations may be solved too quickly. This may discourage the reader to continue the studies if the basic concepts were not mastered. One of the objectives of this repository is to support those who are studying Quantum Computing and Quantum Information by explaining some of these equations step-by-step.

In addition, the exercises present in the book may not be trivial for beginners. Hence, this repository attempts to help the students by showing a detailed solution or, at least, a sketch.

\subsection{Disclaimer}
This repository is being constructed by an \textbf{undergaduate student}. Henceforth, the notes, commentaries and exercises are \textbf{suscetible to errors}. Please, \textbf{do not hesitate to give feedback} (\href{mailto:gustavowl@lcc.ufrn.br}{gustavowl@lcc.ufrn.br}).

\pagebreak
	\section{Introduction}
TODO
	\section{Nielsen and Chuang - Chapter 01}

\subsection{Section 1.2}
\subsubsection{Qubit representation in a Bloch Sphere}

The explanation to the following formula is not given by the book.

\[
\ket{\psi} = e^{i \gamma} \left(
    cos \frac{\theta}{2}\ket{0} + e^{i \varphi} sin \frac{\theta}{2} \ket{1}
\right)
\]

However, \href{https://github.com/victoragnez}{Agnez} came up with a simple explanation using spherical coordinates. Its details can be found at \todo{add link to Computer Society}
\subsection{Section 1.4}
\subsubsection{Deutsch's Algorithm}

While explaining Deutsch's algorithm, state \( \ket{\psi_1} \) is obtained.

\[
\ket{\psi_1} = \left[ \frac{\ket{0} + \ket{1}}{\sqrt{2}} \right]
    \left[ \frac{\ket{0} - \ket{1}}{\sqrt{2}} \right]
\]

Then, the Unitary gate \(U_f\) is applied to state \(\ket{\psi}\) and how the result obtained in state \(\ket{\psi_2}\) may not be clear enough to the reader. First, recall that \(f(x) : \{0, 1\} \to \{0, 1\}\). That is, the function maps the qubits in state \(\ket{0}\) to either state \(\ket{0}\) or \(\ket{1}\). Analogously, qubits in state \(\ket{1}\) are mapped to state \(\ket{0}\) or \(\ket{1}\).

Henceforth, there are for possible functions: two possibilities where \(f(0) = f(1)\) and two possibilities where \(f(0) \neq f(1)\).

\begin{itemize}
    \item \(f(0) = f(1)\)
    \begin{itemize}
        \item \(f(0) = f(1) = 0\)
        \item \(f(0) = f(1) = 1\)
    \end{itemize}
    
    \item \(f(0) \neq f(1)\)
    \begin{itemize}
        \item \(f(0) = 0, f(1) = 1\)
        \item \(f(0) = 1, f(1) = 0\)
    \end{itemize}
\end{itemize}

Note that \(U_f\) does not apply any operation to the first qubit (\(x\)), but applies \(y \oplus f(x)\) to the second qubit (\(y\)). Note that, using the distributive property, the state \(\ket{\psi_1}\) may be written as
\[\ket{\psi_1} = \frac{\ket{00} - \ket{01} + \ket{10} - \ket{11}}{2}\]
Then, analyzing what would happen if any of the four possibilities for \(U_f\) were applied:

\begin{itemize}
    \item Apply \(U_f\) to \(\ket{\psi_1}\) when \(f(0) = f(1) = 0\)
    
    \begin{align}
        \ket{\psi_2} &= \frac{\ket{0\ (0 \oplus f(0)\ )} - \ket{0\ (1 \oplus f(0)\ )} +
        \ket{1\ (0 \oplus f(1)\ )} - \ket{1\ (1 \oplus f(1)\ )}}{2}
        \\
        \ket{\psi_2} &=  \frac{\ket{0\ (0 \oplus 0\ )} - \ket{0\ (1 \oplus 0\ )} +
        \ket{1\ (0 \oplus 0\ )} - \ket{1\ (1 \oplus 0\ )}}{2}
        \\
        \ket{\psi_2} &= \frac{\ket{00} - \ket{01} + \ket{10} - \ket{11}}{2}
    \end{align}
    
    Then, inversely applying the distributive property:
    \begin{equation}
        \ket{\psi_2} =
        \left[ \frac{\ket{0} + \ket{1}}{\sqrt{2}} \right]
        \left[ \frac{\ket{0} - \ket{1}}{\sqrt{2}} \right]
    \end{equation}
        
    \item Apply \(U_f\) to \(\ket{\psi_1}\) when \(f(0) = f(1) = 1\)
    
    \begin{align}
        \ket{\psi_2} &= \frac{\ket{0\ (0 \oplus f(0)\ )} - \ket{0\ (1 \oplus f(0)\ )} +
        \ket{1\ (0 \oplus f(1)\ )} - \ket{1\ (1 \oplus f(1)\ )}}{2}
        \\
        \ket{\psi_2} &=  \frac{\ket{0\ (0 \oplus 1\ )} - \ket{0\ (1 \oplus 1\ )} +
        \ket{1\ (0 \oplus 1\ )} - \ket{1\ (1 \oplus 1\ )}}{2}
        \\
        \ket{\psi_2} &= \frac{\ket{01} - \ket{00} + \ket{11} - \ket{10}}{2}
        \\
        \ket{\psi_2} &= - \frac{\ket{00} - \ket{01} + \ket{10} - \ket{11}}{2}
    \end{align}
    
    Then, inversely applying the distributive property:
    \begin{equation}
        \ket{\psi_2} = -
        \left[ \frac{\ket{0} + \ket{1}}{\sqrt{2}} \right]
        \left[ \frac{\ket{0} - \ket{1}}{\sqrt{2}} \right]
    \end{equation}
    
    \textbf{Henceforth, the first part of Nielsen and Chuangs's \emph{equation 1.43} was obtained:}
    
    \[
    \ket{\psi_2} = \pm \left[ \frac{\ket{0} + \ket{1}}{\sqrt{2}} \right]
        \left[ \frac{\ket{0} - \ket{1}}{\sqrt{2}} \right]\ if \ f(0) = f(1)
    \]
    
    \item Apply \(U_f\) to \(\ket{\psi_1}\) when \(f(0) = 0, f(1) = 1\)
    
        \begin{align}
        \ket{\psi_2} &= \frac{\ket{0\ (0 \oplus f(0)\ )} - \ket{0\ (1 \oplus f(0)\ )} +
        \ket{1\ (0 \oplus f(1)\ )} - \ket{1\ (1 \oplus f(1)\ )}}{2}
        \\
        \ket{\psi_2} &=  \frac{\ket{0\ (0 \oplus 0\ )} - \ket{0\ (1 \oplus 0\ )} +
        \ket{1\ (0 \oplus 1\ )} - \ket{1\ (1 \oplus 1\ )}}{2}
        \\
        \ket{\psi_2} &= \frac{\ket{00} - \ket{01} + \ket{11} - \ket{10}}{2}
    \end{align}
    
    Then, inversely applying the distributive property:
    \begin{equation}
        \ket{\psi_2} =
        \left[ \frac{\ket{0} - \ket{1}}{\sqrt{2}} \right]
        \left[ \frac{\ket{0} - \ket{1}}{\sqrt{2}} \right]
    \end{equation}
    
    \item Apply \(U_f\) to \(\ket{\psi_1}\) when \(f(0) = 1, f(1) = 0\)
    
        \begin{align}
        \ket{\psi_2} &= \frac{\ket{0\ (0 \oplus f(0)\ )} - \ket{0\ (1 \oplus f(0)\ )} +
        \ket{1\ (0 \oplus f(1)\ )} - \ket{1\ (1 \oplus f(1)\ )}}{2}
        \\
        \ket{\psi_2} &=  \frac{\ket{0\ (0 \oplus 1\ )} - \ket{0\ (1 \oplus 1\ )} +
        \ket{1\ (0 \oplus 0\ )} - \ket{1\ (1 \oplus 0\ )}}{2}
        \\
        \ket{\psi_2} &= \frac{\ket{01} - \ket{00} + \ket{10} - \ket{11}}{2}
        \\
        \ket{\psi_2} &= - \frac{\ket{00} - \ket{01} - \ket{10} + \ket{11}}{2}
    \end{align}
    
    Then, inversely applying the distributive property:
    \begin{equation}
        \ket{\psi_2} = -
        \left[ \frac{\ket{0} - \ket{1}}{\sqrt{2}} \right]
        \left[ \frac{\ket{0} - \ket{1}}{\sqrt{2}} \right]
    \end{equation}
    
    \textbf{Henceforth, the second part of Nielsen and Chuangs's \emph{equation 1.43} was obtained:}
    
    \[
    \ket{\psi_2} = \pm \left[ \frac{\ket{0} - \ket{1}}{\sqrt{2}} \right]
        \left[ \frac{\ket{0} - \ket{1}}{\sqrt{2}} \right]\ if \ f(0) \neq f(1)
    \]
\end{itemize}

Also, note that something interesting happened. Even though the \(U_f\) was not supposed to alter the state of the first qubit (\(\ket{x}\)); it is, in fact, changed. As a result, measuring \(\ket{x}\) is sufficient to determine the specified property of \(f(x)\).

\pagebreak
	\section{Nielsen and Chuang - Chapter 02} 
\label{sec:nielsen-and-chuang-chapter-2}

\subsection{Section 2.1.4}
\subsubsection{Outer Product Representation of A}
    \label{sec:nielsen-and-chuang-outer-product-of-a}
    It is stated from Equation 2.25 that it is possible to
    "[...] see from this equation that A has matrix element
    $\bra{w_j}A\ket{v_i}$".
    To see this, it is possible to compare the matrix and
    Dirac representations. Consider two systems $V$ and $W$
    with dimensions $n$ and $m$, respectively.
    In addition, suppose $\ket{v} \in V$ and $\ket{w} \in W$. 
    
    Using matrix representation:
    \begin{align}
        \braket{v}{w} &= \left[ \begin{matrix}
            v_1 \\ \vdots \\ v_i \\ \vdots \\ v_n
            \end{matrix} \right]
            \left[ \begin{matrix}
            w_1 & \cdots & w_j & \cdots & w_m
            \end{matrix} \right] \\
        &= \left[ \begin{matrix}
            v_1 w_1 & \cdots & v_1 w_j & \cdots & v_1 w_m \\
            \vdots & \ddots & \vdots & \ddots & \vdots \\
            v_i w_1 & \cdots & v_i w_j & \cdots & v_i w_m \\
            \vdots & \ddots & \vdots & \ddots & \vdots \\
            v_n w_1 & \cdots & v_n w_j & \cdots & v_n w_m
            \end{matrix} \right]
    \end{align}
    
    Using Dirac notation and the last part of Equation 2.21:
    \begin{align}
        \ketbra{v}{w} &= \sum_{ij} v_i\ket{i} w_j\bra{j} \\
            &= \sum_{ij} v_i w_j \ketbra{i}{j}
    \end{align}
    
    Then, comparing matrix and Dirac representation,
    it is easily verified that the matrix has elements
    $m_{ij} = v_i w_j$ for the $i$-th row and $j$-th column
    ( $\braket{i}{j}$ )
    with respect to the orthonormal basis $\ket{i}$ and $\ket{j}$
    for systems $V$ and $W$, respectively.
\subsection{Section 2.1.5}
\subsubsection{Exercise 2.11}
The eigenvectors, eigenvalues and diagonal representations of $Y$ will be calculated.
The process for $X$ and $Z$ is similar.
All answers are summarised in the end of this section.

Recall that
\begin{align}
    Y = \left[ \begin{matrix}
            0 & -i \\
            i & 0
        \end{matrix} \right].
\end{align}
Therefore,
\begin{align}
    det(Y - \lambda I) &= \left| \begin{matrix}
            - \lambda & -i \\
            i & - \lambda
        \end{matrix} \right| \\
    &= (-\lambda)^2 - (- i^2) \\
    &= \lambda^2 -1 = 0 .
\end{align}
Thus, $\lambda = \pm 1$.

Calculate the corresponding eigenvectors, using row reduction.
For $\lambda_0 = 1$,
\begin{align}
    \left[ \begin{matrix}
        -1 & -i \\ i & -1
    \end{matrix} \right]
    \begin{matrix}
        ~ \\
        +i \cdot row_1
    \end{matrix}
    %
    &\sim
    \left[ \begin{matrix}
        -1 & -i \\ 0 & 0
    \end{matrix} \right].
\end{align}
Therefore, since $-1a -ib = 0$,
the eigenvectors of $\lambda_0$ are the span of
\begin{align}
    \ket{\lambda_0} = \left\{ \left[ \begin{matrix} 1 \\ i \end{matrix} \right] \right\}.
\end{align}
That is, if $z \in \mathbb{C}$, then $-1 \cdot 1 \cdot z - i \cdot i \cdot z = 0$.
Finally, normalise $\ket{\lambda_0}$.
\begin{align}
    \frac{\ket{\lambda_0}}{\sqrt{\braket{\lambda_0}{\lambda_0}}} &=
    \frac{\ket{\lambda_0}}{\sqrt{1\cdot 1 + (-i) \cdot -i}} \\
    &= \frac{1}{\sqrt2} \ket{\lambda_0}.
\end{align}
Henceforth, let $\ket{\lambda_0}$ denote the normalised eigenvector.

Calculate the eigenvector span for $\lambda_1 = -1$.
\begin{align}
    \left[ \begin{matrix}
        1 & -i \\ i & 1
    \end{matrix} \right]
    \begin{matrix}
        ~ \\
        -i \cdot row_1
    \end{matrix}
    %
    &\sim
    \left[ \begin{matrix}
        1 & -i \\ 0 & 0
    \end{matrix} \right].
\end{align}
Therefore, since $1a -ib = 0$,
the eigenvectors of $\lambda_1$ are the span of
\begin{align}
    \ket{\lambda_1} = \left\{ \left[ \begin{matrix} 1 \\ -i \end{matrix} \right] \right\}.
\end{align}
That is, if $z \in \mathbb{C}$, then $1 \cdot 1 \cdot z - i \cdot (-i) \cdot z = 0$.
Finally, normalise $\ket{\lambda_1}$.
\begin{align}
    \frac{\ket{\lambda_1}}{\sqrt{\braket{\lambda_1}{\lambda_1}}} &=
    \frac{\ket{\lambda_1}}{\sqrt{1\cdot 1 + i \cdot (-i)}} \\
    &= \frac{1}{\sqrt2} \ket{\lambda_1}.
\end{align}
Henceforth, let $\ket{\lambda_1}$ denote the normalised eigenvector.

Thus, the diagonal representation is
$Y = \sum_i \lambda_i \ketbra{\lambda_i}{\lambda_i} = 
1 \ketbra{\lambda_0}{\lambda_0} - 1 \ketbra{\lambda_1}{\lambda_1}$.

Repeat the procedure for the remaining Pauli matrices.
\hyperref[tab:nielsen-and-chuang-answers-exercise-2-11]{
    Table \ref{tab:nielsen-and-chuang-answers-exercise-2-11}}
summarises the exercise answers.

\begin{table}[htb]
\def\arraystretch{2}%  1 is the default, change whatever you need
\centering
\begin{tabular}{|P{10em}|P{2em}|P{2em}|P{4em}|P{4em}|P{10em}|}
\hline
Pauli Matrix & $\lambda_0$ & $\lambda_1$  & $\ket{\lambda_0}$ & $\ket{\lambda_1}$ & Diagonal Representation
\\ \hline
$X$ & $1$ & $-1$ & $\ket+$ & $\ket-$ & $\ketbra++ - \ketbra--$
\\ \hline
$Y$ & $1$ & $-1$ & $\frac{1}{\sqrt2} \left[ \begin{matrix} 1 \\ i \end{matrix} \right]$ &
    $\frac{1}{\sqrt2} \left[ \begin{matrix} 1 \\ -i \end{matrix} \right]$ &
    $\ketbra{\lambda_0}{\lambda_0} - \ketbra{\lambda_1}{\lambda_1}$
\\ \hline
$Z$ & $1$ & $-1$ & $\ket0$ & $\ket1$ & $\ketbra00 - \ketbra11$
\\ \hline
\end{tabular}
\caption{Answers of Exercise 2.11}
\label{tab:nielsen-and-chuang-answers-exercise-2-11}
\end{table}
\subsection{Section 2.1.8}
\subsubsection{Exercise 2.35}
\label{sec:nielsen-and-chuang-exercise-2-35}

In order to solve this exercise,
it is necessary to find a spectral decomposition for $\vec{v} \vec{\sigma}$.
Then, it is possible to apply the definition of Operator functions.

With the aid of \hyperref[sec:nielsen-and-chuang-exercise-2-60]{Exercise 2.60},
the required spectral decomposition is obtained:
\begin{align}
    \vec{v}\vec{\sigma} &= +1 P_+ -1 P_- \\
    &= \frac{I + \vec{v}\vec{\sigma}}{2} - \frac{I - \vec{v}\vec{\sigma}}{2}
\end{align}

Now, calculating the value of $exp(i \theta \vec{v}\cdot\vec{\sigma})$
and applying the definition of Operator functions:
\begin{align}
    exp(i \theta \vec{v}\cdot\vec{\sigma}) &=
        exp(i \theta P_+) + exp(-i \theta P_-) \\
    &= exp(i \theta) P_+ + exp(-i \theta) P_- \\
    &= e^{i \theta} P_+ + e^{-i \theta} P_-
\end{align}

Then, applying Euler's Formula:
\begin{align}
    exp(i \theta \vec{v}\cdot\vec{\sigma}) &=
        cos(\theta)P_+ + i sin(\theta)P_+ +
        cos(\theta)P_- - i sin(\theta)P_- \\
    &= cos(\theta)\frac{I + \vec{v}\vec{\sigma}}{2} +
        i sin(\theta)\frac{I + \vec{v}\vec{\sigma}}{2} +
        cos(\theta)\frac{I - \vec{v}\vec{\sigma}}{2} -
        i sin(\theta)\frac{I - \vec{v}\vec{\sigma}}{2} \\
    &= cos(\theta) \left(
            \frac{I + \vec{v}\vec{\sigma} + I - \vec{v}\vec{\sigma}}{2}
        \right)
        + i sin(\theta) \left(
            \frac{I + \vec{v}\vec{\sigma} - I + \vec{v}\vec{\sigma}}{2}
        \right) \\
    &= cos(\theta) I + i\ sin(\theta) \vec{v}\vec{\sigma}
\end{align}

Thus obtaining the required result.
\subsection{Section 2.2.5}
\subsubsection{Equation (2.116)}
\label{sec:nielsen-and-chuang-equation-2-116}

This definition may be rather confusing since $\vec{v}$ is defined
but the definition of $\vec{\sigma}$ is not recapitulated.
More specifically, $\sigma_1$, $\sigma_1$, and $\sigma_3$ were
defined in Table 2.2 of the book.
However, since $X$, $Y$ and $Z$ are used more frequently to
denote the Pauli Matrices, the reader may not remind of the
equivalent $\sigma$ notation.

In order to reduce the calculi on Exercise 2.60
(section \ref{sec:nielsen-and-chuang-exercise-2-60}),
The matrix form of $\vec{v} \vec{\sigma}$ is computed in this section.

Recall that $\sigma_1 \equiv X$, $\sigma_2 \equiv Y$,
and $\sigma_3 \equiv Z$, which have matrix form as defined in
the book's Table 2.2.
Since $\vec{v}$ is a vector with components
$v_1, v_2, v_3 \in \mathbb{R}$,
it is possible to interpret $\vec{\sigma}$ as being a vector of
matrices, i.e. $\vec{\sigma} \in (\mathbb{R}^{2 \times 2})^3$
Therefore:

\begin{align}
    \vec{v} \vec{\sigma} &= v_1 \left[ \begin{matrix}
        0 & 1 \\ 1 & 0 \end{matrix}\right] +
        v_2 \left[ \begin{matrix}
        0 & -i \\ i & 0 \end{matrix} \right] +
        v_3 \left[ \begin{matrix}
        1 & 0 \\ 0 & -1\end{matrix} \right]
        \\
        &= \left[ \begin{matrix}
        v_3 & v_1 -v_2 i \\ v_1 + v_2 i & - v_3
        \end{matrix} \right]
\end{align}

%%%%%%%%%%%%%%%%%%%%%%%%%%%%%%%%%%%%%%%%%%%%%%%%%%%%%%%%%%%%%%%%%%%
\subsubsection{Exercise 2.60}
\label{sec:nielsen-and-chuang-exercise-2-60}
This section will only find the requested eigenvalues,
and the Projector given by $P_+$.
The projector given by $P_-$ can be found by following the same
steps as $P_+$'s solution.

\MakeUppercase{\underline{The eigenvalues}} of
$\vec{v} \vec{\sigma}$ can be found by using
basic Linear Algebra knowledge:
$det(\vec{v} \vec{\sigma} - \lambda I) = 0$.
Therefore, referring to Section \ref{sec:nielsen-and-chuang-equation-2-116},
calculate

\begin{align}
    det(\vec{v} \vec{\sigma} - \lambda I) &= 0 \\
    \left| \begin{matrix}
        v_3 - \lambda & v_1 - v_2i \\ v_1 + v_2i & -v_3 - \lambda
        \end{matrix} \right| &= 0 \\
    \notag \\
    \lambda^2 - v_3^2 - (v_1^2 + v_2^2) &= 0 \\
    \lambda^2 &= v_1^2 + v_2^2 + v_3^2
\end{align}

A bit of cleverness is required here.
Recall that just before the definition of
\hyperref[sec:nielsen-and-chuang-equation-2-116]{Equation 2.116},
$\vec{v}$ is supposed to be a unit vector.
This means that $\vec{v} \cdot \vec{v} = 1$.
Since $\vec{v} \in \mathbb{R}^3$),
$\vec{v} \cdot \vec{v} = v_1 \cdot v_1 +
v_2 \cdot v_2 + v_3 \cdot v_3$.
Therefore, $v_1^2 + v_2^2 + v_3^2 = 1$.
Plugging this into the previous result to find the values of $\lambda$:

\begin{align}
    \lambda^2 &= 1 \\
    \lambda &= \pm 1
\end{align}

as requested.

\MakeUppercase{\underline{To find the Projector}} $P_+$ it is necessary to
calculate the eigenspace of the eigenvector of $+1$.
In order to find the eigenspace, basic Linear Algebra knowledge may be used.
Hence, for $\lambda = +1$ and applying row reducing:

\begin{align}
    \left[ \begin{matrix}
        v_3 - \lambda & v_1 - v_2i \\ v_1 + v_2i & -v_3 - \lambda
        \end{matrix} \right]
        &=
        \left[ \begin{matrix}
            v_3 - 1 & v_1 - v_2i \\ v_1 + v_2i & -v_3 - 1
        \end{matrix} \right] \\ \notag \\
    &= \left[ \begin{matrix}
            (v_3 - 1)(v_1 + v_2i) & v_1^2 + v_2^2 \\
            (v_1 + v_2i)(v_3 - 1) & 1 -v_3^2
        \end{matrix} \right]
\end{align}

Note that $1 - v_3^2 = v_1^2 + v_2^2$,
since $v_1^2 + v_2^2 + v_3^2 = 1$. Therefore,

\begin{align}
    \left[ \begin{matrix}
            (v_3 - 1)(v_1 + v_2i) & v_1^2 + v_2^2 \\
            (v_1 + v_2i)(v_3 - 1) & 1 -v_3^2
        \end{matrix} \right]
        &=
        \left[ \begin{matrix}
            (v_3 - 1)(v_1 + v_2i) & v_1^2 + v_2^2 \\
            (v_3 - 1)(v_1 + v_2i) & v_1^2 + v_2^2
        \end{matrix} \right] \\ \notag \\
    &= \left[ \begin{matrix}
            (v_3 - 1)(v_1 + v_2i) & v_1^2 + v_2^2 \\
            0 & 0
        \end{matrix} \right]
\end{align}

Therefore, if $t$ is a scalar,
the eigenspace can be given by:
\begin{align}
    t \left[ \begin{matrix}
        \frac{-(v_1 - v_2i)}{v_3 - 1} \\ 1
    \end{matrix} \right]
\end{align}
because:
\begin{align}
    (v_3 - 1)(v_1 + v_2i) \cdot
    \frac{-(v_1 - v_2i)}{v_3 - 1} +
    (v_1^2 + v_2^2) \cdot 1 &= 0 \\
    -(v_1^2 + v_2^2) + (v_1^2 + v_2^2) &= 0
\end{align}

However, it is not possible to use the definition
and keep calculating with $P_m = \ketbra{m}{m}$,
where $\ket{m} = \left[ \begin{matrix}
        \frac{-(v_1 - v_2i)}{v_3 - 1} \\ 1
    \end{matrix} \right]$
because it is necessary that $\ket{m}$ is unitary
($\braket{m}{m} = 1$).
And, if $\braket{m}{m}$ is calculated, the following
result would be obtained:
\begin{align}
    \braket{m}{m} &= \left[ \begin{matrix}
        \frac{-(v_1 + v_2i)}{v_3 - 1} & 1
        \end{matrix} \right]
        %inner product
        \left[ \begin{matrix}
            \frac{-(v_1 - v_2i)}{v_3 - 1} \\ 1
        \end{matrix} \right] \\
    &= \left[ \begin{matrix}
        \frac{-v_1 - v_2i}{v_3 - 1} & 1
        \end{matrix} \right]
        %inner product
        \left[ \begin{matrix}
            \frac{-v_1 + v_2i}{v_3 - 1} \\ 1
        \end{matrix} \right] \\
    &= \frac{v_1^2 + v_2^2}{(v_3 - 1)^2} + 1 \\
    &= \frac{1 - v_3^2}{{(v_3 - 1)^2}} +
        \frac{{(v_3 - 1)^2}}{{(v_3 - 1)^2}} \\
    &= \frac{1 - v_3^2 + v_3^2 - 2v_3 + 1}
        {(v_3 - 1)^2} \\
    &= \frac{-2v_3 + 2}{(v_3 - 1)^2} \\
    &= \frac{-2 (v_3 - 1)}{(v_3 - 1)^2} \\
    &= - \frac{2}{v_3 - 1}
\end{align}

Hence, it is necessary to normalize $\ket{m}$.
Recall that the norm of a vector $\ket{m}$ is given by
$\sqrt{\braket{m}{m}}$.
To normalize a vector, divide it by its norm.
\begin{align}
    \ket{\psi} &= \frac{\ket{m}}{\sqrt{\braket{m}{m}}} \\
    &= \ket{m} / \sqrt{- \frac{2}{v_3 - 1}} \\
    &= \sqrt{- \frac{v_3 - 1}{2}} \ket{m} \\
    &= \frac{i}{\sqrt2} \sqrt{v_3 - 1} \ket{m}
\end{align}

However, this would not be right because
$\sqrt{\braket{m}{m}} \geq 0$ and
$\sqrt{\braket{m}{m}} \in \mathbb{R}$.
It is necessary to rearrange the value of $\braket{m}{m}$:
\begin{align}
    \ket{\psi} &= \sqrt{- \frac{v_3 - 1}{2}} \ket{m} \\
    &= \sqrt{\frac{1 - v_3}{2}} \ket{m} \\
    &= \frac{1}{\sqrt2} \sqrt{1 - v_3}
        \left[ \begin{matrix}
            \frac{-(v_1 - v_2i)}{v_3 - 1} \\ 1
        \end{matrix} \right] \\
    &= \frac{1}{\sqrt2} \sqrt{1 - v_3}
        \left[ \begin{matrix}
            \frac{v_1 - v_2i}{1 - v_3} \\ 1
        \end{matrix} \right] \\
    &= \frac{1}{\sqrt2} \left[ \begin{matrix}
            \frac{v_1 - v_2i}{\sqrt{1 - v_3}} \\
            \sqrt{1 - v_3}
        \end{matrix} \right]
\end{align}

Now that the normalized vector was obtained,
it is possible to calculate the respective Projector by:
\begin{align}
    \ketbra{\psi}{\psi} &= \frac{1}{\sqrt2}
        \left[ \begin{matrix}
            \frac{v_1 - v_2i}{\sqrt{1 - v_3}} \\
            \sqrt{1 - v_3}
        \end{matrix} \right]
        %times
        \frac{1}{\sqrt2}
        \left[ \begin{matrix}
            \frac{v_1 + v_2i}{\sqrt{1 - v_3}} &
            \sqrt{1 - v_3}
        \end{matrix} \right] \\
    &= \frac{1}{2} \left[ \begin{matrix}
            \frac{v_1 - v_2i}{\sqrt{1 - v_3}} \\
            \sqrt{1 - v_3}
        \end{matrix} \right]
        %times
        \left[ \begin{matrix}
            \frac{v_1 + v_2i}{\sqrt{1 - v_3}} &
            \sqrt{1 - v_3}
        \end{matrix} \right] \\
    &= \frac{1}{2} \left[ \begin{matrix}
        \frac{v_1^2 + v_2^2}{1 - v_3} & v_1 - v_2i \\
        v_1 + v_2i & 1 - v_3
        \end{matrix} \right] \\
    &= \frac{1}{2} \left[ \begin{matrix}
        \frac{1 - v_3^2}{1 - v_3} & v_1 - v_2i \\
        v_1 + v_2i & 1 - v_3
        \end{matrix} \right] \\
    &= \frac{1}{2} \left[ \begin{matrix}
        \frac{(1 + v_3)(1 - v_3)}{1 - v_3} & v_1 - v_2i \\
        v_1 + v_2i & 1 - v_3
        \end{matrix} \right] \\
    &= \frac{1}{2} \left[ \begin{matrix}
        1 + v_3 & v_1 - v_2i \\
        v_1 + v_2i & 1 - v_3
        \end{matrix} \right] \\
    &= (I + \vec{v} \vec{\sigma}) / 2 = P_+
\end{align}

as requested.
$P_-$ can be easily obtained following the same steps,
but with $\lambda = -1$.

%%%%%%%%%%%%%%%%%%%%%%%%%%%%%%%%%%%%%%%%%%%%%%%%%%%%%%%%%%%%%%%%%%%%%%%%%%%%
\subsubsection{Exercise 2.61}
This Exercise can be done very easily by using Equations (2.103) and (2.104)
alongside the value of $P_+$ obtained in
\hyperref[sec:nielsen-and-chuang-exercise-2-60]{Exercise 2.60}.

The probability can be calculated by using Equation (2.103):
\begin{align}
    p(+1) &= \bra{0}P_+\ket{0} \\
    &= [ \begin{matrix} 1 & 0 \end{matrix} ]
        \ (I + \vec{v} \vec{\sigma}) / 2
        \ \left[ \begin{matrix} 1 \\ 0 \end{matrix} \right] \\
    &= \frac{1}{2} \left[ \begin{matrix} 1 + v_3 & v_1 - v_2i \end{matrix} \right]
        \left[ \begin{matrix} 1 \\ 0 \end{matrix} \right] \\
    &= \frac{1 + v_3}{2}
\end{align}

Then, using Equation (2.104) to obtain the state of the system
after the measurement:
\begin{align}
    \ket{\psi} &= \frac{P_+ \ket{0}}{\sqrt{p(+1)}} \\
    &= \frac{1}{\sqrt{p(+1)}} \frac{1}{2} (I + \vec{v} + \vec{\sigma})
        \left[ \begin{matrix} 1 \\ 0 \end{matrix} \right] \\
    &= \frac{\sqrt{2}}{\sqrt{1 + v_3}} \frac{1}{2}
        \left[ \begin{matrix} v_3 + 1 \\ v_1 + v_2i \end{matrix} \right] \\
    \notag \\
    &= \frac{(v_3 + 1)\ket{0} + (v_1 + v_2i)\ket{1}}{\sqrt{2 + 2v_3}}
\end{align}
\subsection{Section 2.2.8}
\subsubsection{Equation (2.123)}

\emph{I would like to thank
    \href{http://buscatextual.cnpq.br/buscatextual/visualizacv.do?id=K4774458A4}{Rex}
    (\href{mailto:rexmedeiros@ect.ufrn.br}{rexmedeiros@ect.ufrn.br})
    and
    \href{http://buscatextual.cnpq.br/buscatextual/visualizacv.do?id=K4733964Y9&idiomaExibicao=2}{LIB} 
    (\href{mailto:leandro@ect.ufrn.br}{leandro@ect.ufrn.br})
    for helping me to understand this equation. The present subsection mixes some doubts I had alongside with their explanation.
}

The definition of equation (2.122) will be needed for this section. In order to understand equation (2.123), it is necessary to recall the definition of inner product \footnote{For more details, refer to Nielsen and Chuang's section 2.1.4} between two states $\ket{\psi}$ and $\ket{\varphi}$:

\[
(\ket{\varphi}, \ket{\psi}) = \ket{\varphi}^\dagger \ket{\psi} = \braket{\varphi}{\psi}
\]

However, the inner product on equation (2.123) is a composite system inner product. Since composite systems are described using tensor products, it is necessary to apply the definition of equation (2.49). Hence, it is possible to calculate

\begin{align}
    \left( U \ket{\varphi} \ket{0}, U \ket{\psi} \ket{0} \right) &= 
    \left( \sum_m M_m \ket{\varphi} \ket{m} , \sum_{m'.} M_{m'} \ket{\psi} \ket{m'} \right)
    \\
    &= \sum_{m, m'} (M_m\ket{\varphi})^\dagger M_{m'.}\ket{\psi} \braket{m}{m'}
\end{align}

Then, from the definitions on section 2.1.6:
\begin{align}
    \sum_{m, m'} (M_m\ket{\varphi})^\dagger M_{m'.}\ket{\psi} \braket{m}{m'} =
    \sum_{m, m'} \bra{\varphi} M_m^\dagger M_{m'.}\ket{\psi} \braket{m}{m'}
\end{align}


The left side of equation (2.123) may be rather confusing, however. Because according to the definitions on section 2.16 $(U \ket{\varphi 0})^\dagger = \bra{\varphi 0} U^\dagger$. Also, accordingly to the properties on equation (2.53)  $(U \ket{\varphi} \ket{0})^\dagger = \bra{\varphi} \bra{0} U^\dagger$. If this line of thought was followed, then equation

\[
\bra{\varphi} \bra{0} U^\dagger U \ket{\psi} \ket{0} =
\sum_{m,m'} \bra{\varphi}\bra{m} M^\dagger_m M_{m'} \ket{\psi} \ket{m'}
\]

would be obtained. Which would not match equation (2.49)'s definition.

It is a common practice in Physics, however, to write $(U\ket{\varphi} \ket{0})^\dagger = \bra{0} \bra{\varphi} U^\dagger$. In this case, the adjoint operators are read 'backwards'. So, for instance, $U$ operates on $\ket{\varphi}$ (i.e. $U\ket{\varphi}$); while $U^\dagger$ operates on $\bra{\varphi}$ (i.e. $\bra{\varphi} U^\dagger$). Following this line of thought, $(U\ket{\varphi} \ket{0})^\dagger = \bra{\varphi} \bra{0} U^\dagger$ would not make sense because $U^\dagger$ should operate on $\bra{\varphi}$, not on $\bra{0}$. Formally, imagine that an operator $M$ operates on vector space $V$, $\ket{v} \in V$ and $\ket{w} \in W$, then $\bra{v} \bra{w} M^\dagger$ would not be valid because $M$ only acts on vector space $V$, not $W$.

Hence, it is possible to rewrite equation (2.123) as:

\begin{align}
    (U \ket{\varphi} \ket{0}, U \ket{\psi} \ket{0}) &= 
    (U \ket{\varphi} \ket{0})^\dagger U \ket{\psi} \ket{0}
    \\ 
    &= (\sum_m M_m \ket{\varphi} \ket{m})^\dagger \sum_{m'} M_{m'} \ket{\psi} \ket{m'}
    \\
    &= \sum_{m, m'} \bra{m} \bra{\varphi} M_m^\dagger M_{m'} \ket{\psi} \ket{m'}
\end{align}

since $\bra{\varphi} M_m^\dagger M_{m'} \ket{\psi}$ is a scalar:

\begin{align}
    \sum_{m, m'} \bra{m} \bra{\varphi} M_m^\dagger M_{m'} \ket{\psi} \ket{m'} =
    \sum_{m, m'} \bra{\varphi} M_m^\dagger M_{m'} \ket{\psi} \braket{m}{m'}
\end{align}

Which is another way to obtain equation (2.123).
\subsection{Section 2.5}
\subsubsection{Symmetry of $(\ket{00} + \ket{01} + \ket{11})/\sqrt{3}$}

This subsection is dedicated to calculate $tr((\rho^A)^2)$
for $(\ket{00} + \ket{01} + \ket{11})/\sqrt{3}$. By Equation (2.138):

\begin{align}
    \rho^{AB} &= \frac{(\ket{00} + \ket{01} + \ket{11})}{\sqrt{3}}
        \frac{(\bra{00} + \bra{01} + \bra{11})}{\sqrt{3}} \\
        &= \frac{(\ket{00} + \ket{01} + \ket{11})(\bra{00} + \bra{01} + \bra{11})}{3}
\end{align}

Before using Equations (2.177) and (2.178) it is necessary to apply the distributive property.
However from Equation (2.178), the result will be similar to
$\sum_{ijkl} \ket{i}\bra{j} tr(\ket{k}\bra{l})$, where $i, j, k, l \in \{0, 1\}$.
Since $tr(\ket{a}\bra{b}) = \bra{b}\ket{a}$:
\begin{align}
    \sum_{ijkl} \ket{i}\bra{j} tr(\ket{k}\bra{l}) &=
        \sum_{ijkl} \ket{i}\bra{j}\braket{l}{k} \\
        &= \sum_{ijkl} \ket{i}\bra{j}\delta_{lk}
\end{align}

Therefore, when applying the distributive property,
it is not necessary to write $\ket{i}\bra{j}$ if $l \neq k$.
For instance,
$tr_B(\ket{00}\bra{01}) = \ket{0}\bra{0}tr(\ket{0}\bra{1})$
$= \ket{0}\bra{0}\braket{0|1} = 0 \ket{0}\bra{0}$.
Also, since $\braket{i|i} = 1$:
\begin{align}
    \rho^A &= \frac{\ket{0}\bra{0} + \ket{0}\bra{0} + \ket{0}\bra{1} +
        \ket{1}\bra{0} + \ket{1}\bra{1}}{3} \\
    &= \frac{2\ket{0}\bra{0} + \ket{0}\bra{1} +
        \ket{1}\bra{0} + \ket{1}\bra{1}}{3}
\end{align}

Now, calculating $(\rho^A)^2$:
\begin{align}
    (\rho^A)^2 &= \frac{(2\ket{0}\bra{0} + \ket{0}\bra{1} +
        \ket{1}\bra{0} + \ket{1}\bra{1})\ (2\ket{0}\bra{0} + \ket{0}\bra{1} +
        \ket{1}\bra{0} + \ket{1}\bra{1})}{3 \cdot 3} \\
    &= \frac{4\ket{0}\bra{0} + 2\ket{0}\bra{1} + \ket{0}\bra{0} + \ket{0}\bra{1} +
        \ket{1}\bra{1} + \ket{1}\bra{0} + \ket{1}\bra{1}}{9} \\
    &= \frac{5\ket{0}\bra{0} + 3\ket{0}\bra{1} + \ket{1}\bra{0} + 2\ket{1}\bra{1}}{9}
\end{align}

Now, calculate $tr((\rho^A)^2)$:
\begin{align}
    tr((\rho^A)^2) &= \frac{1}{9}\ tr(5\ket{0}\bra{0} + 3\ket{0}\bra{1} +
        \ket{1}\bra{0} + 2\ket{1}\bra{1}) \\
    &= \frac{1}{9}\ (5\braket{0|0} + 3\braket{1|0} +
        \braket{0|1} + 2\braket{1|1}) \\
    &= \frac{1}{9}\ (5 \cdot 1 + 3 \cdot 0 + 0 + 2 \cdot 1) \\
    &= \frac{1}{9}\ (5 + 2) \\
    &= \frac{7}{9}
\end{align}

Using an analogous line of thought $tr((\rho^B)^2) = \frac{7}{9}$ is obtained.

%%%%%%%%%%%%%%%%%%%%%%%%%%%%%%%%%%%%%%%%%%%%%%%%%%%%%%%%%%%%%%%%%%%%%%%%%%%%%%%%%%%%%%%
\subsubsection{Exercise 2.82}

\begin{enumerate}
\item From the definition of purification, it is necessary to prove that $\rho^A = tr_R(\ket{AR}\bra{AR})$.
    
    Suppose $\sum_i \sqrt(p_i) \ket{\psi_i} \ket{i}$ is a purification. From equation (2.138) of Nielsen and Chuang's book:
    \begin{align}
        \rho^{AB} = \sum_{ij} \left( \sqrt{p_i} \ket{\psi_i} \ket{i} \right) \left( \sqrt{p_j} \ket{\psi_j} \ket{j} \right) ^ \dagger
    \end{align}
    
    Since $p_j \in \mathbb{R}$, $p_j^\dagger = p_j$. And since it is a scalar:
    \begin{align}
        \rho^{AB} = \sum_{ij} \sqrt{p_i p_j} \left( \ket{\psi_i} \ket{i} \right) \left( \ket{\psi_j} \ket{j} \right) ^ \dagger
    \end{align}
    
    Recall that for any states $\ket{\varphi}$ and $\ket{\gamma}$ writing $\ket{\varphi} \ket{\gamma}$ is the same as $\ket{\varphi} \otimes \ket{\gamma}$. Then, applying equation (2.48):
    \begin{align}
        \rho^{AB} &= \sum_{ij} \sqrt{p_i p_j} \left( \ket{\psi_i} \otimes \ket{i} \right)
            \left( \bra{\psi_j} \otimes \bra{j} \right) \\
        \rho^{AB} &= \sum_{ij} \sqrt{p_i p_j} (\ket{\psi_i}\bra{\psi_j}) \otimes (\ket{i}\bra{j})
    \end{align}
    
    If $\sum_i \sqrt{p_i} \ket{\psi_i}\ket{i}$ is a purification, then
    $\rho^A = tr_B(\ (\ket{\psi_i}\ket{i}) (\bra{\psi_i}\bra{i})\ )$.
    The question gives $\rho = \sum_i p_i \ket{\psi_i}\bra{\psi_i}$, in other words $\rho = \rho^A$.
    
    Now, it is necessary to calculate $tr_B(\ (\ket{\psi_i}\ket{i}) (\bra{\psi_i}\bra{i})\ )$. Then, using the definitions given in equations (2.177) and (2.178) of Nielsen and Chuang's book:
    \begin{align}
        tr_B(\rho^{AB}) &= tr_B \left(\sum_{ij} \sqrt{p_i p_j} (\ket{\psi_i}\bra{\psi_j}) \otimes (\ket{i}\bra{j}) \right) \\
        &= \sum_{ij} \sqrt{p_i p_j} (\ket{\psi_i}\bra{\psi_j}) tr(\ket{i}\bra{j})
    \end{align}
    
    Since $tr(\ket{a}\bra{b}) = \braket{b}{a}$:
    \begin{align}
        tr_B(\rho^{AB}) &= \sum_{ij} \sqrt{p_i p_j}\ (\ket{\psi_i}\bra{\psi_j}) \braket{j}{i} \\
        &= \sum_{ij} \sqrt{p_i p_j}\ \ket{\psi_i}\bra{\psi_j}\ \delta_{ij} \\
        &= \sum_i \sqrt{p_i p_i}\ \ket{\psi_i}\bra{\psi_i} \\
        &= \sum_i p_i\ket{\psi_i}\bra{\psi_i}
    \end{align}
    
    Since $\rho^A = \sum_i p_i \ket{\psi_i}\bra{\psi_i} = tr_B(\rho^{AB})$,
    it is possible to conclude that $\sum_i \sqrt{p_i} \ket{\psi_i}\ket{i}$ is a purification.

%%%%%%%%%%%%%%%%%%%%%%%%%%%%%%%%%%%%%%%%%%%%%%%%%%%%%%%%%%%%%%%%%%%%%%

\item \todo{Add intuitive solution/explanation}

    For this exercise, it is necessary to review Postulate 3. For the system $\ket{AR}$, there is a set of measurement operators $\{M_m\}$. Only system $R$ is being measured, though. So, it is possible to define every measurement operator $M_m = I \otimes M_i$ where $I$ is the identity operator acting on system $A$ and $M_i$ is the measurement operator acting on system $R$ which corresponds to measuring the state $\ket{i}$.
    
    The probability of measuring $\ket{i}$ is requested, i.e. $p(i)$. Using $\ket{\varphi_i} = \sqrt{p_i}\ket{\psi_i} \ket{i}$ temporarily for simplicity and by Equation (2.92):
    \begin{align}
        p(i) &= \bra{\varphi_i}M_m^\dagger M_m \ket{\varphi_i} \\
        &= \bra{\varphi_i}(I \otimes M_i)^\dagger (I \otimes M_i) \ket{\varphi_i} \\
        &= \sqrt{p_i} \bra{\psi_i}\bra{i}(I \otimes M_i)^\dagger
            (I \otimes M_i) \sqrt{p_i}\ket{\psi_i}\ket{i} \\
        &= p_i \bra{\psi_i}\bra{i}(I^\dagger \otimes M_i^\dagger)
            (I \otimes M_i) \ket{\psi_i}\ket{i} \\
        &= p_i \bra{\psi_i}\bra{i}(I \otimes M_i^\dagger)
            (I \otimes M_i) \ket{\psi_i}\ket{i}
    \end{align}
    
    Using equation (2.48):
    \begin{align}
        p(i) &= p_i (\bra{\psi_i}I \otimes \bra{i}M_i^\dagger)
            \ (I\ket{\psi_i} \otimes M_i\ket{i}) \\
        &= p_i (\bra{\psi_i} \otimes \bra{i}M_i^\dagger )
            \ (\ket{\psi_i} \otimes M_i\ket{i})
    \end{align}
    
    Then, by the definition of inner product (equation (2.49)):
    \begin{align}
        p(i) = p_i \braket{\psi_i}{\psi_i} \bra{i}M_i^\dagger M_i\ket{i}
    \end{align}
    
    Recall that $\ket{\psi_i}$ and $\ket{i}$ are orthonormal. Also, since $M_i$ is the measurement operator that corresponds to obtaining state $\ket{i}$, it is possible to consider $\ket{i}$ as a "measurement basis" defining $M_i = \ket{i}\bra{i}$. A similar example can be seen in Nielsen and Chuang's book in a paragraph between Equations (2.95) and (2.96). Therefore,
    \begin{align}
        p(i) &= p_i \cdot 1 \cdot \bra{i}(\ket{i}\bra{i})^\dagger (\ket{i}\bra{i}) \ket{i} \\
        &= p_i \bra{i}(\ket{i}\bra{i}) (\ket{i}\bra{i}) \ket{i} \\
        &= p_i \braket{i}{i} \braket{i}{i} \braket{i}{i} \\
        &= p_i \cdot 1 \cdot 1 \cdot 1 \\
        &= p_i
    \end{align}
    
    Hence, the probability of measuring state $\ket{i}$ is $p_i$. Now, it is requested to obtain the state of system A after the measurement $M_m$, which is described by Postulate 3 as:
    \begin{align}
        \frac{M_m \ket{\varphi_i}}{\sqrt{p_i}} &=
            \frac{(I \otimes M_i) \sqrt{p_i}\ket{\psi_i}\ket{i}}{\sqrt{p_i}} \\
        &= (I\ket{\psi_i}) \otimes (M_i\ket{i}) \\
        &= (I\ket{\psi_i}) \otimes (\ket{i}\braket{i|i}) \\
        &= \ket{\psi_i} \ket{i}
    \end{align}
    
    Therefore, the measurement of the system A will always be $\ket{\psi_i}$.


%%%%%%%%%%%%%%%%%%%%%%%%%%%%%%%%%%%%%%%%%%%%%%%%%%%%%%%%%%%%%%%%%%%%%%
\item
    \href{https://github.com/goropikari}{Goropikari} attempted to solve this exercise as follows\footnote
    {The original code can be found at \href{https://github.com/goropikari/SolutionForQuantumComputationAndQuantumInformation/blob/master/chapter/chapter2.tex}
    {SolutionForQuantumComputationAndQuantumInformation}}:

    \begin{quotation}
        Suppose $\ket{AR}$ is a purification of $\rho$ such that $\ket{AR} = \sum_i \sqrt{p_i} \ket{\psi_i} \ket{r_i}$.
        By exercise 2.81, the others purification is written as $(I \otimes U) \ket{AR}$.
        \begin{align*}
        	(I \otimes U)  \ket{AR} &= (I \otimes U) \sum_i \sqrt{p_i} \ket{\psi_i} \ket{r_i}\\
        		&= \sum_i \sqrt{p_i} \ket{\psi_i} U\ket{r_i}\\
        		&= \sum_i \sqrt{p_i} \ket{\psi_i} \ket{i}
        \end{align*}
        where $U = \sum_i \ket{i}\bra{r_i}$.
        
        By (2), if we measure the system $R$ w.r.t $\ket{i}$, post-measurement state for system $A$ is $\ket{\psi_i}$ with probability $p_i$, which prove the assertion.
    \end{quotation}
    
    \todo{Is the previous solution plausible in same way?}
    
    However, if system $R$ is measured with respected to $\ket{i}$
    (that is, the measurement operator $M_m = I \otimes \ket{i}\bra{i}$
    is applied to $\sum_i \sqrt{p_i} \ket{\psi_i}\ket{i}$)
    the same result of Exercise 2.82(2) will be achieved.
    
    A similar way to try to solve this problem is:
    define $\ket{AR_i} = \ket{\psi_i}\ket{r_i}$
    and use the measurement operator $M_m = I \otimes \ket{i}\bra{i}$.
    Then, the probability of measuring $\ket{i}$ is calculated
    according to Equation (2.92):
    \begin{align}
        p(i) &= (\sqrt{p_i}\ (I \otimes \ketbra{i}{i}) \ket{\psi_i}\ket{r_i})^\dagger
            \ \sqrt{p_i}\ (I \otimes \ketbra{i}{i}) \ket{\psi_i}\ket{r_i} \\
        &= p_i \bra{r_i}\bra{\psi_i} (\ketbra{i}{i} \otimes I)
            \ (I \otimes \ketbra{i}{i}) \ket{\psi_i}\ket{r_i}
    \end{align}
    
    Then, using Equations (2.48) and (2.49):
    \begin{align}
        p(i) &= p_i  (\bra{r_i} \ketbra{i}{i}) \otimes (\bra{\psi_i} I)
            \ (I \ket{\psi_i}) \otimes (\ketbra{i}{i} \ket{r_i}) \\
        &= p_i (\braket{r_i}{i} \bra{i}) \otimes \bra{\psi_i}
            \ \ket{\psi_i} \otimes (\braket{i}{r_i} \ket{i}) \\
        &= p_i \braket{r_i}{i} \braket{i}{r_i}\ (\bra{i} \bra{\psi_i} \ket{\psi_i} \ket{i}) \\
        &= p_i\ \norm{\braket{i}{r_i}}^2
    \end{align}
    
    if $\norm{\braket{i}{r_i}}^2 = 1/p_i$, the desired result would obtained.
    However, $0 < p_i < 1$, then $1/p_i > 1$, which is not possible
    since both $\ket{i}$ and $\ket{r_i}$ are orthonormal vectors.
    Independently, if the post-measurement state is calculated as given by Equation (2.93):
    \begin{align}
        \frac{M_m\ket{\psi}}{\sqrt{\bra{\psi}M_m^\dagger M_m\ket{\psi}}} &=
            \frac{(I \otimes \ketbra{i}{i})\sqrt{p_i}\ket{\psi_i}\ket{r_i}}
            {\sqrt{p_i \norm{\braket{i}{r_i}}^2}} \\
        &= \frac{\sqrt{p_i} \ket{\psi_i} \ketbra{i}{i} \ket{r_i}}
            {\sqrt{p_i}\ \norm{\ketbra{i}{r_i}}} \\
        &= \frac{\braket{i}{r_i} \ket{\psi_i} \ket{i}}{\norm{\braket{i}{r_i}}}
    \end{align}
    
    Therefore, $\frac{\braket{i}{r_i}}{\sqrt{\braket{i}{r_i}\braket{r_i}{i}}} = p_i$.
    \todo{In conclusion ?????}
    
\end{enumerate}
\subsection{Chapter 2 Problems}
\subsubsection{Problem 2.1}
This problem can be solved easily by combining the
logic of Exercises
\hyperref[sec:nielsen-and-chuang-exercise-2-35]{2.35} and
\hyperref[sec:nielsen-and-chuang-exercise-2-60]{2.60}.

It is known from exercise \hyperref[sec:nielsen-and-chuang-exercise-2-60]{2.60}
that $\vec{n} \vec{\sigma}$ has spectral decomposition
$+1P_+ - 1P_- = \\ +1 \left( \frac{I + \vec{n}\vec{\sigma}}{2} \right)
-1 \left( \frac{I - \vec{v}\vec{\sigma}}{2} \right)$.
Therefore, $\theta \vec{n} \vec{\sigma} =
\theta \left( \frac{I + \vec{n}\vec{\sigma}}{2} \right)
- \theta \left( \frac{I - \vec{v}\vec{\sigma}}{2} \right)$.
Then, by applying the definition of function operators and
the distributive property:
%
\begin{align}
    f(\theta \vec{n} \vec{\sigma}) &= f(\theta) \left(
        \frac{I + \vec{n}\vec{\sigma}}{2} \right) +
        f(-\theta) \left( \frac{I - \vec{v}\vec{\sigma}}{2} \right) \\
    &= \frac{f(\theta)}{2}I + \frac{f(-\theta)}{2}I +
        \frac{f(\theta)}{2}\vec{n}\vec{\sigma} -
        \frac{f(-\theta)}{2}\vec{n}\vec{\sigma} \\
    &= \frac{f(\theta) + f(-\theta)}{2}I +
        \frac{f(\theta) - f(-\theta)}{2}\vec{n}\vec{\sigma}
\end{align}

%%%%%%%%%%%%%%%%%%%%%%%%%%%%%%%%%%%%%%%%%%%%%%%%%%%%%%%%%%%%%%%%%%%%%%%%%%%%%%%%%%%%%%%

\subsubsection{Problem 2.2}

\begin{enumerate}
    \item Since $\ket{\psi}$ is pure, it follows from Theorem 2.7, that
    $\ket{\psi} = \sum_i \lambda_i \ket{i_A} \ket{i_B}$.
    In addition, $\rho^A = \sum_i \lambda_i^2 \ketbra{i_A}{i_A}$.
    From the rank-nullity theorem, it is known that
    $rank(A) + nullity(A) = dim(A)$ where $dim(A)$ is the dimension of the matrix $A$.
    Also, $rank(A) = dim(Col(A))$ where $dim(Col(A))$ is the dimension of the columnspan of $A$.
    Notwithstanding, $row(A) \cup kernel(A)$ spans $\mathbb{C}^n$, and is linearly independent.
    Thus, forming a basis for $\mathbb{C}^n$.

    To prove that $rank(\rho^A) = Sch(\ket{\psi})$,
    it is possible to use the rank-nullity theorem.
    Henceforth, find $kernel(\rho^A)$,
    that is, $\set{\ket{x} \in \mathbb{C}^n}$ such that $\rho^A \ket{x} = \vec{0}$.
    Assume the possibility that $\exists k, \lambda_k = 0$.
    Also, since $\ket{i_A}$ spans $\mathbb{C}^n$,
    any vector $\ket{x}$ can written as a linear combination of $\ket{i_A}$.
    Then,
    %
    \begin{align}
        &\left( \sum_i \lambda_i^2 \ketbra{i_A}{i_A} \right) \ket{x}
        = \left( \sum_i \lambda_i^2 \ketbra{i_A}{i_A} \right) \sum_j x_j \ket{j_A}
        = \sum_{ij} \lambda_i^2 x_j \ket{i_A} \braket{i_A}{j_A} \\
        %
        &= \sum_j \lambda_j^2 x_j \ket{j_A}
        = \sum_{j \neq k} \lambda_j^2 x_j \ket{j_A} + \sum_k \lambda_k^2 x_k \ket{k_A}
        = \sum_{j \neq k} \lambda_j^2 x_j \ket{j_A} + \sum_k 0 x_k \ket{k_A}
    \end{align}
    
    Therefore, $kernel(\rho^A) = \ket{k_A}$ where $\forall k, \lambda_k = 0$.
    As such, $Sch(\ket{\psi}) = n - dim(kernel(\rho^A)) = n - nullity(\rho^A)$.
    Then, using the rank-nullity theorem conclude that $rank(\rho^A) = Sch(\ket{\psi})$.

    \dotfill

    \item To prove that $j > Sch(\psi)$ note that if $j > 1$,
    then it is possible to obtain an orthonormal basis for $\ket{\psi}$
    such that $\ket{\psi}$ is equal to the tensor product of
    exactly one element of each subsystem's basis
    ($\ket{\psi} = \ket{a} \otimes \ket{b}$).
    
    To prove that $j = Sch(\psi)$, suppose that $\ket{\alpha_j}$ and $\ket{\beta_j}$
    are linearly independent (thus forming a basis for a $\ket{\psi}$).
    Then, obtain bases $\ket{a_i}$ and $\ket{b_i}$ through Gram-Schmidt.
    Therefore, $Sch(\psi) = i = j$.
    
    In addition, no restriction is given regarding the number of $j$ elements.
    Since they are not normalised, it is possible that
    $\ket{\alpha_j}$ and $\ket{\beta_j}$ are not linearly independent.
    Thus, by obtaining an orthonormal basis $\ket{a_i}$ and $\ket{b_i}$ from
    $\ket{\alpha_j}$ and $\ket{\beta_j}$ via Gram-Schmidt, $j > i$.
    Also proving that $j > Sch(\psi)$.
    
    \dotfill
    
    \item The solution when $\alpha = 0$ or $\beta = 0$ is trivial,
    since $Sch(\psi) = max(Sch(\varphi), Sch(\gamma))$.
    Henceforth, assume that $\alpha \neq 0$ and $\beta \neq 0$.
    Also, assume that $\ket{\varphi} \neq \mu \ket{\gamma}$,
    $\mu \in \mathbb{C}$.
    
    $\ket{\psi}$ is a pure state of $A$ and $B$,
    and it is a linear combination of $\ket{\varphi}$ and $\ket{\gamma}$.
    Also, note that $\ket{\psi}$, $\ket{\varphi}$, and $\ket{\gamma}$
    must be written in the same orthonormal basis
    $\ket{a_i}$ for subsystem $A$ and $\ket{b_i}$ analogously for $B$.
    Therefore,
    %
    \begin{align}
        \ket{\psi} &= \ket{\psi} \\
        \sum_i \lambda_i \ket{a_i} \ket{b_i} &= \alpha \ket{\varphi} + \beta \ket{\gamma} \\
        \sum_i \lambda_i \ket{a_i} \ket{b_i} &=
            \alpha \sum_i \varphi_i \ket{a_i} \ket{b_i} + \beta \sum_i \gamma_i \ket{a_i} \ket{b_i} \\
    \end{align}
    
    Where $\varphi_i, \gamma_i \in \mathbb{C}$ respectively are the indexes of the linear combinations
    of $\ket{\varphi}$ and $\ket{\gamma}$, according to the basis $\ket{a_i} \ket{b_i}$.
    Then,
    %
    \begin{align}
        \sum_i \lambda_i \ket{a_i} \ket{b_i} &=
            \sum_i (\alpha \varphi_i + \beta \gamma_i) \ket{a_i} \ket{b_i} \\
        \lambda_i &= \alpha \varphi_i + \beta \gamma_i
    \end{align}
    
    Using set theory, define $\Psi \equiv \set{i \mid \lambda_i \neq 0}$,
    $\Phi \equiv \set{i \mid \varphi_i \neq 0}$, and
    $\Gamma \equiv \set{i \mid \gamma_i \neq 0}$.
    Note that it is possible that $\alpha \varphi_i + \beta \gamma_i = 0$,
    thus, it is useful to define another set
    $\Delta \equiv \set{i \mid \alpha \varphi_i + \beta \gamma_i = 0
    \land \varphi_i \neq 0 \land \gamma_i \neq 0}$
    (where $\land$ indicates logical conjunction).
    Then, $Sch(\psi) = |\Psi|$, $Sch(\varphi) = |\Phi|$, and $Sch(\gamma) = |\Gamma|$.
    Therefore,
    %
    \begin{align}
        \Psi &= (\Phi \cup \Gamma) - \Delta \\
        |\Psi| &= |(\Phi \cup \Gamma)| - |\Delta|
    \end{align}
    
    Note that whenever
    $(\forall \varphi_i \neq 0, \alpha \varphi_i + \beta \gamma_i = 0) \lor
    (\forall \gamma_i \neq 0, \alpha \varphi_i + \beta \gamma_i = 0)$,
    then $max(|\Delta|) = min(|\Phi|, |\Gamma|)$,
    where $\lor$ indicates logical disjunction.
    Thus, $Sch(\psi) = |Sch(\varphi) - Sch(\gamma)|$.
    Also, $min(|\Delta|) = 0$ whenever
    $\forall \varphi_i \neq 0 \land \gamma_i \neq 0,\ \alpha \varphi_i + \beta \gamma_i \neq 0$.
    Thus, $Sch(\psi) > |Sch(\varphi) - Sch(\gamma)|$.
    Therefore, $Sch(\psi) \geq |Sch(\varphi) - Sch(\gamma)|$, as required.
    
\end{enumerate}

%%%%%%%%%%%%%%%%%%%%%%%%%%%%%%%%%%%%%%%%%%%%%%%%%%%%%%%%%%%%%%%%%%%%%%%%%%%%%%%%%%%%%%%

\subsubsection{Problem 2.3}

$Q$, $R$, $S$, $T$ are unitary matrices.
Referring back to
\hyperref[sec:nielsen-and-chuang-equation-2-116]{Equation 2.116},
%
\begin{align}
    Q = \left[ \begin{matrix}
        q_3 & q_1 - q_2i \\
        q_1 + q_2i & - q_3
        \end{matrix} \right]
\end{align}

Note that $Q = Q^\dagger$ and that
%
\begin{align}
    Q Q^\dagger &= \left[ \begin{matrix}
        q_3 & q_1 - q_2i \\
        q_1 + q_2i & - q_3
        \end{matrix} \right]
        %
        \left[ \begin{matrix}
        q_3 & q_1 - q_2i \\
        q_1 + q_2i & - q_3
        \end{matrix} \right] \\[5pt]
    &= \left[ \begin{matrix}
        q_1^2 + q_2^2 + q_3^2 & q_3(q_1 - q_2i) - q_3(q_1 - q_2i) \\
        q_3(q_1 - q_2i) - q_3(q_1 - q_2i) & q_1^2 + q_2^2 + q_3^2
        \end{matrix} \right]
\end{align}

Since $\vec{q} \in \mathbb{R}^3$ is a unit vector,
$q_1^2 + q_2^2 + q_3^2 = 1$,
%
\begin{align}
    Q Q^\dagger = I
\end{align}

Thus proving that $Q$ is an unitary matrix
(and analogously for $R$, $S$, and $T$).

Calculating $(Q \otimes S + R \otimes S + R \otimes T - Q \otimes T)^2$,
and writing it as $M^2$
%
\begin{align}
    M^2 =
    &(Q \otimes S)(Q \otimes S) + (Q \otimes S)(R \otimes S) +
    (Q \otimes S)(R \otimes T) - (Q \otimes S)(Q \otimes T) +
    \nonumber \\
    & (R \otimes S)(Q \otimes S) + (R \otimes S)(R \otimes S) +
    (R \otimes S)(R \otimes T) - (R \otimes S)(Q \otimes T) +
    \nonumber \\
    &(R \otimes T)(Q \otimes S) + (R \otimes T)(R \otimes S) +
    (R \otimes T)(R \otimes T) - (R \otimes T)(Q \otimes T) -
    \nonumber \\
    &(Q \otimes T)(Q \otimes S) - (Q \otimes T)(R \otimes S) -
    (Q \otimes T)(R \otimes T) + (Q \otimes T)(Q \otimes T)
\end{align}

Using Equation 2.48,
%
\begin{align}
    M^2 =
    &QQ \otimes SS + QR \otimes SS + QR \otimes ST - QQ \otimes ST +
    \nonumber \\
    &RQ \otimes SS + RR \otimes SS + RR \otimes ST - RQ \otimes ST +
    \nonumber \\
    &RQ \otimes TS + RR \otimes TS + RR \otimes TT - RQ \otimes TT -
    \nonumber \\
    &QQ \otimes TS - QR \otimes TS - QR \otimes TT + QQ \otimes TT
\end{align}

Since $QQ^\dagger = I$ and $Q = Q^\dagger$, then $QQ = I$,
analogously for $R, S, T$.
%
\begin{align}
    M^2 = 
    &I \otimes I + QR \otimes I + QR \otimes ST - I \otimes ST +
    \nonumber \\
    &RQ \otimes I + I \otimes I + I \otimes ST - RQ \otimes ST +
    \nonumber \\
    &RQ \otimes TS + I \otimes TS + I \otimes I - RQ \otimes I -
    \nonumber \\
    &I \otimes TS - QR \otimes TS - QR \otimes I + I \otimes I
\end{align}

Refactoring,
%
\begin{alignat}{2}
    M^2 &= 
    &&4I \otimes I + (QR + RQ - RQ - QR) \otimes I +
    I \otimes (-ST + ST + TS - TS) +
    \nonumber \\
    & &&QR \otimes ST - RQ \otimes ST +
    RQ \otimes TS - QR \otimes TS
    \\
    &= && 4I + QR \otimes ST - RQ \otimes ST + RQ \otimes TS - QR \otimes TS
    \\
    &= && 4I + QR \otimes (ST - TS) - RQ \otimes (ST - TS)
    \\
    &= && 4I + (QR - RQ) \otimes (ST - TS)
\end{alignat}

Therefore, by the definition of commutator (Equation 2.66),
%
\begin{align}
    (Q \otimes S + R \otimes S + R \otimes T - Q \otimes T)^2 =
    4I + [Q, R] \otimes [S, T]
\end{align}

Recall that $Var(x) = E(x^2) - E(x)^2$ 
(as described in the book's Appendix 1).
Also, since $Var(x) \geq 0$,
\begin{align}
    E(x^2) - E(x)^2 &\geq 0 \\
    E(x)^2 &\leq E(x^2) \\
    E(x) &\leq \sqrt{E(x^2)}
\end{align}

Then, calculate $E(x^2)$.
Recall that $E(M) = \bra{\psi} M \ket{\psi} \equiv \mean{M}$
(as described in Equations 2.110 to 2.115).
Also, $E(M + N) = E(M) + E(N)$ and $E(MN) = E(M) E(N)$
Thus, 
\begin{align}
    E(M^2) &= \mean{M^2} \\
    &= \mean{4I + [Q, R] \otimes [S, T]} \\
    &= \mean{4I} + \mean{[Q, R] \otimes [S, T]}
\end{align}

Calculate the mean for $[Q, R]$. The mean of $[S, T]$ is analagous.
\begin{align}
    \mean{[Q, R]} &= \mean{QR - RQ} \\
    &= \mean{QR} - \mean{RQ} \\
    &= \bra{\psi}QR\ket{\psi} - \bra{\psi}RQ\ket{\psi} \\
    &= \bra{\psi}QR\ket{\psi} - \bra{\psi}R^\dagger Q^\dagger \ket{\psi} \\
    &= \bra{\psi}QR\ket{\psi} - \bra{\psi}(QR)^\dagger \ket{\psi} \\
    &= \bra{\psi}QR\ket{\psi} - (\bra{\psi} QR \ket{\psi})^\dagger
\end{align}

Since $\bra{\psi} QR \ket{\psi} \in \mathbb{C}$,
write the resulting complex number as
$\bra{\psi} QR \ket{\psi} = a + bi$,
where $a, b \in \mathbb{R}$.
Therefore,
\begin{align}
    \bra{\psi}QR\ket{\psi} - (\bra{\psi} QR \ket{\psi})^\dagger
    &= a + bi - (a + bi)* \\
    &= a + bi - (a - bi) \\
    &= 2bi \in [-2i, 2i]
\end{align}

Note that $QR$ is unitary
($(QR)^\dagger QR = R^\dagger Q^\dagger Q R = R^\dagger I R = I$).
As a consequence, the inner product between vectors is preserved.
Thus, from $\braket{\psi}{\psi}$,
it is possible to conclude that
$(a + bi)(a + bi)* = a^2 + b^2 = 1$.
As such, $-1 \leq b \leq 1$.
Back to the mean calculus,
and considering $\mean{\bra{\psi} ST \Ket{\psi} = c + di}$,
\begin{align}
    \mean{4I} + \mean{[Q, R] \otimes [S, T]} &=
        \bra{\psi} 4I \ket{\psi} + 2bi \cdot 2di \\
	&= 4 \braket{\psi}{\psi} + 4 bd i^2 \\
	&= 4 - 4 bd
\end{align}

Where $-1 \leq bd \leq 1$.
Therefore, the maximum value of $E(x^2)$ is 8 (whenever $bd = -1$).
Since $E(x) \leq \sqrt{E(x^2)}$,
\begin{align}
	\mean{Q \otimes S + R \otimes S + R \otimes T - Q \otimes T} &\leq
		\sqrt{4 - 4bd} \\
	&\leq 2 \sqrt2
\end{align}

As requested.

\pagebreak
	\section{Nielsen and Chuang - Chapter 04}

\subsection{Section 4.2}
\subsubsection{Bloch Vector}

At this point, it is strongly recommended that the reader understands the contents of
\hyperref[nielsen-and-chuang-qubit-representation-in-a-bloch-sphere]{
    Section \ref{nielsen-and-chuang-qubit-representation-in-a-bloch-sphere}
} and the sections mentioned therein.
Also, in order to properly understand the Bloch Vector,
refer back to Figure 1.3 in the book.

The given vector can be easily understood by interpreting a Bloch Sphere
as two unit circles with an overlapping axis.
For instance, the first circle is determined by the $x$ and $y$ axes
(the ``equatorial line circle''),
while the second circle is determined by $z$ and any
appropriate axis in the ``equatorial line''
(this axis depends on the point's position).

The $z$ position is the easiest one to understand.
It is simply the projection of the point ($\ket \psi$) along the $z$ axis ($\braket z \psi$).
Since it does not change depending on the azimuthal angle $\varphi$.

The positions described by both $x$ and $y$ axes depend on the projection
of the point $\ket \psi$ on the ``equatorial circle'' (name it $\ket{\psi_{xy}}$).
Consider the $x$ axis, if $z = 0$, then the projection of $\ket{\psi_{xy}}$
along $x$ would equal $cos(\varphi)$.
Note that $x \neq cos(\varphi)$ otherwise.
This happens because $\ket{\psi_{xy}}$ depends on $sin(\theta)$.
Thus, $\braket x \psi = cos(\varphi) sin(\theta)$.
Analogously, $\braket y \psi = sin(\varphi) sin(\theta)$.

In conclusion, the Bloch Vector is described by
$(x, y, z) = (cos\varphi sin\theta, sin\varphi sin\theta, cos\theta)$.

%%%%%%%%%%%%%%%%%%%%%%%%%%%%%%%%%%%%%%%%%%%%%%%%%%%%%%%%%%%%%%%%%%%%%%%%%%%%%%%%%%%%%%%%%%%%%%%%%%%%%%%%

\subsubsection{Exercise 4.1}
It is recommended to answer \hyperref[sec:nielsen-and-chuang-exercise-2-11]{Exercise 2.11} beforehand.
Additionally, the reader should attempt to understand the
\hyperref[nielsen-and-chuang-qubit-representation-in-a-bloch-sphere]{Bloch Sphere Equation}.
Throughout this exercise, the reader may constantly refer back to the Bloch Sphere Equation,
\begin{align}
    \ket \psi = \cos \frac \theta 2 \ket0 + e^{i \varphi} sin \frac \theta 2 \ket1 .
\end{align}

\begin{itemize}
    \item The eigenvectors of $X$ are $\ket+$ and $\ket -$;
    \begin{itemize}
        \item $\ket +$;
        \begin{itemize}
            \item $\ket+ = \frac{\ket0 + \ket1}{\sqrt2}$. Hence,
                $cos \frac \theta 2 \ket0 = \frac{1}{\sqrt2} \ket0$. Thus,
                $\theta = \frac \pi 2$;
            \item $\ket+ = \frac{\ket0 + \ket1}{\sqrt2}$. Since
                $\theta = \frac \pi 2$,
                $e^{i \varphi} sin \frac \pi 4 \ket1 = \frac{1}{\sqrt2} \ket1$. Thus,
                $e^{i \varphi} = 1$, and $\varphi = 0$;
            \item Substituting the values of $\theta$ and $\varphi$ in the Bloch Vector formula,
                $\ket+ = (cos\varphi sin\theta, sin\varphi sin\theta, cos\theta) = (1, 0, 0)$;
        \end{itemize}
        
        \item $\ket -$;
        \begin{itemize}
            \item $\ket- = \frac{\ket0 - \ket1}{\sqrt2}$. Hence,
                $cos \frac \theta 2 \ket0 = \frac{1}{\sqrt2} \ket0$. Thus,
                $\theta = \frac \pi 2$;
            \item $\ket- = \frac{\ket0 - \ket1}{\sqrt2}$. Since
                $\theta = \frac \pi 2$,
                $e^{i \varphi} sin \frac \pi 4 \ket1 = - \frac{1}{\sqrt2} \ket1$. Thus,
                $e^{i \varphi} = -1$, and $\varphi = \pi$;
            \item Substituting the values of $\theta$ and $\varphi$ in the Bloch Vector formula,
                $\ket- = (cos\varphi sin\theta, sin\varphi sin\theta, cos\theta) = (-1, 0, 0)$;
        \end{itemize}
    \end{itemize}
    
    \item The eigenvectors of $Y$ are $\frac{\ket0 + i \ket1}{\sqrt2}$ and $\frac{\ket0 - i \ket1}{\sqrt2}$;
    \begin{itemize}
        \item $\frac{\ket0 + i \ket1}{\sqrt2}$;
        \begin{itemize}
            \item $cos \frac \theta 2 \ket0 = \frac{1}{\sqrt2} \ket0$. Thus,
                $\theta = \frac \pi 2$;
            \item Since $\theta = \frac \pi 2$,
                $e^{i \varphi} sin \frac \pi 4 \ket1 = \frac{i}{\sqrt2} \ket1$. Thus,
                $e^{i \varphi} = i$, and $\varphi = \frac \pi 2$;
            \item Substituting the values of $\theta$ and $\varphi$ in the Bloch Vector formula,
                $\ket+ = (cos\varphi sin\theta, sin\varphi sin\theta, cos\theta) = (0, 1, 0)$;
        \end{itemize}
        
        \item $\frac{\ket0 - i \ket1}{\sqrt2}$;
        \begin{itemize}
            \item $cos \frac \theta 2 \ket0 = \frac{1}{\sqrt2} \ket0$. Thus,
                $\theta = \frac \pi 2$;
            \item Since $\theta = \frac \pi 2$,
                $e^{i \varphi} sin \frac \pi 4 \ket1 = - \frac{i}{\sqrt2} \ket1$. Thus,
                $e^{i \varphi} = - i$, and $\varphi = \frac{3\pi}{2}$;
            \item Substituting the values of $\theta$ and $\varphi$ in the Bloch Vector formula,
                $\ket+ = (cos\varphi sin\theta, sin\varphi sin\theta, cos\theta) = (0, -1, 0)$;
        \end{itemize}
    \end{itemize}
    
    \item The eigenvectors of $Z$ are $\ket0$ and $\ket1$;
    \begin{itemize}
        \item $\ket0$;
        \begin{itemize}
            \item $cos \frac \theta 2 \ket0 = 1 \ket0$. Thus, $\theta = 0$;
            \item Since $\theta = 0$, $sin(0) = 0$. And
                $\varphi$ can assume any value in the $[0, 2\pi)$ range;
            \item Substituting the values of $\theta$ and $\varphi$ in the Bloch Vector formula,
                $\ket0 = (cos\varphi sin\theta, sin\varphi sin\theta, cos\theta) = (0, 0, 1)$
        \end{itemize}
        
        \item $\ket1$;
        \begin{itemize}
            \item $cos \frac \theta 2 \ket0 = 0 \ket0$. Thus, $\theta = \pi$;
            \item Thus, $\ket1 = e^{i \varphi} sin \frac \pi 2 \ket 1$.
                It is known that \hyperref[sec:noson-equation-4-5]{
                    multiplying a state by any complex number does not change it}.
                Hence, $e^{i \varphi} sin \frac \pi 2 \ket1 =
                    e^{-i \varphi} e^{i \varphi} sin \frac \pi 2 \ket1 =
                    sin \frac \pi 2 \ket1$.
                In conclusion, the value of $\varphi$ is negligible,
                and can assume any value in the $[0, 2\pi)$ range;
            \item Substituting the values of $\theta$ and $\varphi$ in the Bloch Vector formula,
                $\ket1 = (cos\varphi sin\theta, sin\varphi sin\theta, cos\theta) = (0, 0, -1)$.
        \end{itemize}
    \end{itemize}
\end{itemize}

%%%%%%%%%%%%%%%%%%%%%%%%%%%%%%%%%%%%%%%%%%%%%%%%%%%%%%%%%%%%%%%%%%%%%%%%%%%%%%%%%%%%%%%%%%%%%%%%%%%%%%%%

\subsubsection{Exercise 4.2}
In Nielsen and Chuang's Section 2.1.8, operator functions are discussed.
Thus, $A$ has spectral decomposition, and
\begin{align}
    A^2 &= \sum_{ab} a \ketbra{a}{a} b \ketbra{b}{b} \\
    &= \sum_{ab} ab \ket{a} \braket{a}{b} \bra{b} \\
    &= \sum_{ab} ab \braket{a}{b} \ket{a} \bra{b}.
\end{align}
Since $\set{\ket a \mid \forall a}$ form an orthonormal basis defined by the eigenspace
- and $\set{\ket b \mid \forall b}$ describes the same basis -
$\braket{a}{b} = \delta_{ab}$
\footnote{$\delta_{ij}$ is defined in the paragraph that follows
    Nielsen and Chuang's Equation (2.16)
},
\begin{align}
    A^2 &= \sum_{ab} ab \delta_{ab} \ket{a} \bra{b} \\
    &= \sum_a a^2 \ketbra a a \\
    &= I.
\end{align}
Due to the completeness relation $\sum_a \ketbra a a = I$,
it is possible to conclude that $a = \pm 1$.

Compute the value of $exp(iAx)$ using the definition of operator functions
Nielsen and Chuang's Section 2.1.8, and Euler's Formula.
\begin{align}
    exp(iAx) &= \sum_a exp(iax) \ketbra a a \\
    &= \sum_a cos(ax) \ketbra a a + i\ sin(ax) \ketbra a a.
\end{align}
Recall that $a = \pm 1$.
From trigonometry, it is known $cos(x) = cos(-x)$.
Also, $sin(-x) = - sin(x)$.
Thus,
\begin{align}
    exp(iAx) &= \sum_a cos(x) \ketbra a a + i\ sin(x) a \ketbra a a \\
    &= cos(x) I + i\ sin(x) A.
\end{align}

Using this to verify Equations (4.4) to (4.6) is straightforward,
since $X^2 = Y^2 = Z^2 = I$.
For $R_x(\theta)$,
\begin{align}
    e^{i \theta X / 2} &= cos( - \frac \theta 2) I + i\ sin( - \frac \theta 2) X \\
    &= cos \frac \theta 2 I - i\ sin \frac \theta 2 X \\
    &= \left[ \begin{matrix} cos \frac \theta 2 & 0 \\ 0 & cos \frac \theta 2 \end{matrix} \right] - i
        \left[ \begin{matrix} 0 & sin \frac \theta 2 \\ sin \frac \theta 2 & 0 \end{matrix} \right] \\
    &= \left[ \begin{matrix} cos \frac \theta 2 & - i\ sin \frac \theta 2 \\
        - i\ sin \frac \theta 2 & cos \frac \theta 2 \end{matrix} \right].
\end{align}
For $R_y(\theta)$,
\begin{align}
    e^{i \theta Y / 2} &= cos( - \frac \theta 2) I + i\ sin( - \frac \theta 2) Y \\
    &= cos \frac \theta 2 I - i\ sin \frac \theta 2 Y \\
    &= \left[ \begin{matrix} cos \frac \theta 2 & 0 \\ 0 & cos \frac \theta 2 \end{matrix} \right] - i
        \left[ \begin{matrix} 0 & -i\ sin \frac \theta 2 \\ i\ sin \frac \theta 2 & 0 \end{matrix} \right] \\
    &= \left[ \begin{matrix} cos \frac \theta 2 & - sin \frac \theta 2 \\
        sin \frac \theta 2 & cos \frac \theta 2 \end{matrix} \right].
\end{align}
For $R_z(\theta)$,
\begin{align}
    e^{i \theta Z / 2} &= cos( - \frac \theta 2) I + i\ sin( - \frac \theta 2) Z \\
    &= cos \frac \theta 2 I - i\ sin \frac \theta 2 Z \\
    &= \left[ \begin{matrix} cos \frac \theta 2 & 0 \\ 0 & cos \frac \theta 2 \end{matrix} \right] - i
        \left[ \begin{matrix} sin \frac \theta 2 & 0 \\ 0 & - sin \frac \theta 2\end{matrix} \right] \\
    &= \left[ \begin{matrix} cos \frac \theta 2 - i\ sin \frac \theta 2 & 0 \\
        0 & cos \frac \theta 2 + i sin \frac \theta 2 \end{matrix} \right] \\
    &= \left[ \begin{matrix} cos(- \frac \theta 2) + i\ sin(- \frac \theta 2) & 0 \\
        0 & cos \frac \theta 2 + i sin \frac \theta 2 \end{matrix} \right] \\
    &= \left[ \begin{matrix} e^{-i \theta/2} & 0 \\ 0 & e^{i \theta/2} \end{matrix} \right]
\end{align}

%%%%%%%%%%%%%%%%%%%%%%%%%%%%%%%%%%%%%%%%%%%%%%%%%%%%%%%%%%%%%%%%%%%%%%%%%%%%%%%%%%%%%%%%%%%%%%%%%%%%%%%%

\subsubsection{Exercise 4.3}
Compute $\pi/4$,
\begin{align}
    R_z(\pi/4) &= cos \frac{\pi}{8} I - sin \frac{\pi}{8} Z \\
    &= \left[ \begin{matrix} e^{-i \pi/8} & 0 \\ 0 & e^{i \pi/8} \end{matrix} \right].
\end{align}
Note that this practically matches $T$ - refer back to Equation (4.3).
Thus,
\begin{align}
    T = e^{i \pi/8} R_z(\pi/4).
\end{align}


\pagebreak
	\section{Nielsen and Chuang - Chapter 06}

\subsection{Section 6.1}

\subsubsection{Exercise 6.1}
This Exercise refers to Equation (6.5),
not to Equation (6.3), which describes the oracle's action.
Recall that the $\delta_{ij}$ notation means
\begin{align}
    \delta_{ij} = \begin{cases}
        1 & \text{if\ } i = j \\
        0 & \text{if\ } i \neq j
    \end{cases}.
\end{align}
Thus, Equation (6.5) essentially means that if
$x = 0$, then $\ket0 \rightarrow \ket0$, else,
$x \neq 0$ and $\ket x \rightarrow - \ket x$.
This is exactly the behaviour of $2 \ketbra{0}{0} - I$,
which is described in the following paragraphs.

First, the operator is easily verifiable to be unitary,
\begin{align}
    (2 \ketbra{0}{0} - I)^\dagger (2 \ketbra{0}{0} - I) &=
        (2 \ketbra{0}{0} - I)^2 \\
    &= 4 \ket0 \braket{0}{0} \bra0 - 2 \cdot 2 I \ketbra{0}{0} + I^2 \\
    &= 4 \ketbra00 - 4 \ketbra00 + I \\
    &= I.
\end{align}

Show that $(2 \ketbra00 - I) \ket0 = \ket0$,
\begin{align}
    (2 \ketbra{0}{0} - I) \ket0 &= 2 \ket{0} \braket{0}{0} - I \ket{0} \\
    &= 2 \ket0 - \ket0 \\
    &= \ket 0.
\end{align}
For any other state
    \footnote{Recall that Grover's Algorithm operates on the database's indices,
    and that they form an orthonormal basis, i.e. $\braket{i}{j} = \delta_{ij}$.}
$\ket x$, the phase is shifted,
\begin{align}
    (2 \ketbra{0}{0} - I) \ket x &= 2 \ket{0} \braket{0}{x} - I \ket{x} \\
    &= 0 \ket0 - \ket{x} \\
    &= - \ket{x}.
\end{align}

These actions can be easily inferred by analysing $(2 \ketbra{0}{0} - I)$'s matricial form,
\begin{align}
    \left[ \begin{matrix}
        2 & 0 & 0 & \cdots & 0 \\
        0 & 0 & 0 & \cdots & 0 \\
        0 & 0 & 0 & \cdots & 0 \\
        \vdots & \vdots & \vdots & \ddots & \vdots \\
        0 & 0 & 0 & \cdots & 0
    \end{matrix} \right]
    -
    \left[ \begin{matrix}
        1 & 0 & 0 & \cdots & 0 \\
        0 & 1 & 0 & \cdots & 0 \\
        0 & 0 & 1 & \cdots & 0 \\
        \vdots & \vdots & \vdots & \ddots & \vdots \\
        0 & 0 & 0 & \cdots & 1
    \end{matrix} \right]
    =
    \left[ \begin{matrix}
        1 & 0 & 0 & \cdots & 0 \\
        0 & -1 & 0 & \cdots & 0 \\
        0 & 0 & -1 & \cdots & 0 \\
        \vdots & \vdots & \vdots & \ddots & \vdots \\
        0 & 0 & 0 & \cdots & -1
    \end{matrix} \right].
\end{align}

\subsubsection{Exercise 6.2}
Applying the operation to the general state,
\begin{align}
    (2 \ketbra{\psi}{\psi} - I) \sum_k \alpha_k \ket{k} = 
    \sum_k 2 \alpha_k \ket{\psi} \braket{\psi}{k} - \alpha_k \ket{k}.
\end{align}
Using Equation (6.4), and the definition of $\delta_{ij}$,
\begin{align}
    \sum_k 2 \alpha_k \ket{\psi} \braket{\psi}{k} - \alpha_k \ket{k} &=
    \sum_k 2 \alpha_k \frac{1}{N} \sum_{ij} \ket{i} \braket{j}{k} - \alpha_k \ket{k} \\
    & = \sum_k 2 \frac{\alpha_k}{N} \sum_{ij} \ket{i} \delta_{jk} - \alpha_k \ket{k} \\
    & = \sum_k 2 \frac{\alpha_k}{N} \sum_i \ket{i} - \alpha_k \ket{k}.
\end{align}
Note that $\sum_k \alpha_k/N$ is a constant
(written as $\langle\alpha\rangle$). Thus,
\begin{align}
    \sum_k 2 \frac{\alpha_k}{N} \sum_i \ket{i} - \alpha_k \ket{k} =
    2 \langle\alpha\rangle \sum_i \ket{i} - \sum_k \alpha_k \ket{k}.
\end{align}
Since $\ket{k}$ and $\ket{i}$ correspond to the same orthonormal basis,
it is possible to rename and rearrange
\begin{align}
    2 \langle\alpha\rangle \sum_i \ket{i} - \sum_k \alpha_k \ket{k} &=
    \sum_k 2 \langle\alpha\rangle \ket{k} - \sum_k \alpha_k \ket{k} \\
    &= \sum_k \left[ 2 \langle\alpha\rangle - \alpha_k \right] \ket{k}.
\end{align}
Thus obtaining the desired answer.

\subsubsection{Notes on Grover's Algorithm}
\label{sec:nielsen-and-chuang-notes-on-grovers-algorithm}
\todo{Notes on finding $\ket0$ index}
\begin{comment}
If the searched item is \emph{not} in the database,
then applying the Grover Iteration should have no effects.
In fact, the conditional shift is partially responsible for this behaviour.
For instance,this is exactly what happens and exactly what is described by
$2 \ketbra{0}{0} - I$, applying it to $\ket 0$,

The first step of Grover's Algorithm is to apply
$H^{\otimes n}$ to the initial state $\ket0^{\otimes n}$.
Thus, the oracle will compute all the possible results due to Quantum Superposition
(Quantum Parallelism).
Suppose that the action of the oracle is none.
That is, the searched value is \emph{not} in the database.
Then, $O H^{\otimes n} \ket0^{\otimes n} = H^{\otimes n} \ket0^{\otimes n}$.
Since the first action after the oracle is to apply $H^{\otimes n}$ again,
$H^{\otimes n} H^{\otimes n} \ket0^{\otimes n} = \ket0^{\otimes n}$.

On the other hand, if the oracle changes the $H^{\otimes n} \ket0^{\otimes n}$ state,
then the searched item is in the database.

refer back to matricial form of conditional phase shift? to when
$(2 \ketbra{0}{0})\ket0$ and $(2 \ketbra{0}{0})(- \ket0)$?

Note that the conditional phase shift does not flip $- \ket 0$:
$(2 \ketbra00 - I)(- \ket0) 0 -2 \ket0 + \ket0 = - \ket0$.

step-by-step example?
\end{comment}

\pagebreak
	\section{Nielsen and Chuang - Chapter 04}

\subsection{Section 4.2}
\subsubsection{Bloch Vector}

At this point, it is strongly recommended that the reader understands the contents of
\hyperref[nielsen-and-chuang-qubit-representation-in-a-bloch-sphere]{
    Section \ref{nielsen-and-chuang-qubit-representation-in-a-bloch-sphere}
} and the sections mentioned therein.
Also, in order to properly understand the Bloch Vector,
refer back to Figure 1.3 in the book.

The given vector can be easily understood by interpreting a Bloch Sphere
as two unit circles with an overlapping axis.
For instance, the first circle is determined by the $x$ and $y$ axes
(the ``equatorial line circle''),
while the second circle is determined by $z$ and any
appropriate axis in the ``equatorial line''
(this axis depends on the point's position).

The $z$ position is the easiest one to understand.
It is simply the projection of the point ($\ket \psi$) along the $z$ axis ($\braket z \psi$).
Since it does not change depending on the azimuthal angle $\varphi$.

The positions described by both $x$ and $y$ axes depend on the projection
of the point $\ket \psi$ on the ``equatorial circle'' (name it $\ket{\psi_{xy}}$).
Consider the $x$ axis, if $z = 0$, then the projection of $\ket{\psi_{xy}}$
along $x$ would equal $cos(\varphi)$.
Note that $x \neq cos(\varphi)$ otherwise.
This happens because $\ket{\psi_{xy}}$ depends on $sin(\theta)$.
Thus, $\braket x \psi = cos(\varphi) sin(\theta)$.
Analogously, $\braket y \psi = sin(\varphi) sin(\theta)$.

In conclusion, the Bloch Vector is described by
$(x, y, z) = (cos\varphi sin\theta, sin\varphi sin\theta, cos\theta)$.

%%%%%%%%%%%%%%%%%%%%%%%%%%%%%%%%%%%%%%%%%%%%%%%%%%%%%%%%%%%%%%%%%%%%%%%%%%%%%%%%%%%%%%%%%%%%%%%%%%%%%%%%

\subsubsection{Exercise 4.1}
It is recommended to answer \hyperref[sec:nielsen-and-chuang-exercise-2-11]{Exercise 2.11} beforehand.
Additionally, the reader should attempt to understand the
\hyperref[nielsen-and-chuang-qubit-representation-in-a-bloch-sphere]{Bloch Sphere Equation}.
Throughout this exercise, the reader may constantly refer back to the Bloch Sphere Equation,
\begin{align}
    \ket \psi = \cos \frac \theta 2 \ket0 + e^{i \varphi} sin \frac \theta 2 \ket1 .
\end{align}

\begin{itemize}
    \item The eigenvectors of $X$ are $\ket+$ and $\ket -$;
    \begin{itemize}
        \item $\ket +$;
        \begin{itemize}
            \item $\ket+ = \frac{\ket0 + \ket1}{\sqrt2}$. Hence,
                $cos \frac \theta 2 \ket0 = \frac{1}{\sqrt2} \ket0$. Thus,
                $\theta = \frac \pi 2$;
            \item $\ket+ = \frac{\ket0 + \ket1}{\sqrt2}$. Since
                $\theta = \frac \pi 2$,
                $e^{i \varphi} sin \frac \pi 4 \ket1 = \frac{1}{\sqrt2} \ket1$. Thus,
                $e^{i \varphi} = 1$, and $\varphi = 0$;
            \item Substituting the values of $\theta$ and $\varphi$ in the Bloch Vector formula,
                $\ket+ = (cos\varphi sin\theta, sin\varphi sin\theta, cos\theta) = (1, 0, 0)$;
        \end{itemize}
        
        \item $\ket -$;
        \begin{itemize}
            \item $\ket- = \frac{\ket0 - \ket1}{\sqrt2}$. Hence,
                $cos \frac \theta 2 \ket0 = \frac{1}{\sqrt2} \ket0$. Thus,
                $\theta = \frac \pi 2$;
            \item $\ket- = \frac{\ket0 - \ket1}{\sqrt2}$. Since
                $\theta = \frac \pi 2$,
                $e^{i \varphi} sin \frac \pi 4 \ket1 = - \frac{1}{\sqrt2} \ket1$. Thus,
                $e^{i \varphi} = -1$, and $\varphi = \pi$;
            \item Substituting the values of $\theta$ and $\varphi$ in the Bloch Vector formula,
                $\ket- = (cos\varphi sin\theta, sin\varphi sin\theta, cos\theta) = (-1, 0, 0)$;
        \end{itemize}
    \end{itemize}
    
    \item The eigenvectors of $Y$ are $\frac{\ket0 + i \ket1}{\sqrt2}$ and $\frac{\ket0 - i \ket1}{\sqrt2}$;
    \begin{itemize}
        \item $\frac{\ket0 + i \ket1}{\sqrt2}$;
        \begin{itemize}
            \item $cos \frac \theta 2 \ket0 = \frac{1}{\sqrt2} \ket0$. Thus,
                $\theta = \frac \pi 2$;
            \item Since $\theta = \frac \pi 2$,
                $e^{i \varphi} sin \frac \pi 4 \ket1 = \frac{i}{\sqrt2} \ket1$. Thus,
                $e^{i \varphi} = i$, and $\varphi = \frac \pi 2$;
            \item Substituting the values of $\theta$ and $\varphi$ in the Bloch Vector formula,
                $\ket+ = (cos\varphi sin\theta, sin\varphi sin\theta, cos\theta) = (0, 1, 0)$;
        \end{itemize}
        
        \item $\frac{\ket0 - i \ket1}{\sqrt2}$;
        \begin{itemize}
            \item $cos \frac \theta 2 \ket0 = \frac{1}{\sqrt2} \ket0$. Thus,
                $\theta = \frac \pi 2$;
            \item Since $\theta = \frac \pi 2$,
                $e^{i \varphi} sin \frac \pi 4 \ket1 = - \frac{i}{\sqrt2} \ket1$. Thus,
                $e^{i \varphi} = - i$, and $\varphi = \frac{3\pi}{2}$;
            \item Substituting the values of $\theta$ and $\varphi$ in the Bloch Vector formula,
                $\ket+ = (cos\varphi sin\theta, sin\varphi sin\theta, cos\theta) = (0, -1, 0)$;
        \end{itemize}
    \end{itemize}
    
    \item The eigenvectors of $Z$ are $\ket0$ and $\ket1$;
    \begin{itemize}
        \item $\ket0$;
        \begin{itemize}
            \item $cos \frac \theta 2 \ket0 = 1 \ket0$. Thus, $\theta = 0$;
            \item Since $\theta = 0$, $sin(0) = 0$. And
                $\varphi$ can assume any value in the $[0, 2\pi)$ range;
            \item Substituting the values of $\theta$ and $\varphi$ in the Bloch Vector formula,
                $\ket0 = (cos\varphi sin\theta, sin\varphi sin\theta, cos\theta) = (0, 0, 1)$
        \end{itemize}
        
        \item $\ket1$;
        \begin{itemize}
            \item $cos \frac \theta 2 \ket0 = 0 \ket0$. Thus, $\theta = \pi$;
            \item Thus, $\ket1 = e^{i \varphi} sin \frac \pi 2 \ket 1$.
                It is known that \hyperref[sec:noson-equation-4-5]{
                    multiplying a state by any complex number does not change it}.
                Hence, $e^{i \varphi} sin \frac \pi 2 \ket1 =
                    e^{-i \varphi} e^{i \varphi} sin \frac \pi 2 \ket1 =
                    sin \frac \pi 2 \ket1$.
                In conclusion, the value of $\varphi$ is negligible,
                and can assume any value in the $[0, 2\pi)$ range;
            \item Substituting the values of $\theta$ and $\varphi$ in the Bloch Vector formula,
                $\ket1 = (cos\varphi sin\theta, sin\varphi sin\theta, cos\theta) = (0, 0, -1)$.
        \end{itemize}
    \end{itemize}
\end{itemize}

%%%%%%%%%%%%%%%%%%%%%%%%%%%%%%%%%%%%%%%%%%%%%%%%%%%%%%%%%%%%%%%%%%%%%%%%%%%%%%%%%%%%%%%%%%%%%%%%%%%%%%%%

\subsubsection{Exercise 4.2}
In Nielsen and Chuang's Section 2.1.8, operator functions are discussed.
Thus, $A$ has spectral decomposition, and
\begin{align}
    A^2 &= \sum_{ab} a \ketbra{a}{a} b \ketbra{b}{b} \\
    &= \sum_{ab} ab \ket{a} \braket{a}{b} \bra{b} \\
    &= \sum_{ab} ab \braket{a}{b} \ket{a} \bra{b}.
\end{align}
Since $\set{\ket a \mid \forall a}$ form an orthonormal basis defined by the eigenspace
- and $\set{\ket b \mid \forall b}$ describes the same basis -
$\braket{a}{b} = \delta_{ab}$
\footnote{$\delta_{ij}$ is defined in the paragraph that follows
    Nielsen and Chuang's Equation (2.16)
},
\begin{align}
    A^2 &= \sum_{ab} ab \delta_{ab} \ket{a} \bra{b} \\
    &= \sum_a a^2 \ketbra a a \\
    &= I.
\end{align}
Due to the completeness relation $\sum_a \ketbra a a = I$,
it is possible to conclude that $a = \pm 1$.

Compute the value of $exp(iAx)$ using the definition of operator functions
Nielsen and Chuang's Section 2.1.8, and Euler's Formula.
\begin{align}
    exp(iAx) &= \sum_a exp(iax) \ketbra a a \\
    &= \sum_a cos(ax) \ketbra a a + i\ sin(ax) \ketbra a a.
\end{align}
Recall that $a = \pm 1$.
From trigonometry, it is known $cos(x) = cos(-x)$.
Also, $sin(-x) = - sin(x)$.
Thus,
\begin{align}
    exp(iAx) &= \sum_a cos(x) \ketbra a a + i\ sin(x) a \ketbra a a \\
    &= cos(x) I + i\ sin(x) A.
\end{align}

Using this to verify Equations (4.4) to (4.6) is straightforward,
since $X^2 = Y^2 = Z^2 = I$.
For $R_x(\theta)$,
\begin{align}
    e^{i \theta X / 2} &= cos( - \frac \theta 2) I + i\ sin( - \frac \theta 2) X \\
    &= cos \frac \theta 2 I - i\ sin \frac \theta 2 X \\
    &= \left[ \begin{matrix} cos \frac \theta 2 & 0 \\ 0 & cos \frac \theta 2 \end{matrix} \right] - i
        \left[ \begin{matrix} 0 & sin \frac \theta 2 \\ sin \frac \theta 2 & 0 \end{matrix} \right] \\
    &= \left[ \begin{matrix} cos \frac \theta 2 & - i\ sin \frac \theta 2 \\
        - i\ sin \frac \theta 2 & cos \frac \theta 2 \end{matrix} \right].
\end{align}
For $R_y(\theta)$,
\begin{align}
    e^{i \theta Y / 2} &= cos( - \frac \theta 2) I + i\ sin( - \frac \theta 2) Y \\
    &= cos \frac \theta 2 I - i\ sin \frac \theta 2 Y \\
    &= \left[ \begin{matrix} cos \frac \theta 2 & 0 \\ 0 & cos \frac \theta 2 \end{matrix} \right] - i
        \left[ \begin{matrix} 0 & -i\ sin \frac \theta 2 \\ i\ sin \frac \theta 2 & 0 \end{matrix} \right] \\
    &= \left[ \begin{matrix} cos \frac \theta 2 & - sin \frac \theta 2 \\
        sin \frac \theta 2 & cos \frac \theta 2 \end{matrix} \right].
\end{align}
For $R_z(\theta)$,
\begin{align}
    e^{i \theta Z / 2} &= cos( - \frac \theta 2) I + i\ sin( - \frac \theta 2) Z \\
    &= cos \frac \theta 2 I - i\ sin \frac \theta 2 Z \\
    &= \left[ \begin{matrix} cos \frac \theta 2 & 0 \\ 0 & cos \frac \theta 2 \end{matrix} \right] - i
        \left[ \begin{matrix} sin \frac \theta 2 & 0 \\ 0 & - sin \frac \theta 2\end{matrix} \right] \\
    &= \left[ \begin{matrix} cos \frac \theta 2 - i\ sin \frac \theta 2 & 0 \\
        0 & cos \frac \theta 2 + i sin \frac \theta 2 \end{matrix} \right] \\
    &= \left[ \begin{matrix} cos(- \frac \theta 2) + i\ sin(- \frac \theta 2) & 0 \\
        0 & cos \frac \theta 2 + i sin \frac \theta 2 \end{matrix} \right] \\
    &= \left[ \begin{matrix} e^{-i \theta/2} & 0 \\ 0 & e^{i \theta/2} \end{matrix} \right]
\end{align}

%%%%%%%%%%%%%%%%%%%%%%%%%%%%%%%%%%%%%%%%%%%%%%%%%%%%%%%%%%%%%%%%%%%%%%%%%%%%%%%%%%%%%%%%%%%%%%%%%%%%%%%%

\subsubsection{Exercise 4.3}
Compute $\pi/4$,
\begin{align}
    R_z(\pi/4) &= cos \frac{\pi}{8} I - sin \frac{\pi}{8} Z \\
    &= \left[ \begin{matrix} e^{-i \pi/8} & 0 \\ 0 & e^{i \pi/8} \end{matrix} \right].
\end{align}
Note that this practically matches $T$ - refer back to Equation (4.3).
Thus,
\begin{align}
    T = e^{i \pi/8} R_z(\pi/4).
\end{align}


\pagebreak
	\section{Noson - Chapter 05} 
\label{sec:noson-chapter-5}

\subsection{Section 5.4}
\label{sec:noson-section-5-4}

\subsubsection{Bloch Sphere - Equations (5.80) to (5.88)}
\label{sec:noson-bloch-sphere-eq-5-80-to-5-88}
Equations (5.80) to (5.88) explain how the Bloch Sphere representation of a Qubit is derived
from the standard representation
($\ket\psi = \alpha \ket0 + \beta \ket1$, where $\alpha$ and $\beta \in \mathbb{C}$,
and $|\alpha|^2 + |\beta|^2 = 1$).
Since the logical steps of the derivation are flawlessly explained,
only complementary comments are necessary
(which are easily deductible if a solid background was built to this point).

\begin{itemize}
    \item \emph{Equations (5.81) and (5.82)}:
        Recall that any complex number can be written in the polar form.
        That is, if $z \in \mathbb{C}$ ($z = a + bi$, where $a, b \in \mathbb{R}$),
        then $z = r(cos\theta + i\ sin\theta$),
        where $\theta \in [0, 2\pi)$,
        and $r \in \mathbb{R}$ and it is the norm (length) of $z$
        ($|z| = r = \sqrt{a^2 + b^2}$).
        To obtain Equations (5.81) and (5.82), apply Euler's formula
        ($e^{ix} = cos(x) + i\ sin(x)$).
        The polar form is useful because it simplifies complex numbers multiplications and exponentiation.
        Additionally, it gives a geometrical interpretation for complex numbers:
        they can be represented as a point in a plane;
        
    \item \emph{Equation 5.84}:
        Multiplying any state $\ket \psi$ by a scalar
        $z \in \mathbb{C}, z \neq 0$ does not change the state $\ket \psi$.
        For a detailed explanation, refer back to Section \ref{sec:noson-equation-4-5};
        
    \item \emph{Equations (5.86) and (5.87)}:
        It is possible to rename the $r_0$ and $r_1$ due to the
        Pythagorean trigonometric identity ($\sin^2\theta + cos^2\theta = 1$);
    
    \item \emph{Equations (5.87) and (5.88)}:
        Substitute the values of Equation(5.87) in the
        final result of Equation(5.84);
        
    \item \emph{Equation (5.88) - range of $\theta$ and $\phi$}:
        The range for unique spherical coordinates is $0 \leq \theta \leq \pi$ for
        the polar angle (elevation) and $0 \leq \phi < 2\pi$ for the azimuthal angle.
        Noson states that the ranges are $0 \leq \theta < \frac \pi 2$ and $0 \leq \phi < 2\pi$.
        In this case, however, there would exist an unrepresentable point in the Bloch Sphere
        ($\theta = \frac \pi 2$).
        Given the description of \hyperref[sec:noson-exercise-5-4-4]{Exercise 5.4.4},
        it is possible to conclude that this was a typo.
        Therefore, the correct range for the polar angle $\theta$ is $0 \leq \theta \leq \frac \pi 2$.
        The proof that $\theta \in [0, \frac \pi 2]$ instead of $\theta \in [0, \pi]$ can be found in
        \hyperref[sec:noson-exercise-5-4-4]{Exercise 5.4.4}.
\end{itemize}

\subsubsection{Exercise 5.4.4}
\label{sec:noson-exercise-5-4-4}
A Qubit is represented in the Bloch Sphere by the formula
\begin{align}
    \ket \psi = cos \theta \ket 0 + e^{i \phi} sin \theta \ket 1 .
\end{align}
In spherical coordinates, the polar angle $\theta \in [0, \pi]$.
From trigonometry, it is known that
$cos(\theta) = - cos(\theta + \pi) = - cos(\pi - \theta)$, and
$sin(\theta) = sin(\pi - \theta)$.
Substituting these identities in $\ket \psi$,
\begin{align}
    \ket \psi = -cos (\pi - \theta) \ket 0 + e^{i\phi} sin(\pi - \theta) .
\end{align}

From \hyperref[sec:noson-equation-4-5]{Equation (4.5)}, it is known that
multiplying a Qubit $\ket \psi$ by $z \in \mathbb{C}$ does not change
$\ket \psi$'s state.
Therefore, using Euler's formula and multiplying $\ket \psi$ by $e^{i \pi}$,
\begin{align}
    \ket \psi &= e^{i \pi} (-cos (\pi - \theta) \ket 0 + e^{i\phi} sin(\pi - \theta)) \\
        &= (cos (\pi) + i\ sin(\pi)) (-cos (\pi - \theta) \ket 0) +
            e^{i (\phi + \pi)} sin(\pi - \theta) \\
        &= (-1) -cos (\pi - \theta) \ket 0) +
            e^{i (\phi + \pi)} sin(\pi - \theta) \\
        &= cos (\pi - \theta) \ket 0) + e^{i (\phi + \pi)} sin(\pi - \theta) .
\end{align}
Hence, if $\theta > \pi / 2$, it is possible to rewrite it in terms of
$\theta \in [0, \frac \pi 2]$ by simply adding $\pi$ degrees to the
azimuthal angle $\phi$,
and subtracting the polar angle $\theta$ from $\pi$.

By a similar line of thought, it is possible to prove that even if the ranges were inverted
($\theta \in [0, 2\pi)$, and $\phi \in [0, \pi]$),
then they could be mapped back to the $\theta \in [0, \pi]$, and $\phi \in [0, 2\pi)$ ranges.

\pagebreak

\end{document}