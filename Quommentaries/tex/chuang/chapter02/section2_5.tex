\subsection{Section 2.5}
\subsubsection{Exercise 2.82}

\begin{enumerate}
\item From the definition of purification, it is necessary to prove that $\rho^A = tr_R(\ket{AR}\bra{AR})$.

Suppose $\sum_i \sqrt(p_i) \ket{\psi_i} \ket{i}$ is a purification. From equation (2.138) of Nielsen and Chuang's book:
\begin{align}
    \rho^{AB} = \sum_{ij} \left( \sqrt{p_i} \ket{\psi_i} \ket{i} \right) \left( \sqrt{p_j} \ket{\psi_j} \ket{j} \right) ^ \dagger
\end{align}

Since $p_j \in \mathbb{R}$, $p_j^\dagger = p_j$. And since it is a scalar:
\begin{align}
    \rho^{AB} = \sum_{ij} \sqrt{p_i p_j} \left( \ket{\psi_i} \ket{i} \right) \left( \ket{\psi_j} \ket{j} \right) ^ \dagger
\end{align}

Recall that for any states $\ket{\varphi}$ and $\ket{\gamma}$ writing $\ket{\varphi} \ket{\gamma}$ is the same as $\ket{\varphi} \otimes \ket{\gamma}$. Then, applying equation (2.48):
\begin{align}
    \rho^{AB} &= \sum_{ij} \sqrt{p_i p_j} \left( \ket{\psi_i} \otimes \ket{i} \right)
        \left( \bra{\psi_j} \otimes \bra{j} \right) \\
    \rho^{AB} &= \sum_{ij} \sqrt{p_i p_j} (\ket{\psi_i}\bra{\psi_j}) \otimes (\ket{i}\bra{j})
\end{align}

If $\sum_i \sqrt{p_i} \ket{\psi_i}\ket{i}$ is a purification, then
$\rho^A = tr_B(\ (\ket{\psi_i}\ket{i}) (\bra{\psi_i}\bra{i})\ )$.
The question gives $\rho = \sum_i p_i \ket{\psi_i}\bra{\psi_i}$, in other words $\rho = \rho^A$.

Now, it is necessary to calculate $tr_B(\ (\ket{\psi_i}\ket{i}) (\bra{\psi_i}\bra{i})\ )$. Then, using the definitions given in equations (2.177) and (2.178) of Nielsen and Chuang's book:
\begin{align}
    tr_B(\rho^{AB}) &= tr_B \left(\sum_{ij} \sqrt{p_i p_j} (\ket{\psi_i}\bra{\psi_j}) \otimes (\ket{i}\bra{j}) \right) \\
    &= \sum_{ij} \sqrt{p_i p_j} (\ket{\psi_i}\bra{\psi_j}) tr(\ket{i}\bra{j})
\end{align}

Since $tr(\ket{a}\bra{b}) = \braket{b|a}$:
\begin{align}
    tr_B(\rho^{AB}) &= \sum_{ij} \sqrt{p_i p_j}\ (\ket{\psi_i}\bra{\psi_j}) \braket{j|i} \\
    &= \sum_{ij} \sqrt{p_i p_j}\ \ket{\psi_i}\bra{\psi_j}\ \delta_{ij} \\
    &= \sum_i \sqrt{p_i p_i}\ \ket{\psi_i}\bra{\psi_i} \\
    &= \sum_i p_i\ket{\psi_i}\bra{\psi_i}
\end{align}

Since $\rho^A = \sum_i p_i \ket{\psi_i}\bra{\psi_i} = tr_B(\rho^{AB})$,
it is possible to conclude that $\sum_i \sqrt{p_i} \ket{\psi_i}\ket{i}$ is a purification.

%%%%%%%%%%%%%%%%%%%%%%%%%%%%%%%%%%%%%%%%%%%%%%%%%%%%%%%%%%%%%%%%%%%%%%

\item \todo[Add intuitive solution/explanation]

For this exercise it is necessary to review Postulate 3. For the system $\ket{AR}$, there is a set of measurement operators $\{M_m\}$. Only system $R$ is being measured, though. So, it is possible to define every measurement operator $M_m = I \otimes M_i$ where $I$ is the identity operator acting on system $A$ and $M_i$ is the measurement operator acting on system $R$ which corresponds to measuring the state $\ket{i}$.

The probability of measuring $\ket{i}$ is requested, i.e. $p(i)$. Using $\ket{\varphi_i} = \sqrt{p_i}\ket{\psi_i} \ket{i}$ temporarily for simplicity and by Equation (2.92):
\begin{align}
    p(i) &= \bra{\varphi_i}M_m^\dagger M_m \ket{\varphi_i} \\
    &= \bra{\varphi_i}(I \otimes M_i)^\dagger (I \otimes M_i) \ket{\varphi_i} \\
    &= \sqrt{p_i} \bra{\psi_i}\bra{i}(I \otimes M_i)^\dagger
        (I \otimes M_i) \sqrt{p_i}\ket{\psi_i}\ket{i} \\
    &= p_i \bra{\psi_i}\bra{i}(I^\dagger \otimes M_i^\dagger)
        (I \otimes M_i) \ket{\psi_i}\ket{i} \\
    &= p_i \bra{\psi_i}\bra{i}(I \otimes M_i^\dagger)
        (I \otimes M_i) \ket{\psi_i}\ket{i}
\end{align}

Using equation (2.48):
\begin{align}
    p(i) &= p_i (\bra{\psi_i}I \otimes \bra{i}M_i^\dagger)
        \ (I\ket{\psi_i} \otimes M_i\ket{i}) \\
    &= p_i (\bra{\psi_i} \otimes \bra{i}M_i^\dagger )
        \ (\ket{\psi_i} \otimes M_i\ket{i})
\end{align}

Then, by the definition of inner product (equation (2.49)):
\begin{align}
    p(i) = p_i \braket{\psi_i|\psi_i} \bra{i}M_i^\dagger M_i\ket{i}
\end{align}

Recall that $\ket{\psi_i}$ and $\ket{i}$ are orthonormal. Also, since $M_i$ is the measurement operator that corresponds to obtaining state $\ket{i}$, it is possible to consider $\ket{i}$ as a "measurement basis" defining $M_i = \ket{i}\bra{i}$. A similar example can be seen in Nielsen and Chuang's book in a paragraph between Equations (2.95) and (2.96). Therefore,
\begin{align}
    p(i) &= p_i \cdot 1 \cdot \bra{i}(\ket{i}\bra{i})^\dagger (\ket{i}\bra{i}) \ket{i} \\
    &= p_i \bra{i}(\ket{i}\bra{i}) (\ket{i}\bra{i}) \ket{i} \\
    &= p_i \braket{i|i} \braket{i|i} \braket{i|i} \\
    &= p_i \cdot 1 \cdot 1 \cdot 1 \\
    &= p_i
\end{align}

Hence, the probability of measuring state $\ket{i}$ is $p_i$. Now, it requested to obtain the state in system A, which is described by Postulate 3 as:
\begin{align}
    \frac{M_m \ket{\varphi_i}}{\sqrt{p_i}} &=
        \frac{(I \otimes M_i) \sqrt{p_i}\ket{\psi_i}\ket{i}}{\sqrt{p_i}} \\
    &= (I\ket{\psi_i}) \otimes (M_i\ket{i}) \\
    &= (I\ket{\psi_i}) \otimes (\ket{i}\braket{i|i}) \\
    &= \ket{\psi_i} \ket{i}
\end{align}

Therefore, the measurement of the system A will always be $\ket{\psi_i}$.


%%%%%%%%%%%%%%%%%%%%%%%%%%%%%%%%%%%%%%%%%%%%%%%%%%%%%%%%%%%%%%%%%%%%%%
\item c
\end{enumerate}

\hline