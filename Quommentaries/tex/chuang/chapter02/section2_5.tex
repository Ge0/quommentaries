\subsection{Section 2.5}
\subsubsection[Symmetry of Triple State Entanglement]
{Symmetry of $(\ket{00} + \ket{01} + \ket{11})/\sqrt{3}$}
\label{sec:nielsen-and-chuang-symmetry-of-triple-state-entanglement}

This subsection is dedicated to calculate $tr((\rho^A)^2)$
for $(\ket{00} + \ket{01} + \ket{11})/\sqrt{3}$. By Equation (2.138):

\begin{align}
    \rho^{AB} &= \frac{(\ket{00} + \ket{01} + \ket{11})}{\sqrt{3}}
        \frac{(\bra{00} + \bra{01} + \bra{11})}{\sqrt{3}} \\
        &= \frac{(\ket{00} + \ket{01} + \ket{11})(\bra{00} + \bra{01} + \bra{11})}{3}
\end{align}

Before using Equations (2.177) and (2.178) it is necessary to apply the distributive property.
However from Equation (2.178), the result will be similar to
$\sum_{ijkl} \ket{i}\bra{j} tr(\ket{k}\bra{l})$, where $i, j, k, l \in \{0, 1\}$.
Since $tr(\ket{a}\bra{b}) = \braket{b}{a}$
(check Section \ref{sec:nielsen-and-chuang-trace-of-ketbra-equals-braket}):
\begin{align}
    \sum_{ijkl} \ket{i}\bra{j} tr(\ket{k}\bra{l}) &=
        \sum_{ijkl} \ket{i}\bra{j}\braket{l}{k} \\
        &= \sum_{ijkl} \ket{i}\bra{j}\delta_{lk}
\end{align}

Therefore, when applying the distributive property,
it is not necessary to write $\ket{i}\bra{j}$ if $l \neq k$.
For instance,
$tr_B(\ket{00}\bra{01}) = \ket{0}\bra{0}tr(\ket{0}\bra{1})$
$= \ket{0}\bra{0}\braket{0}{1} = 0 \ket{0}\bra{0}$.
Also, since $\braket{i}{i} = 1$:
\begin{align}
    \rho^A &= \frac{\ket{0}\bra{0} + \ket{0}\bra{0} + \ket{0}\bra{1} +
        \ket{1}\bra{0} + \ket{1}\bra{1}}{3} \\
    &= \frac{2\ket{0}\bra{0} + \ket{0}\bra{1} +
        \ket{1}\bra{0} + \ket{1}\bra{1}}{3}
\end{align}

Now, calculating $(\rho^A)^2$:
\begin{align}
    (\rho^A)^2 &= \frac{(2\ket{0}\bra{0} + \ket{0}\bra{1} +
        \ket{1}\bra{0} + \ket{1}\bra{1})\ (2\ket{0}\bra{0} + \ket{0}\bra{1} +
        \ket{1}\bra{0} + \ket{1}\bra{1})}{3 \cdot 3} \\
    &= \frac{4\ket{0}\bra{0} + 2\ket{0}\bra{1} + \ket{0}\bra{0} + \ket{0}\bra{1} +
        \ket{1}\bra{1} + \ket{1}\bra{0} + \ket{1}\bra{1}}{9} \\
    &= \frac{5\ket{0}\bra{0} + 3\ket{0}\bra{1} + \ket{1}\bra{0} + 2\ket{1}\bra{1}}{9}
\end{align}

Now, calculate $tr((\rho^A)^2)$:
\begin{align}
    tr((\rho^A)^2) &= \frac{1}{9}\ tr(5\ket{0}\bra{0} + 3\ket{0}\bra{1} +
        \ket{1}\bra{0} + 2\ket{1}\bra{1}) \\
    &= \frac{1}{9}\ (5\braket{0}{0} + 3\braket{1}{0} +
        \braket{0}{1} + 2\braket{1}{1}) \\
    &= \frac{1}{9}\ (5 \cdot 1 + 3 \cdot 0 + 0 + 2 \cdot 1) \\
    &= \frac{1}{9}\ (5 + 2) \\
    &= \frac{7}{9}
\end{align}

Using an analogous line of thought $tr((\rho^B)^2) = \frac{7}{9}$ is obtained.

%%%%%%%%%%%%%%%%%%%%%%%%%%%%%%%%%%%%%%%%%%%%%%%%%%%%%%%%%%%%%%%%%%%%%%%%%%%%%%%%%%%%%%%

\subsubsection{Exercise 2.78}
\MakeUppercase{\underline{Product State if and only if Schimidt number 1}}.

Suppose state $\ket{\psi}$ is a product state of systems
$A$ and $B$, i.e. $A \otimes B$.
Then, there exist orthonormal states $\ket{a}$ and $\ket{b}$,
respectively for systems $A$ and $B$, such that
$\ket{\psi} = \ket{a} \ket{b}$.
Therefore, the only possible Schmidt decomposition for state $\ket{\psi}$ is
$\ket{\psi} = 1 \ket{a} \ket{b} + \sum_i 0 \ket{i_A} \ket{i_B}$ where
$\ket{i_A}$ and $\ket{i_B}$ are part of the orthonormal bases alongside
$\ket{a}$ and $\ket{b}$; in other words:
$\braket{a}{a} = \braket{i_A}{i_A} = 1$ and
$\braket{a}{i_A} = \braket{i_A}{a} = 0$,
analogously for $\ket{b}$ and $\ket{i_B}$.
As a consequence, the Schmidt number is 1
(refer to equation
$\ket{\psi} = 1 \ket{a} \ket{b} + \sum_i 0 \ket{i_A} \ket{i_B}$).

Suppose state $\ket{\psi}$ has Schmidt number 1.
Then, from Theorem 2.7,
$\ket{\psi} = \sum_i \lambda_i \ket{i_A} \ket{i_B}$.
Since $\ket{\psi}$ has Schmidt number 1,
there exist $\lambda_i = 1$ and $\lambda_j = 0$ such that
$\ket{\psi} = 1 \ket{i_A} \ket{i_B} + \sum_j 0 \ket{j_A} \ket{j_B}
= \ket{i_A} \ket{i_B}$.
Therefore, $\ket{\psi}$ is a product state.

Quod erat demonstrandum.

\MakeUppercase{\underline{Product State if and only if $\rho^A$ are pure states}}.

Suppose $\ket{\psi}$ is a product state of composite system
$A \otimes B$, then $\ket{\psi} = \ket{a}\ket{b}$ where
$\ket{a}$ and $\ket{b}$ are orthonormal states of systems
$A$ and $B$, respectively.

Then, by definition of density operator in Equation (2.138):
$\rho = 1 \cdot \ketbra{ab}{ab}$, and $\rho$ is pure if $tr(\rho^2) = 1$:
\begin{align}
    tr(\rho^2) &= tr(\ket{ab}\braket{ab}{ab}\bra{ab}) \\
    &= tr(\ketbra{ab}{ab}) \\
    &= tr(\rho)
\end{align}

Them, by Theorem 2.5, $tr(\rho) = 1$.

If $\rho$ is pure, consequently $\rho^A$ and $\rho^B$ are pure.
Otherwise, $\braket{a}{a} \neq 1 \neq \braket{b}{b}$,
which would be a contradiction with $\rho$ being pure.

The converse can be proved naturally following these steps reversely.
Suppose $\rho^A$ is pure. Then $\rho^B$ is pure. Then $\rho$ is pure.
Then $\ket{\psi}$ is a state product and can be written as
$\ket{\psi} = \ket{a} \ket{b}$.

Quod erat demonstrandum.

%%%%%%%%%%%%%%%%%%%%%%%%%%%%%%%%%%%%%%%%%%%%%%%%%%%%%%%%%%%%%%%%%%%%%%%%%%%%%%%%%%%%%%%

\subsubsection{Exercise 2.79}
In order to solve this exercise, refer back to Theorem 2.7 and
factor each state.

\begin{enumerate}
    \item \begin{align}
            \frac{\ket{00} + \ket{11}}{\sqrt2} &= \frac{1}{\sqrt2} \ket{0} \ket{0} +
                \frac{1}{\sqrt2} \ket{1} \ket{1} \\
            &= \sum_{i \in \{0, 1\}} \lambda_i \ket{i} \ket{i}
        \end{align}
        where $\lambda_0 = \lambda_1 = 1 / \sqrt{2}$.
    
    \item Since Schmidt's decomposition requires that $\ket{i_A}$ and $\ket{i_B}$ are
        orthonormal states and $\ket{+}$ and $\ket{-}$ are examples of such states:
        \begin{align}
            \frac{\ket{00} + \ket{01} + \ket{10} + \ket{11}}{2} &=
                \frac{\ket{0} + \ket{1}}{\sqrt2}\ \frac{\ket{0} + \ket{1}}{\sqrt2} \\
            &= \ket{+} \ket{+} \\
            &= 1 \cdot \ket{+} \ket{+} + 0 \cdot \ket{-} \ket{-} \\
            &= \sum_{i \in \{+, -\}} \lambda_i \ket{i} \ket{i}
        \end{align}
        where $\lambda_+ = 1$ and $\lambda_- = 0$.
        
    \item Analysing the state
        \footnote{Note that similarly to state $(\ket{00} + \ket{01} + \ket{11})/\sqrt{3}$
            described in Section
            \ref{sec:nielsen-and-chuang-symmetry-of-triple-state-entanglement}
        , $\ket{\psi}$ has symmetry $7/9$.
        Hence, obtaining its Schmidt Decomposition is not intuitive.}
        $\ket{\psi} = \frac{\ket{00} + \ket{01} + \ket{10}}{\sqrt3}$
        for both systems separately:
        
        For the first qubit: $\frac{\sqrt2 \ket{0} + \ket{1}}{\sqrt3}$.
        
        For the second qubit: $\frac{\sqrt2 \ket{0} + \ket{1}}{\sqrt3}$
        as well.
        
        It is easy to check that $\frac{\sqrt2 \ket{0} + \ket{1}}{\sqrt3}$ is        orthonormal.
        However, following this line of thought may lead to erroneous solutions.
        A bit more of cleverness is required:
        it is possible to use the interesting results obtained for
        $\rho^A$, $\rho^B$, and their eigenvalues as stated in the paragraph
        that follows Theorem 2.7.
        
        Then, calculate $\rho$ to obtain $\rho^A$ and $\rho^B$ afterwards.
        Since $\ket{\psi}$ is pure:
        \begin{align}
            \rho &= \ketbra{\psi}{\psi} \\
            %
            &= \frac{\ket{00} + \ket{01} + \ket{10}}{\sqrt3} \ 
                \frac{\bra{00} + \bra{01} + \bra{10}}{\sqrt3} \\
            %
            &= \frac{1}{3} \left[ \begin{matrix}
                    1 & 1 & 1 & 0 \\ 1 & 1 & 1 & 0 \\
                    1 & 1 & 1 & 0 \\ 0 & 0 & 0 & 0
                \end{matrix} \right]
        \end{align}
        
        Calculating $\rho^A$ using the definition in Equations (2.177) and (2.178),
        alongside $tr(\ketbra{\psi}{\varphi}) = \braket{\varphi}{\psi}$
        (Section \ref{sec:nielsen-and-chuang-trace-of-ketbra-equals-braket}):
        \begin{align}
            \rho^A &= \frac{1}{3} tr_B(\ (\ket{00} + \ket{01} + \ket{10})
                (\bra{00} + \bra{01} + \bra{10})\ ) \\
            &= \frac{1}{3} (\ \ketbra{0}{0} tr(\ketbra{0}{0} + \ketbra{1}{1}) +
                \ketbra{0}{1} tr(\ketbra{0}{0}) +
                \ketbra{1}{0} tr(\ketbra{0}{0}) +
                \ketbra{1}{1} tr(\ketbra{0}{0})
                \ ) \\
            &= \frac{1}{3} (\ 2 \ketbra{0}{0} + \ketbra{0}{1} + \ketbra{1}{0} +
                \ketbra{1}{1} \ ) \\
            &= \frac{1}{3} \left[ \begin{matrix}
                2 & 1 \\ 1 & 1
                \end{matrix} \right]
        \end{align}
        
        Also, calculating $\rho^B$ it is possible to verify that $\rho^B = \rho^A$.
        In order to write the Schmidt Decomposition, it is necessary to find the
        eigenvalues and eigenvectors of $\rho^A$:
        \begin{align}
            \left| \begin{matrix}
            2/3 - v & 1 \\ 1 & 1/3 - v
            \end{matrix} \right|
            = v^2 - v + \frac{1}{9} = 0
        \end{align}
        
        Solving the polynomial, the eigenvalues found are $v = \frac{3 \pm \sqrt5}{6}$.
        Calculate the corresponding eigenvectors $\ket{v_1}$ and $\ket{v_2}$.
        
        For $v_1 = \frac{3 + \sqrt5}{6}$: substitute and row reduce
        \begin{align}
            \left[ \begin{matrix}
            \frac{4}{6} - \frac{3 + \sqrt5}{6} & \frac{1}{3} \\
            \frac{1}{3} & \frac{2}{6} - \frac{3 + \sqrt5}{6}
            \end{matrix} \right]
            &=
            \left[ \begin{matrix}
            \frac{1 - \sqrt5}{6} & \frac{1}{3} \\
            \frac{1}{3} & \frac{-1 - \sqrt5}{6}
            \end{matrix} \right] \\
            %
            &\sim \left[ \begin{matrix}
                1 - \sqrt5 & 2 \\ 2 & -1 -\sqrt5 \end{matrix} \right]
            %
            &\sim \left[ \begin{matrix}
                1 - \sqrt5 & 2 \\ 0 & 0 \end{matrix} \right]
        \end{align}
        
        Then, the eigenspace of $v_1$ is
        $\left\{ \left[ \begin{matrix} \frac{-2}{1 - \sqrt5} \\
        1 \end{matrix} \right] \right\}$.
        The orthonormal vector of the eigenspace is of interest.
        Therefore, normalise $\ket{v_1}$:
        \begin{align}
            \ket{v_1} &= \frac{1}{\sqrt{\braket{v_1}{v_1}}} \ket{v_1} \\
            %
            &= \sqrt{\frac{2}{5 + \sqrt5}} \left[ \begin{matrix}
                \frac{-2}{1 - \sqrt5} \\ 1 \end{matrix} \right]
        \end{align}
        
        For $v_2 = \frac{3 - \sqrt5}{6}$, the normalised vector
        $\ket{v_2} = \sqrt{\frac{2}{5 - \sqrt5}} \left[ \begin{matrix}
                \frac{-2}{1 + \sqrt5} \\ 1 \end{matrix} \right]$
        is found.
        \footnote{Verify that the values found match
            $\braket{v_1}{v_2} = \braket{v_2}{v_1} = 0$,
            $\braket{v_1}{v_1} = \braket{v_2}{v_2} = 1$,
            and $\rho^A = \rho^B = \sum_i v_i \ketbra{v_i}{v_i}$}

        From the results that follow Theorem 2.7,
        it is known that $\rho^A = \sum_i \lambda_i^2 \ketbra{i_A}{i_A}$
        where $\lambda_i^2$ are the eigenvalues of $\rho^A$,
        i.e. $\lambda_i^2 = v_i$.
        Therefore, since $rho^A$ and $rho^B$ have the same eigenvalues,
        by Theorem 2.7, $\ket{\psi}$ has Schmidt Decomposition
        \begin{align}
            \ket{\psi} &= \sum_i \lambda_i \ket{i_A} \ket{i_B}
                = \sum_i \sqrt{v_i} \ketbra{v_i}{v_i}\\
            %
            \begin{split} &= \sqrt{\frac{3 + \sqrt5}{6}} \left( 
                    \sqrt{\frac{2}{5 + \sqrt5}} \left[ \begin{matrix}
                    \frac{-2}{1 - \sqrt5} \\ 1 \end{matrix} \right]
                \right) \otimes \left(
                    \sqrt{\frac{2}{5 + \sqrt5}} \left[ \begin{matrix}
                    \frac{-2}{1 - \sqrt5} \\ 1 \end{matrix} \right]
                \right)
                \\ &+
                \sqrt{\frac{3 - \sqrt5}{6}} \left(
                    \sqrt{\frac{2}{5 - \sqrt5}} \left[ \begin{matrix}
                    \frac{-2}{1 + \sqrt5} \\ 1 \end{matrix} \right]
                \right) \otimes \left(
                    \sqrt{\frac{2}{5 - \sqrt5}} \left[ \begin{matrix}
                    \frac{-2}{1 + \sqrt5} \\ 1 \end{matrix} \right]
                \right) \end{split} \\
            %
            &= \sqrt{\frac{3 + \sqrt5}{6}}\ \frac{2}{5 + \sqrt5}
                \left[ \begin{matrix}
                    \frac{4}{6 - 2 \sqrt5} \\[5pt] \frac{-2}{1 - \sqrt5} \\[5pt]
                    \frac{-2}{1 - \sqrt5} \\[5pt] 1
                \end{matrix} \right] +
                \sqrt{\frac{3 - \sqrt5}{6}}\ \frac{2}{5 - \sqrt5}
                \left[ \begin{matrix}
                    \frac{4}{6 + 2 \sqrt5} \\[5pt] \frac{-2}{1 + \sqrt5} \\[5pt]
                    \frac{-2}{1 + \sqrt5} \\[5pt] 1
                \end{matrix} \right]
                \label{eq:schmidt_dec_after_tensor_prod}
        \end{align}
        
        In order to keep calculating, it is necessary to rewrite
        $\sqrt{3 + \sqrt5}$ with the help of quadratic polynomials:
        \begin{align}
            \sqrt{3 + \sqrt5} &= \sqrt{\frac{1 + \sqrt5^2 + 2\sqrt5}{2}} \\
            &= \sqrt{\frac{(1 + \sqrt5)^2}{2}} \\
            &= \frac{1 + \sqrt5}{\sqrt2}
        \end{align}
        Similarly, $\sqrt{3 - \sqrt5} = \frac{1 - \sqrt5}{\sqrt2}$.
        
        Therefore, substituting in Equation \ref{eq:schmidt_dec_after_tensor_prod}:
        \begin{align}
            \ket{\psi} &= \frac{1 + \sqrt5}{\sqrt2 \sqrt6}\ \frac{2}{5 + \sqrt5}
                \left[ \begin{matrix}
                    \frac{4}{6 - 2 \sqrt5} \\[5pt] \frac{-2}{1 - \sqrt5} \\[5pt]
                    \frac{-2}{1 - \sqrt5} \\[5pt] 1
                \end{matrix} \right] +
                \frac{1 - \sqrt5}{\sqrt2 \sqrt6}\ \frac{2}{5 - \sqrt5}
                \left[ \begin{matrix}
                    \frac{4}{6 + 2 \sqrt5} \\[5pt] \frac{-2}{1 + \sqrt5} \\[5pt]
                    \frac{-2}{1 + \sqrt5} \\[5pt] 1
                \end{matrix} \right] \\
            %
            &= \frac{1 + \sqrt5}{\sqrt3}\ \frac{1}{5 + \sqrt5}
                \left[ \begin{matrix}
                    \frac{2}{3 - \sqrt5} \\[5pt] \frac{-2}{1 - \sqrt5} \\[5pt]
                    \frac{-2}{1 - \sqrt5} \\[5pt] 1
                \end{matrix} \right] +
                \frac{1 - \sqrt5}{\sqrt3}\ \frac{1}{5 - \sqrt5}
                \left[ \begin{matrix}
                    \frac{2}{3 + \sqrt5} \\[5pt] \frac{-2}{1 + \sqrt5} \\[5pt]
                    \frac{-2}{1 + \sqrt5} \\[5pt] 1
                \end{matrix} \right] \\
            &= \left[ \begin{matrix}
                \frac{2(1 + \sqrt5)}{\sqrt3(5 + \sqrt5)(3 - \sqrt5)} +
                    \frac{2(1 - \sqrt5)}{\sqrt3(5 - \sqrt5)(3 + \sqrt5)} \\[5pt]
                \frac{-2 (1 + \sqrt5)}{\sqrt3 (5 + \sqrt5)(1 - \sqrt5)} +
                    \frac{-2 (1 - \sqrt5)}{\sqrt3 (5 - \sqrt5)(1 + \sqrt5)} \\[5pt]
                \frac{-2 (1 + \sqrt5)}{\sqrt3 (5 + \sqrt5)(1 - \sqrt5)} +
                    \frac{-2 (1 - \sqrt5)}{\sqrt3 (5 - \sqrt5)(1 + \sqrt5)} \\[5pt]
                \frac{1 + \sqrt5}{\sqrt3 (5 + \sqrt5)} +
                    \frac{1 - \sqrt5}{\sqrt3 (5 - \sqrt5)}
                \end{matrix} \right]
                = \left[ \begin{matrix}
                \frac{2(1 + \sqrt5)}{\sqrt3(10 -2\sqrt5)} +
                    \frac{2(1 - \sqrt5)}{\sqrt3(10 + 2\sqrt5)} \\[5pt]
                \frac{-2 (1 + \sqrt5)}{\sqrt3 (-4 \sqrt5)} +
                    \frac{-2 (1 - \sqrt5)}{\sqrt3 (4 \sqrt5)} \\[5pt]
                \frac{-2 (1 + \sqrt5)}{\sqrt3 (-4 \sqrt5)} +
                    \frac{-2 (1 - \sqrt5)}{\sqrt3 (4 \sqrt5)} \\[5pt]
                \frac{1 + \sqrt5}{\sqrt3 (5 + \sqrt5)} +
                    \frac{1 - \sqrt5}{\sqrt3 (5 - \sqrt5)}
                \end{matrix} \right] \\
            &= \left[ \begin{matrix}
                \frac{(1 + \sqrt5)(5 + \sqrt5) + (1 - \sqrt5)(5 - \sqrt5)}
                    {\sqrt3(5 + \sqrt5)(5 - \sqrt5)} \\[5pt]
                \frac{1 + \sqrt5 -1 + \sqrt5}{2 \sqrt5 \sqrt3} \\[5pt]
                \frac{1 + \sqrt5 -1 + \sqrt5}{2 \sqrt5 \sqrt3} \\[5pt]
                \frac{(1 + \sqrt5)(5 - \sqrt5) + (1 - \sqrt5)(5 + \sqrt5)}{20\sqrt3}
                \end{matrix} \right]
                = \left[ \begin{matrix}
                \frac{10 + 6\sqrt5 + 10 - 6\sqrt5}{20\sqrt3} \\[5pt]
                \frac{2\sqrt5}{2\sqrt5\sqrt3} \\[5pt]
                \frac{2\sqrt5}{2\sqrt5\sqrt3} \\[5pt]
                \frac{4\sqrt5 -4\sqrt5}{20\sqrt3}
                \end{matrix} \right]
                = \left[ \begin{matrix}
                \frac{1}{\sqrt3} \\[5pt] \frac{1}{\sqrt3} \\[5pt]
                \frac{1}{\sqrt3} \\[5pt] 0
                \end{matrix} \right] \\
            &= \frac{\ket{00} + \ket{01} + \ket{10}}{\sqrt3}
        \end{align}
        as requested.
\end{enumerate}

%%%%%%%%%%%%%%%%%%%%%%%%%%%%%%%%%%%%%%%%%%%%%%%%%%%%%%%%%%%%%%%%%%%%%%%%%%%%%%%%%%%%%%%

\subsubsection{Equations 2.208 and 2.209}
\label{sec:nielsen-and-chuang-equations-2-208-209}
When I firstly read these equations I thought there was a possibility that
an extra explanation would be necessary.
This thought raised, most likely,
because I was unaccustomed to Tensor Product Properties and
the Reduced Density Operator.

Using $\ket{AR}$ as defined in Equation 2.207:
\begin{align}
    \ketbra{AR}{AR} &= \left( \sum_i \sqrt{p_i} \ket{i^A}\ket{i^R} \right)
        \left( \sum_j \sqrt{p_j} \bra{j^A}\bra{j^R} \right) \\
    &= \left( \sum_i \sqrt{p_i} \ket{i^A} \otimes \ket{i^R} \right)
        \left( \sum_j \sqrt{p_j} \bra{j^A} \otimes \bra{j^R} \right) \\
    &= \sum_{ij} \sqrt{p_i p_j} (\ \ket{i^A} \otimes \ket{i^R}\ )
        (\ \bra{j^A} \otimes \bra{j^R}\ )
\end{align}

Then, by applying the properties as similarly defined in Equation 2.46:
\begin{align}
    \sum_{ij} \sqrt{p_i p_j} (\ \ket{i^A} \otimes \ket{i^R}\ )
        (\ \bra{j^A} \otimes \bra{j^R}\ ) &=
        \sum_{ij} \sqrt{p_i p_j} \ketbra{i^A}{j^A} \otimes \ketbra{i^R}{j^R}
\end{align}

Therefore, using the definition of the Reduced Density Operator (Equation 2.178):
\begin{align}
    tr_R(\ketbra{AR}{AR}) &= tr_R \left( \sum_{ij} \sqrt{p_i p_j}
        \ketbra{i^A}{j^A} \otimes \ketbra{i^R}{j^R} \right) \\
    &= \sum_{ij} \sqrt{p_i p_j} \ketbra{i^A}{j^A} tr(\ketbra{i^R}{j^R}) \\
\end{align}

Thus obtaining Equation 2.208.

In order to obtain Equation 2.209 it is necessary to apply
\hyperref[sec:nielsen-and-chuang-trace-of-ketbra-equals-braket]
{$tr(\ketbra{\psi}{\varphi}) = \braket{\varphi}{\psi}$}.
Also, recall that $\ket{i^R}$ are orthonormal states and that
$\delta_{ij} = 1$ if $i = j$ and $\delta_{ij} = 0$ otherwise.
Hence,
\begin{align}
    \sum_{ij} \sqrt{p_i p_j} \ketbra{i^A}{j^A} tr(\ketbra{i^R}{j^R}) &=
        \sum_{ij} \sqrt{p_i p_j} \ketbra{i^A}{j^A} \braket{j^R}{i^R} \\
    &= \sum_{ij} \sqrt{p_i p_j} \ketbra{i^A}{j^A} \delta_{ij}
\end{align}

as required.

%%%%%%%%%%%%%%%%%%%%%%%%%%%%%%%%%%%%%%%%%%%%%%%%%%%%%%%%%%%%%%%%%%%%%%%%%%%%%%%%%%%%%%%
\subsubsection{Exercise 2.82}
\label{sec:nielsen-and-chuang-exercise-2-82}

\begin{enumerate}
\item From the definition of purification, it is necessary to prove that $\rho^A = tr_R(\ket{AR}\bra{AR})$.
    
    Suppose $\sum_i \sqrt(p_i) \ket{\psi_i} \ket{i}$ is a purification. From equation (2.138) of Nielsen and Chuang's book:
    \begin{align}
        \rho^{AB} = \sum_{ij} \left( \sqrt{p_i} \ket{\psi_i} \ket{i} \right) \left( \sqrt{p_j} \ket{\psi_j} \ket{j} \right) ^ \dagger
    \end{align}
    
    Since $p_j \in \mathbb{R}$, $p_j^\dagger = p_j$. And since it is a scalar:
    \begin{align}
        \rho^{AB} = \sum_{ij} \sqrt{p_i p_j} \left( \ket{\psi_i} \ket{i} \right) \left( \ket{\psi_j} \ket{j} \right) ^ \dagger
    \end{align}
    
    Recall that for any states $\ket{\varphi}$ and $\ket{\gamma}$ writing $\ket{\varphi} \ket{\gamma}$ is the same as $\ket{\varphi} \otimes \ket{\gamma}$. Then, applying equation (2.48):
    \begin{align}
        \rho^{AB} &= \sum_{ij} \sqrt{p_i p_j} \left( \ket{\psi_i} \otimes \ket{i} \right)
            \left( \bra{\psi_j} \otimes \bra{j} \right) \\
        \rho^{AB} &= \sum_{ij} \sqrt{p_i p_j} (\ket{\psi_i}\bra{\psi_j}) \otimes (\ket{i}\bra{j})
    \end{align}
    
    If $\sum_i \sqrt{p_i} \ket{\psi_i}\ket{i}$ is a purification, then
    $\rho^A = tr_B(\ (\ket{\psi_i}\ket{i}) (\bra{\psi_i}\bra{i})\ )$.
    The question gives $\rho = \sum_i p_i \ket{\psi_i}\bra{\psi_i}$, in other words $\rho = \rho^A$.
    
    Now, it is necessary to calculate $tr_B(\ (\ket{\psi_i}\ket{i}) (\bra{\psi_i}\bra{i})\ )$. Then, using the definitions given in equations (2.177) and (2.178) of Nielsen and Chuang's book:
    \begin{align}
        tr_B(\rho^{AB}) &= tr_B \left(\sum_{ij} \sqrt{p_i p_j} (\ket{\psi_i}\bra{\psi_j}) \otimes (\ket{i}\bra{j}) \right) \\
        &= \sum_{ij} \sqrt{p_i p_j} (\ket{\psi_i}\bra{\psi_j}) tr(\ket{i}\bra{j})
    \end{align}
    
    Since $tr(\ket{a}\bra{b}) = \braket{b}{a}$
    (refer back to Section \ref{sec:nielsen-and-chuang-trace-of-ketbra-equals-braket}):
    \begin{align}
        tr_B(\rho^{AB}) &= \sum_{ij} \sqrt{p_i p_j}\ (\ket{\psi_i}\bra{\psi_j}) \braket{j}{i} \\
        &= \sum_{ij} \sqrt{p_i p_j}\ \ket{\psi_i}\bra{\psi_j}\ \delta_{ij} \\
        &= \sum_i \sqrt{p_i p_i}\ \ket{\psi_i}\bra{\psi_i} \\
        &= \sum_i p_i\ket{\psi_i}\bra{\psi_i}
    \end{align}
    
    Since $\rho^A = \sum_i p_i \ket{\psi_i}\bra{\psi_i} = tr_B(\rho^{AB})$,
    it is possible to conclude that $\sum_i \sqrt{p_i} \ket{\psi_i}\ket{i}$ is a purification.

%%%%%%%%%%%%%%%%%%%%%%%%%%%%%%%%%%%%%%%%%%%%%%%%%%%%%%%%%%%%%%%%%%%%%%

\item \todo{Add intuitive solution/explanation}

    For this exercise, it is necessary to review Postulate 3. For the system $\ket{AR}$, there is a set of measurement operators $\{M_m\}$. Only system $R$ is being measured, though. So, it is possible to define every measurement operator $M_m = I \otimes M_i$ where $I$ is the identity operator acting on system $A$ and $M_i$ is the measurement operator acting on system $R$ which corresponds to measuring the state $\ket{i}$.
    
    The probability of measuring $\ket{i}$ is requested, i.e. $p(i)$. Using $\ket{\varphi_i} = \sqrt{p_i}\ket{\psi_i} \ket{i}$ temporarily for simplicity and by Equation (2.92):
    \begin{align}
        p(i) &= \bra{\varphi_i}M_m^\dagger M_m \ket{\varphi_i} \\
        &= \bra{\varphi_i}(I \otimes M_i)^\dagger (I \otimes M_i) \ket{\varphi_i} \\
        &= \sqrt{p_i} \bra{\psi_i}\bra{i}(I \otimes M_i)^\dagger
            (I \otimes M_i) \sqrt{p_i}\ket{\psi_i}\ket{i} \\
        &= p_i \bra{\psi_i}\bra{i}(I^\dagger \otimes M_i^\dagger)
            (I \otimes M_i) \ket{\psi_i}\ket{i} \\
        &= p_i \bra{\psi_i}\bra{i}(I \otimes M_i^\dagger)
            (I \otimes M_i) \ket{\psi_i}\ket{i}
    \end{align}
    
    Using equation (2.48):
    \begin{align}
        p(i) &= p_i (\bra{\psi_i}I \otimes \bra{i}M_i^\dagger)
            \ (I\ket{\psi_i} \otimes M_i\ket{i}) \\
        &= p_i (\bra{\psi_i} \otimes \bra{i}M_i^\dagger )
            \ (\ket{\psi_i} \otimes M_i\ket{i})
    \end{align}
    
    Then, by the definition of inner product (equation (2.49)):
    \begin{align}
        p(i) = p_i \braket{\psi_i}{\psi_i} \bra{i}M_i^\dagger M_i\ket{i}
    \end{align}
    
    Recall that $\ket{\psi_i}$ and $\ket{i}$ are orthonormal. Also, since $M_i$ is the measurement operator that corresponds to obtaining state $\ket{i}$, it is possible to consider $\ket{i}$ as a "measurement basis" defining $M_i = \ket{i}\bra{i}$. A similar example can be seen in Nielsen and Chuang's book in a paragraph between Equations (2.95) and (2.96). Therefore,
    \begin{align}
        p(i) &= p_i \cdot 1 \cdot \bra{i}(\ket{i}\bra{i})^\dagger (\ket{i}\bra{i}) \ket{i} \\
        &= p_i \bra{i}(\ket{i}\bra{i}) (\ket{i}\bra{i}) \ket{i} \\
        &= p_i \braket{i}{i} \braket{i}{i} \braket{i}{i} \\
        &= p_i \cdot 1 \cdot 1 \cdot 1 \\
        &= p_i
    \end{align}
    
    Hence, the probability of measuring state $\ket{i}$ is $p_i$. Now, it is requested to obtain the state of system A after the measurement $M_m$, which is described by Postulate 3 as:
    \begin{align}
        \frac{M_m \ket{\varphi_i}}{\sqrt{p_i}} &=
            \frac{(I \otimes M_i) \sqrt{p_i}\ket{\psi_i}\ket{i}}{\sqrt{p_i}} \\
        &= (I\ket{\psi_i}) \otimes (M_i\ket{i}) \\
        &= (I\ket{\psi_i}) \otimes (\ket{i}\braket{i}{i}) \\
        &= \ket{\psi_i} \ket{i}
    \end{align}
    
    Therefore, the measurement of the system A will always be $\ket{\psi_i}$.


%%%%%%%%%%%%%%%%%%%%%%%%%%%%%%%%%%%%%%%%%%%%%%%%%%%%%%%%%%%%%%%%%%%%%%
\item
    \href{https://github.com/goropikari}{Goropikari} attempted to solve this exercise as follows\footnote
    {The original code can be found at \href{https://github.com/goropikari/SolutionForQuantumComputationAndQuantumInformation/blob/master/chapter/chapter2.tex}
    {SolutionForQuantumComputationAndQuantumInformation}}:

    \begin{quotation}
        Suppose $\ket{AR}$ is a purification of $\rho$ such that $\ket{AR} = \sum_i \sqrt{p_i} \ket{\psi_i} \ket{r_i}$.
        By exercise 2.81, the others purification is written as $(I \otimes U) \ket{AR}$.
        \begin{align*}
        	(I \otimes U)  \ket{AR} &= (I \otimes U) \sum_i \sqrt{p_i} \ket{\psi_i} \ket{r_i}\\
        		&= \sum_i \sqrt{p_i} \ket{\psi_i} U\ket{r_i}\\
        		&= \sum_i \sqrt{p_i} \ket{\psi_i} \ket{i}
        \end{align*}
        where $U = \sum_i \ket{i}\bra{r_i}$.
        
        By (2), if we measure the system $R$ w.r.t $\ket{i}$, post-measurement state for system $A$ is $\ket{\psi_i}$ with probability $p_i$, which prove the assertion.
    \end{quotation}
    
    \todo{Is the previous solution plausible in some way?}
    
    However, if system $R$ is measured with respected to $\ket{i}$
    (that is, the measurement operator $M_m = I \otimes \ket{i}\bra{i}$
    is applied to $\sum_i \sqrt{p_i} \ket{\psi_i}\ket{i}$)
    the same result of Exercise 2.82(2) will be achieved.
    
    A similar way to try to solve this problem is:
    define $\ket{AR_i} = \ket{\psi_i}\ket{r_i}$
    and use the measurement operator $M_m = I \otimes \ket{i}\bra{i}$.
    Then, the probability of measuring $\ket{i}$ is calculated
    according to Equation (2.92):
    \begin{align}
        p(i) &= (\sqrt{p_i}\ (I \otimes \ketbra{i}{i}) \ket{\psi_i}\ket{r_i})^\dagger
            \ \sqrt{p_i}\ (I \otimes \ketbra{i}{i}) \ket{\psi_i}\ket{r_i} \\
        &= p_i \bra{r_i}\bra{\psi_i} (\ketbra{i}{i} \otimes I)
            \ (I \otimes \ketbra{i}{i}) \ket{\psi_i}\ket{r_i}
    \end{align}
    
    Then, using Equations (2.48) and (2.49):
    \begin{align}
        p(i) &= p_i  (\bra{r_i} \ketbra{i}{i}) \otimes (\bra{\psi_i} I)
            \ (I \ket{\psi_i}) \otimes (\ketbra{i}{i} \ket{r_i}) \\
        &= p_i (\braket{r_i}{i} \bra{i}) \otimes \bra{\psi_i}
            \ \ket{\psi_i} \otimes (\braket{i}{r_i} \ket{i}) \\
        &= p_i \braket{r_i}{i} \braket{i}{r_i}\ (\bra{i} \bra{\psi_i} \ket{\psi_i} \ket{i}) \\
        &= p_i\ \norm{\braket{i}{r_i}}^2
    \end{align}
    
    if $\norm{\braket{i}{r_i}}^2 = 1/p_i$, the desired result would obtained.
    However, $0 < p_i < 1$, then $1/p_i > 1$, which is not possible
    since both $\ket{i}$ and $\ket{r_i}$ are orthonormal vectors.
    Independently, if the post-measurement state is calculated as given by Equation (2.93):
    \begin{align}
        \frac{M_m\ket{\psi}}{\sqrt{\bra{\psi}M_m^\dagger M_m\ket{\psi}}} &=
            \frac{(I \otimes \ketbra{i}{i})\sqrt{p_i}\ket{\psi_i}\ket{r_i}}
            {\sqrt{p_i \norm{\braket{i}{r_i}}^2}} \\
        &= \frac{\sqrt{p_i} \ket{\psi_i} \ketbra{i}{i} \ket{r_i}}
            {\sqrt{p_i}\ \norm{\ketbra{i}{r_i}}} \\
        &= \frac{\braket{i}{r_i} \ket{\psi_i} \ket{i}}{\norm{\braket{i}{r_i}}}
    \end{align}
    
    Therefore, $\frac{\braket{i}{r_i}}{\sqrt{\braket{i}{r_i}\braket{r_i}{i}}} = p_i$.
    \todo{In conclusion ?????}
    
\end{enumerate}