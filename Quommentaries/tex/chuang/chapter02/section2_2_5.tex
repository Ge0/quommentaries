\subsection{Section 2.2.5}
\subsubsection{Equation (2.116)}

This definition may be rather confusing since $\vec{v}$ is defined
but the definition of $\vec{\sigma}$ is not recapitulated.
More specifically, $\sigma_1$, $\sigma_1$, and $\sigma_3$ were
defined in Table 2.2 of the book.
However, since $X$, $Y$ and $Z$ are used more frequently to
denote the Pauli Matrices, the reader may not remind of the
equivalent $\sigma$ notation.

In order to reduce the calculi on Exercise 2.60
(section \ref{sec:nielsen_and_chuang_exercise_2_60}),
The matrix form of $\vec{v} \vec{\sigma}$ is computed in this section.

Recall that $\sigma_1 \equiv X$, $\sigma_2 \equiv Y$,
and $\sigma_3 \equiv Z$, which have matrix form as defined in
the book's Table 2.2.
Since $\vec{v}$ is a vector with components
$v_1, v_2, v_3 \in \mathbb{R}$,
it is possible to interpret $\vec{\sigma}$ as being a vector of
matrices, i.e. $\vec{\sigma} \in (\mathbb{R}^{2 \times 2})^3$
Therefore:

\begin{align}
    \vec{v} \vec{\sigma} &= v_1 \left[ \begin{matrix}
        0 & 1 \\ 1 & 0 \end{matrix}\right] +
        v_2 \left[ \begin{matrix}
        0 & -i \\ i & 0 \end{matrix} \right] +
        v_3 \left[ \begin{matrix}
        1 & 0 \\ 0 & -1\end{matrix} \right]
        \\
        &= \left[ \begin{matrix}
        v_3 & v_1 -v_2 i \\ v_1 + v_2 i & v_3
        \end{matrix} \right]
\end{align}

%%%%%%%%%%%%%%%%%%%%%%%%%%%%%%%%%%%%%%%%%%%%%%%%%%%%%%%%%%%%%%%%%%%
\subsubsection{Exercise 2.60}
\label{sec:nielsen_and_chuang_exercise_2_60}
\todo{Write solution}