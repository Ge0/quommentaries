\subsection{Section 2.2.5}
\subsubsection{Equation (2.116)}
\label{sec:nielsen-and-chuang-equation-2-116}

This definition may be rather confusing since $\vec{v}$ is defined
but the definition of $\vec{\sigma}$ is not recapitulated.
More specifically, $\sigma_1$, $\sigma_1$, and $\sigma_3$ were
defined in Table 2.2 of the book.
However, since $X$, $Y$ and $Z$ are used more frequently to
denote the Pauli Matrices, the reader may not remind of the
equivalent $\sigma$ notation.

In order to reduce the calculi on Exercise 2.60
(section \ref{sec:nielsen-and-chuang-exercise-2-60}),
The matrix form of $\vec{v} \vec{\sigma}$ is computed in this section.

Recall that $\sigma_1 \equiv X$, $\sigma_2 \equiv Y$,
and $\sigma_3 \equiv Z$, which have matrix form as defined in
the book's Table 2.2.
Since $\vec{v}$ is a vector with components
$v_1, v_2, v_3 \in \mathbb{R}$,
it is possible to interpret $\vec{\sigma}$ as being a vector of
matrices, i.e. $\vec{\sigma} \in (\mathbb{R}^{2 \times 2})^3$
Therefore:

\begin{align}
    \vec{v} \vec{\sigma} &= v_1 \left[ \begin{matrix}
        0 & 1 \\ 1 & 0 \end{matrix}\right] +
        v_2 \left[ \begin{matrix}
        0 & -i \\ i & 0 \end{matrix} \right] +
        v_3 \left[ \begin{matrix}
        1 & 0 \\ 0 & -1\end{matrix} \right]
        \\
        &= \left[ \begin{matrix}
        v_3 & v_1 -v_2 i \\ v_1 + v_2 i & - v_3
        \end{matrix} \right]
\end{align}

%%%%%%%%%%%%%%%%%%%%%%%%%%%%%%%%%%%%%%%%%%%%%%%%%%%%%%%%%%%%%%%%%%%
\subsubsection{Exercise 2.60}
\label{sec:nielsen-and-chuang-exercise-2-60}
This section will only find the requested eigenvalues,
and the Projector given by $P_+$.
The projector given by $P_-$ can be found by following the same
steps as $P_+$'s solution.

\MakeUppercase{\underline{The eigenvalues}} of
$\vec{v} \vec{\sigma}$ can be found by using
basic Linear Algebra knowledge:
$det(\vec{v} \vec{\sigma} - \lambda I) = 0$.
Therefore, referring to Section \ref{sec:nielsen-and-chuang-equation-2-116},
calculate

\begin{align}
    det(\vec{v} \vec{\sigma} - \lambda I) &= 0 \\
    \left| \begin{matrix}
        v_3 - \lambda & v_1 - v_2i \\ v_1 + v_2i & -v_3 - \lambda
        \end{matrix} \right| &= 0 \\
    \notag \\
    \lambda^2 - v_3^2 - (v_1^2 + v_2^2) &= 0 \\
    \lambda^2 &= v_1^2 + v_2^2 + v_3^2
\end{align}

A bit of cleverness is required here.
Recall that just before the definition of
\hyperref[sec:nielsen-and-chuang-equation-2-116]{Equation 2.116},
$\vec{v}$ is supposed to be a unit vector.
This means that $\vec{v} \cdot \vec{v} = 1$.
Since $\vec{v} \in \mathbb{R}^3$),
$\vec{v} \cdot \vec{v} = v_1 \cdot v_1 +
v_2 \cdot v_2 + v_3 \cdot v_3$.
Therefore, $v_1^2 + v_2^2 + v_3^2 = 1$.
Plugging this into the previous result to find the values of $\lambda$:

\begin{align}
    \lambda^2 &= 1 \\
    \lambda &= \pm 1
\end{align}

as requested.

\MakeUppercase{\underline{To find the Projector}} $P_+$ it is necessary to
calculate the eigenspace of the eigenvector of $+1$.
In order to find the eigenspace, basic Linear Algebra knowledge may be used.
Hence, for $\lambda = +1$ and applying row reducing:

\begin{align}
    \left[ \begin{matrix}
        v_3 - \lambda & v_1 - v_2i \\ v_1 + v_2i & -v_3 - \lambda
        \end{matrix} \right]
        &=
        \left[ \begin{matrix}
            v_3 - 1 & v_1 - v_2i \\ v_1 + v_2i & -v_3 - 1
        \end{matrix} \right] \\ \notag \\
    &= \left[ \begin{matrix}
            (v_3 - 1)(v_1 + v_2i) & v_1^2 + v_2^2 \\
            (v_1 + v_2i)(v_3 - 1) & 1 -v_3^2
        \end{matrix} \right]
\end{align}

Note that $1 - v_3^2 = v_1^2 + v_2^2$,
since $v_1^2 + v_2^2 + v_3^2 = 1$. Therefore,

\begin{align}
    \left[ \begin{matrix}
            (v_3 - 1)(v_1 + v_2i) & v_1^2 + v_2^2 \\
            (v_1 + v_2i)(v_3 - 1) & 1 -v_3^2
        \end{matrix} \right]
        &=
        \left[ \begin{matrix}
            (v_3 - 1)(v_1 + v_2i) & v_1^2 + v_2^2 \\
            (v_3 - 1)(v_1 + v_2i) & v_1^2 + v_2^2
        \end{matrix} \right] \\ \notag \\
    &= \left[ \begin{matrix}
            (v_3 - 1)(v_1 + v_2i) & v_1^2 + v_2^2 \\
            0 & 0
        \end{matrix} \right]
\end{align}

Therefore, if $t$ is a scalar,
the eigenspace can be given by:
\begin{align}
    t \left[ \begin{matrix}
        \frac{-(v_1 - v_2i)}{v_3 - 1} \\ 1
    \end{matrix} \right]
\end{align}
because:
\begin{align}
    (v_3 - 1)(v_1 + v_2i) \cdot
    \frac{-(v_1 - v_2i)}{v_3 - 1} +
    (v_1^2 + v_2^2) \cdot 1 &= 0 \\
    -(v_1^2 + v_2^2) + (v_1^2 + v_2^2) &= 0
\end{align}

However, it is not possible to use the definition
and keep calculating with $P_m = \ketbra{m}{m}$,
where $\ket{m} = \left[ \begin{matrix}
        \frac{-(v_1 - v_2i)}{v_3 - 1} \\ 1
    \end{matrix} \right]$
because it is necessary that $\ket{m}$ is unitary
($\braket{m}{m} = 1$).
And, if $\braket{m}{m}$ is calculated, the following
result would be obtained:
\begin{align}
    \braket{m}{m} &= \left[ \begin{matrix}
        \frac{-(v_1 + v_2i)}{v_3 - 1} & 1
        \end{matrix} \right]
        %inner product
        \left[ \begin{matrix}
            \frac{-(v_1 - v_2i)}{v_3 - 1} \\ 1
        \end{matrix} \right] \\
    &= \left[ \begin{matrix}
        \frac{-v_1 - v_2i}{v_3 - 1} & 1
        \end{matrix} \right]
        %inner product
        \left[ \begin{matrix}
            \frac{-v_1 + v_2i}{v_3 - 1} \\ 1
        \end{matrix} \right] \\
    &= \frac{v_1^2 + v_2^2}{(v_3 - 1)^2} + 1 \\
    &= \frac{1 - v_3^2}{{(v_3 - 1)^2}} +
        \frac{{(v_3 - 1)^2}}{{(v_3 - 1)^2}} \\
    &= \frac{1 - v_3^2 + v_3^2 - 2v_3 + 1}
        {(v_3 - 1)^2} \\
    &= \frac{-2v_3 + 2}{(v_3 - 1)^2} \\
    &= \frac{-2 (v_3 - 1)}{(v_3 - 1)^2} \\
    &= - \frac{2}{v_3 - 1}
\end{align}

Hence, it is necessary to normalize $\ket{m}$.
Recall that the norm of a vector $\ket{m}$ is given by
$\sqrt{\braket{m}{m}}$.
To normalize a vector, divide it by its norm.
\begin{align}
    \ket{\psi} &= \frac{\ket{m}}{\sqrt{\braket{m}{m}}} \\
    &= \ket{m} / \sqrt{- \frac{2}{v_3 - 1}} \\
    &= \sqrt{- \frac{v_3 - 1}{2}} \ket{m} \\
    &= \frac{i}{\sqrt2} \sqrt{v_3 - 1} \ket{m}
\end{align}

However, this would not be right because
$\sqrt{\braket{m}{m}} \geq 0$ and
$\sqrt{\braket{m}{m}} \in \mathbb{R}$.
It is necessary to rearrange the value of $\braket{m}{m}$:
\begin{align}
    \ket{\psi} &= \sqrt{- \frac{v_3 - 1}{2}} \ket{m} \\
    &= \sqrt{\frac{1 - v_3}{2}} \ket{m} \\
    &= \frac{1}{\sqrt2} \sqrt{1 - v_3}
        \left[ \begin{matrix}
            \frac{-(v_1 - v_2i)}{v_3 - 1} \\ 1
        \end{matrix} \right] \\
    &= \frac{1}{\sqrt2} \sqrt{1 - v_3}
        \left[ \begin{matrix}
            \frac{v_1 - v_2i}{1 - v_3} \\ 1
        \end{matrix} \right] \\
    &= \frac{1}{\sqrt2} \left[ \begin{matrix}
            \frac{v_1 - v_2i}{\sqrt{1 - v_3}} \\
            \sqrt{1 - v_3}
        \end{matrix} \right]
\end{align}

Now that the normalized vector was obtained,
it is possible to calculate the respective Projector by:
\begin{align}
    \ketbra{\psi}{\psi} &= \frac{1}{\sqrt2}
        \left[ \begin{matrix}
            \frac{v_1 - v_2i}{\sqrt{1 - v_3}} \\
            \sqrt{1 - v_3}
        \end{matrix} \right]
        %times
        \frac{1}{\sqrt2}
        \left[ \begin{matrix}
            \frac{v_1 + v_2i}{\sqrt{1 - v_3}} &
            \sqrt{1 - v_3}
        \end{matrix} \right] \\
    &= \frac{1}{2} \left[ \begin{matrix}
            \frac{v_1 - v_2i}{\sqrt{1 - v_3}} \\
            \sqrt{1 - v_3}
        \end{matrix} \right]
        %times
        \left[ \begin{matrix}
            \frac{v_1 + v_2i}{\sqrt{1 - v_3}} &
            \sqrt{1 - v_3}
        \end{matrix} \right] \\
    &= \frac{1}{2} \left[ \begin{matrix}
        \frac{v_1^2 + v_2^2}{1 - v_3} & v_1 - v_2i \\
        v_1 + v_2i & 1 - v_3
        \end{matrix} \right] \\
    &= \frac{1}{2} \left[ \begin{matrix}
        \frac{1 - v_3^2}{1 - v_3} & v_1 - v_2i \\
        v_1 + v_2i & 1 - v_3
        \end{matrix} \right] \\
    &= \frac{1}{2} \left[ \begin{matrix}
        \frac{(1 + v_3)(1 - v_3)}{1 - v_3} & v_1 - v_2i \\
        v_1 + v_2i & 1 - v_3
        \end{matrix} \right] \\
    &= \frac{1}{2} \left[ \begin{matrix}
        1 + v_3 & v_1 - v_2i \\
        v_1 + v_2i & 1 - v_3
        \end{matrix} \right] \\
    &= (I + \vec{v} \vec{\sigma}) / 2 = P_+
\end{align}

as requested.
$P_-$ can be easily obtained following the same steps,
but with $\lambda = -1$.

%%%%%%%%%%%%%%%%%%%%%%%%%%%%%%%%%%%%%%%%%%%%%%%%%%%%%%%%%%%%%%%%%%%%%%%%%%%%
\subsubsection{Exercise 2.61}
This Exercise can be done very easily by using Equations (2.103) and (2.104)
alongside the value of $P_+$ obtained in
\hyperref[sec:nielsen-and-chuang-exercise-2-60]{Exercise 2.60}.

The probability can be calculated by using Equation (2.103):
\begin{align}
    p(+1) &= \bra{0}P_+\ket{0} \\
    &= [ \begin{matrix} 1 & 0 \end{matrix} ]
        \ (I + \vec{v} \vec{\sigma}) / 2
        \ \left[ \begin{matrix} 1 \\ 0 \end{matrix} \right] \\
    &= \frac{1}{2} \left[ \begin{matrix} 1 + v_3 & v_1 - v_2i \end{matrix} \right]
        \left[ \begin{matrix} 1 \\ 0 \end{matrix} \right] \\
    &= \frac{1 + v_3}{2}
\end{align}

Then, using Equation (2.104) to obtain the state of the system
after the measurement:
\begin{align}
    \ket{\psi} &= \frac{P_+ \ket{0}}{\sqrt{p(+1)}} \\
    &= \frac{1}{\sqrt{p(+1)}} \frac{1}{2} (I + \vec{v} + \vec{\sigma})
        \left[ \begin{matrix} 1 \\ 0 \end{matrix} \right] \\
    &= \frac{\sqrt{2}}{\sqrt{1 + v_3}} \frac{1}{2}
        \left[ \begin{matrix} v_3 + 1 \\ v_1 + v_2i \end{matrix} \right] \\
    \notag \\
    &= \frac{(v_3 + 1)\ket{0} + (v_1 + v_2i)\ket{1}}{\sqrt{2 + 2v_3}}
\end{align}