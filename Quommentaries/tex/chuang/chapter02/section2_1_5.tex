\subsection{Section 2.1.5}
\subsubsection{Exercise 2.11}
The eigenvectors, eigenvalues and diagonal representations of $Y$ will be calculated.
The process for $X$ and $Z$ is similar.
All answers are summarised in the end of this section.

Recall that
\begin{align}
    Y = \left[ \begin{matrix}
            0 & -i \\
            i & 0
        \end{matrix} \right].
\end{align}
Therefore,
\begin{align}
    det(Y - \lambda I) &= \left| \begin{matrix}
            - \lambda & -i \\
            i & - \lambda
        \end{matrix} \right| \\
    &= (-\lambda)^2 - (- i^2) \\
    &= \lambda^2 -1 = 0 .
\end{align}
Thus, $\lambda = \pm 1$.

Calculate the corresponding eigenvectors, using row reduction.
For $\lambda_0 = 1$,
\begin{align}
    \left[ \begin{matrix}
        -1 & -i \\ i & -1
    \end{matrix} \right]
    \begin{matrix}
        ~ \\
        +i \cdot row_1
    \end{matrix}
    %
    &\sim
    \left[ \begin{matrix}
        -1 & -i \\ 0 & 0
    \end{matrix} \right].
\end{align}
Therefore, since $-1a -ib = 0$,
the eigenvectors of $\lambda_0$ are the span of
\begin{align}
    \ket{\lambda_0} = \left\{ \left[ \begin{matrix} 1 \\ i \end{matrix} \right] \right\}.
\end{align}
That is, if $z \in \mathbb{C}$, then $-1 \cdot 1 \cdot z - i \cdot i \cdot z = 0$.
Finally, normalise $\ket{\lambda_0}$.
\begin{align}
    \frac{\ket{\lambda_0}}{\sqrt{\braket{\lambda_0}{\lambda_0}}} &=
    \frac{\ket{\lambda_0}}{\sqrt{1\cdot 1 + (-i) \cdot -i}} \\
    &= \frac{1}{\sqrt2} \ket{\lambda_0}.
\end{align}
Henceforth, let $\ket{\lambda_0}$ denote the normalised eigenvector.

Calculate the eigenvector span for $\lambda_1 = -1$.
\begin{align}
    \left[ \begin{matrix}
        1 & -i \\ i & 1
    \end{matrix} \right]
    \begin{matrix}
        ~ \\
        -i \cdot row_1
    \end{matrix}
    %
    &\sim
    \left[ \begin{matrix}
        1 & -i \\ 0 & 0
    \end{matrix} \right].
\end{align}
Therefore, since $1a -ib = 0$,
the eigenvectors of $\lambda_1$ are the span of
\begin{align}
    \ket{\lambda_1} = \left\{ \left[ \begin{matrix} 1 \\ -i \end{matrix} \right] \right\}.
\end{align}
That is, if $z \in \mathbb{C}$, then $1 \cdot 1 \cdot z - i \cdot (-i) \cdot z = 0$.
Finally, normalise $\ket{\lambda_1}$.
\begin{align}
    \frac{\ket{\lambda_1}}{\sqrt{\braket{\lambda_1}{\lambda_1}}} &=
    \frac{\ket{\lambda_1}}{\sqrt{1\cdot 1 + i \cdot (-i)}} \\
    &= \frac{1}{\sqrt2} \ket{\lambda_1}.
\end{align}
Henceforth, let $\ket{\lambda_1}$ denote the normalised eigenvector.

Thus, the diagonal representation is
$Y = \sum_i \lambda_i \ketbra{\lambda_i}{\lambda_i} = 
1 \ketbra{\lambda_0}{\lambda_0} - 1 \ketbra{\lambda_1}{\lambda_1}$.

Repeat the procedure for the remaining Pauli matrices.
\hyperref[tab:nielsen-and-chuang-answers-exercise-2-11]{
    Table \ref{tab:nielsen-and-chuang-answers-exercise-2-11}}
summarises the exercise answers.

\begin{table}[htb]
\def\arraystretch{2}%  1 is the default, change whatever you need
\centering
\begin{tabular}{|P{10em}|P{2em}|P{2em}|P{4em}|P{4em}|P{10em}|}
\hline
Pauli Matrix & $\lambda_0$ & $\lambda_1$  & $\ket{\lambda_0}$ & $\ket{\lambda_1}$ & Diagonal Representation
\\ \hline
$X$ & $1$ & $-1$ & $\ket+$ & $\ket-$ & $\ketbra++ - \ketbra--$
\\ \hline
$Y$ & $1$ & $-1$ & $\frac{1}{\sqrt2} \left[ \begin{matrix} 1 \\ i \end{matrix} \right]$ &
    $\frac{1}{\sqrt2} \left[ \begin{matrix} 1 \\ -i \end{matrix} \right]$ &
    $\ketbra{\lambda_0}{\lambda_0} - \ketbra{\lambda_1}{\lambda_1}$
\\ \hline
$Z$ & $1$ & $-1$ & $\ket0$ & $\ket1$ & $\ketbra00 - \ketbra11$
\\ \hline
\end{tabular}
\caption{Answers of Exercise 2.11}
\label{tab:nielsen-and-chuang-answers-exercise-2-11}
\end{table}