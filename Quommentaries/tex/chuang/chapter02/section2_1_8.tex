\subsection{Section 2.1.8}
\subsubsection{Exercise 2.35}

In order to solve this exercise,
it is necessary to find a spectral decomposition for $\vec{v} \vec{\sigma}$.
Then, it is possible to apply the definition of Operator functions.

With the aid of \hyperref[sec:nielsen-and-chuang-exercise-2-60]{Exercise 2.60},
the required spectral decomposition is obtained:
\begin{align}
    \vec{v}\vec{\sigma} &= +1 P_+ -1 P_- \\
    &= \frac{I + \vec{v}\vec{\sigma}}{2} - \frac{I - \vec{v}\vec{\sigma}}{2}
\end{align}

Now, calculating the value of $exp(i \theta \vec{v}\cdot\vec{\sigma})$
and applying the definition of Operator functions:
\begin{align}
    exp(i \theta \vec{v}\cdot\vec{\sigma}) &=
        exp(i \theta P_+) + exp(-i \theta P_-) \\
    &= exp(i \theta) P_+ + exp(-i \theta) P_- \\
    &= e^{i \theta} P_+ + e^{-i \theta} P_-
\end{align}

Then, applying Euler's Formula:
\begin{align}
    exp(i \theta \vec{v}\cdot\vec{\sigma}) &=
        cos(\theta)P_+ + i sin(\theta)P_+ +
        cos(\theta)P_- - i sin(\theta)P_- \\
    &= cos(\theta)\frac{I + \vec{v}\vec{\sigma}}{2} +
        i sin(\theta)\frac{I + \vec{v}\vec{\sigma}}{2} +
        cos(\theta)\frac{I - \vec{v}\vec{\sigma}}{2} -
        i sin(\theta)\frac{I - \vec{v}\vec{\sigma}}{2} \\
    &= cos(\theta) \left(
            \frac{I + \vec{v}\vec{\sigma} + I - \vec{v}\vec{\sigma}}{2}
        \right)
        + i sin(\theta) \left(
            \frac{I + \vec{v}\vec{\sigma} - I + \vec{v}\vec{\sigma}}{2}
        \right) \\
    &= cos(\theta) I + i\ sin(\theta) \vec{v}\vec{\sigma}
\end{align}

Thus obtaining the required result.