\subsection{Section 2.1.4}
\subsubsection{Outer Product Representation of A}
    \label{sec:nielsen-and-chuang-outer-product-of-a}
    It is stated from Equation 2.25 that it is possible to
    "[...] see from this equation that A has matrix element
    $\bra{w_j}A\ket{v_i}$".
    To see this, it is possible to compare the matrix and
    Dirac representations. Consider two systems $V$ and $W$
    with dimensions $n$ and $m$, respectively.
    In addition, suppose $\ket{v} \in V$ and $\ket{w} \in W$. 
    
    Using matrix representation:
    \begin{align}
        \braket{v}{w} &= \left[ \begin{matrix}
            v_1 \\ \vdots \\ v_i \\ \vdots \\ v_n
            \end{matrix} \right]
            \left[ \begin{matrix}
            w_1 & \cdots & w_j & \cdots & w_m
            \end{matrix} \right] \\
        &= \left[ \begin{matrix}
            v_1 w_1 & \cdots & v_1 w_j & \cdots & v_1 w_m \\
            \vdots & \ddots & \vdots & \ddots & \vdots \\
            v_i w_1 & \cdots & v_i w_j & \cdots & v_i w_m \\
            \vdots & \ddots & \vdots & \ddots & \vdots \\
            v_n w_1 & \cdots & v_n w_j & \cdots & v_n w_m
            \end{matrix} \right]
    \end{align}
    
    Using Dirac notation and the last part of Equation 2.21:
    \begin{align}
        \ketbra{v}{w} &= \sum_{ij} v_i\ket{i} w_j\bra{j} \\
            &= \sum_{ij} v_i w_j \ketbra{i}{j}
    \end{align}
    
    Then, comparing matrix and Dirac representation,
    it is easily verified that the matrix has elements
    $m_{ij} = v_i w_j$ for the $i$-th row and $j$-th column
    ( $\braket{i}{j}$ )
    with respect to the orthonormal basis $\ket{i}$ and $\ket{j}$
    for systems $V$ and $W$, respectively.