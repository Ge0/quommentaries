\subsection{Chapter 2 Problems}
\subsubsection{Problem 2.1}
This problem can be solved easily by combining the
logic of Exercises
\hyperref[sec:nielsen-and-chuang-exercise-2-35]{2.35} and
\hyperref[sec:nielsen-and-chuang-exercise-2-60]{2.60}.

It is known from exercise \hyperref[sec:nielsen-and-chuang-exercise-2-60]{2.60}
that $\vec{n} \vec{\sigma}$ has spectral decomposition
$+1P_+ - 1P_- = \\ +1 \left( \frac{I + \vec{n}\vec{\sigma}}{2} \right)
-1 \left( \frac{I - \vec{v}\vec{\sigma}}{2} \right)$.
Therefore, $\theta \vec{n} \vec{\sigma} =
\theta \left( \frac{I + \vec{n}\vec{\sigma}}{2} \right)
- \theta \left( \frac{I - \vec{v}\vec{\sigma}}{2} \right)$.
Then, by applying the definition of function operators and
the distributive property:

\begin{align}
    f(\theta \vec{n} \vec{\sigma}) &= f(\theta) \left(
        \frac{I + \vec{n}\vec{\sigma}}{2} \right) +
        f(-\theta) \left( \frac{I - \vec{v}\vec{\sigma}}{2} \right) \\
    &= \frac{f(\theta)}{2}I + \frac{f(-\theta)}{2}I +
        \frac{f(\theta)}{2}\vec{n}\vec{\sigma} -
        \frac{f(-\theta)}{2}\vec{n}\vec{\sigma} \\
    &= \frac{f(\theta) + f(-\theta)}{2}I +
        \frac{f(\theta) - f(-\theta)}{2}\vec{n}\vec{\sigma}
\end{align}

\subsubsection{Problem 2.2}

\subsubsection{Problem 2.3}