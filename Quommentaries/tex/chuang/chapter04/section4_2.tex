\subsection{Section 4.2}
\subsubsection{Bloch Vector}

At this point, it is strongly recommended that the reader understands the contents of
\hyperref[nielsen-and-chuang-qubit-representation-in-a-bloch-sphere]{
    Section \ref{nielsen-and-chuang-qubit-representation-in-a-bloch-sphere}
} and the sections mentioned therein.
Also, in order to properly understand the Bloch Vector,
refer back to Figure 1.3 in the book.

The given vector can be easily understood by interpreting a Bloch Sphere
as two unit circles with an overlapping axis.
For instance, the first circle is determined by the $x$ and $y$ axes
(the ``equatorial line circle''),
while the second circle is determined by $z$ and any
appropriate axis in the ``equatorial line''
(this axis depends on the point's position).

The $z$ position is the easiest one to understand.
It is simply the projection of the point ($\ket \psi$) along the $z$ axis ($\braket z \psi$).
Since it does not change depending on the azimuthal angle $\varphi$.

The positions described by both $x$ and $y$ axes depend on the projection
of the point $\ket \psi$ on the ``equatorial circle'' (name it $\ket{\psi_{xy}}$).
Consider the $x$ axis, if $z = 0$, then the projection of $\ket{\psi_{xy}}$
along $x$ would equal $cos(\varphi)$.
Note that $x \neq cos(\varphi)$ otherwise.
This happens because $\ket{\psi_{xy}}$ depends on $sin(\theta)$.
Thus, $\braket x \psi = cos(\varphi) sin(\theta)$.
Analogously, $\braket y \psi = sin(\varphi) sin(\theta)$.

In conclusion, the Bloch Vector is described by
$(x, y, z) = (cos\varphi sin\theta, sin\varphi sin\theta, cos\theta)$.