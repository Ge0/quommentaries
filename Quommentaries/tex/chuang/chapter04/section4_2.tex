\subsection{Section 4.2}
\subsubsection{Bloch Vector}

At this point, it is strongly recommended that the reader understands the contents of
\hyperref[nielsen-and-chuang-qubit-representation-in-a-bloch-sphere]{
    Section \ref{nielsen-and-chuang-qubit-representation-in-a-bloch-sphere}
} and the sections mentioned therein.
Also, in order to properly understand the Bloch Vector,
refer back to Figure 1.3 in the book.

The given vector can be easily understood by interpreting a Bloch Sphere
as two unit circles with an overlapping axis.
For instance, the first circle is determined by the $x$ and $y$ axes
(the ``equatorial line circle''),
while the second circle is determined by $z$ and any
appropriate axis in the ``equatorial line''
(this axis depends on the point's position).

The $z$ position is the easiest one to understand.
It is simply the projection of the point ($\ket \psi$) along the $z$ axis ($\braket z \psi$).
Since it does not change depending on the azimuthal angle $\varphi$.

The positions described by both $x$ and $y$ axes depend on the projection
of the point $\ket \psi$ on the ``equatorial circle'' (name it $\ket{\psi_{xy}}$).
Consider the $x$ axis, if $z = 0$, then the projection of $\ket{\psi_{xy}}$
along $x$ would equal $cos(\varphi)$.
Note that $x \neq cos(\varphi)$ otherwise.
This happens because $\ket{\psi_{xy}}$ depends on $sin(\theta)$.
Thus, $\braket x \psi = cos(\varphi) sin(\theta)$.
Analogously, $\braket y \psi = sin(\varphi) sin(\theta)$.

In conclusion, the Bloch Vector is described by
$(x, y, z) = (cos\varphi sin\theta, sin\varphi sin\theta, cos\theta)$.

%%%%%%%%%%%%%%%%%%%%%%%%%%%%%%%%%%%%%%%%%%%%%%%%%%%%%%%%%%%%%%%%%%%%%%%%%%%%%%%%%%%%%%%%%%%%%%%%%%%%%%%%

\subsubsection{Exercise 4.1}
It is recommended to answer \hyperref[sec:nielsen-and-chuang-exercise-2-11]{Exercise 2.11} beforehand.
Additionally, the reader should attempt to understand the
\hyperref[nielsen-and-chuang-qubit-representation-in-a-bloch-sphere]{Bloch Sphere Equation}.
Throughout this exercise, the reader may constantly refer back to the Bloch Sphere Equation,
\begin{align}
    \ket \psi = \cos \frac \theta 2 \ket0 + e^{i \varphi} sin \frac \theta 2 \ket1 .
\end{align}

\begin{itemize}
    \item The eigenvectors of $X$ are $\ket+$ and $\ket -$;
    \begin{itemize}
        \item $\ket +$;
        \begin{itemize}
            \item $\ket+ = \frac{\ket0 + \ket1}{\sqrt2}$. Hence,
                $cos \frac \theta 2 \ket0 = \frac{1}{\sqrt2} \ket0$. Thus,
                $\theta = \frac \pi 2$;
            \item $\ket+ = \frac{\ket0 + \ket1}{\sqrt2}$. Since
                $\theta = \frac \pi 2$,
                $e^{i \varphi} sin \frac \pi 4 \ket1 = \frac{1}{\sqrt2} \ket1$. Thus,
                $e^{i \varphi} = 1$, and $\varphi = 0$;
            \item Substituting the values of $\theta$ and $\varphi$ in the Bloch Vector formula,
                $\ket+ = (cos\varphi sin\theta, sin\varphi sin\theta, cos\theta) = (1, 0, 0)$;
        \end{itemize}
        
        \item $\ket -$;
        \begin{itemize}
            \item $\ket- = \frac{\ket0 - \ket1}{\sqrt2}$. Hence,
                $cos \frac \theta 2 \ket0 = \frac{1}{\sqrt2} \ket0$. Thus,
                $\theta = \frac \pi 2$;
            \item $\ket- = \frac{\ket0 - \ket1}{\sqrt2}$. Since
                $\theta = \frac \pi 2$,
                $e^{i \varphi} sin \frac \pi 4 \ket1 = - \frac{1}{\sqrt2} \ket1$. Thus,
                $e^{i \varphi} = -1$, and $\varphi = \pi$;
            \item Substituting the values of $\theta$ and $\varphi$ in the Bloch Vector formula,
                $\ket- = (cos\varphi sin\theta, sin\varphi sin\theta, cos\theta) = (-1, 0, 0)$;
        \end{itemize}
    \end{itemize}
    
    \item The eigenvectors of $Y$ are $\frac{\ket0 + i \ket1}{\sqrt2}$ and $\frac{\ket0 - i \ket1}{\sqrt2}$;
    \begin{itemize}
        \item $\frac{\ket0 + i \ket1}{\sqrt2}$;
        \begin{itemize}
            \item $cos \frac \theta 2 \ket0 = \frac{1}{\sqrt2} \ket0$. Thus,
                $\theta = \frac \pi 2$;
            \item Since $\theta = \frac \pi 2$,
                $e^{i \varphi} sin \frac \pi 4 \ket1 = \frac{i}{\sqrt2} \ket1$. Thus,
                $e^{i \varphi} = i$, and $\varphi = \frac \pi 2$;
            \item Substituting the values of $\theta$ and $\varphi$ in the Bloch Vector formula,
                $\ket+ = (cos\varphi sin\theta, sin\varphi sin\theta, cos\theta) = (0, 1, 0)$;
        \end{itemize}
        
        \item $\frac{\ket0 - i \ket1}{\sqrt2}$;
        \begin{itemize}
            \item $cos \frac \theta 2 \ket0 = \frac{1}{\sqrt2} \ket0$. Thus,
                $\theta = \frac \pi 2$;
            \item Since $\theta = \frac \pi 2$,
                $e^{i \varphi} sin \frac \pi 4 \ket1 = - \frac{i}{\sqrt2} \ket1$. Thus,
                $e^{i \varphi} = - i$, and $\varphi = \frac{3\pi}{2}$;
            \item Substituting the values of $\theta$ and $\varphi$ in the Bloch Vector formula,
                $\ket+ = (cos\varphi sin\theta, sin\varphi sin\theta, cos\theta) = (0, -1, 0)$;
        \end{itemize}
    \end{itemize}
    
    \item The eigenvectors of $Z$ are $\ket0$ and $\ket1$;
    \begin{itemize}
        \item $\ket0$;
        \begin{itemize}
            \item $cos \frac \theta 2 \ket0 = 1 \ket0$. Thus, $\theta = 0$;
            \item Since $\theta = 0$, $sin(0) = 0$. And
                $\varphi$ can assume any value in the $[0, 2\pi)$ range;
            \item Substituting the values of $\theta$ and $\varphi$ in the Bloch Vector formula,
                $\ket0 = (cos\varphi sin\theta, sin\varphi sin\theta, cos\theta) = (0, 0, 1)$
        \end{itemize}
        
        \item $\ket1$;
        \begin{itemize}
            \item $cos \frac \theta 2 \ket0 = 0 \ket0$. Thus, $\theta = \pi$;
            \item Thus, $\ket1 = e^{i \varphi} sin \frac \pi 2 \ket 1$.
                It is known that \hyperref[sec:noson-equation-4-5]{
                    multiplying a state by any complex number does not change it}.
                Hence, $e^{i \varphi} sin \frac \pi 2 \ket1 =
                    e^{-i \varphi} e^{i \varphi} sin \frac \pi 2 \ket1 =
                    sin \frac \pi 2 \ket1$.
                In conclusion, the value of $\varphi$ is negligible,
                and can assume any value in the $[0, 2\pi)$ range;
            \item Substituting the values of $\theta$ and $\varphi$ in the Bloch Vector formula,
                $\ket1 = (cos\varphi sin\theta, sin\varphi sin\theta, cos\theta) = (0, 0, -1)$.
        \end{itemize}
    \end{itemize}
\end{itemize}

%%%%%%%%%%%%%%%%%%%%%%%%%%%%%%%%%%%%%%%%%%%%%%%%%%%%%%%%%%%%%%%%%%%%%%%%%%%%%%%%%%%%%%%%%%%%%%%%%%%%%%%%

\subsubsection{Exercise 4.2}
In Nielsen and Chuang's Section 2.1.8, operator functions are discussed.
Thus, $A$ has spectral decomposition, and
\begin{align}
    A^2 &= \sum_{ab} a \ketbra{a}{a} b \ketbra{b}{b} \\
    &= \sum_{ab} ab \ket{a} \braket{a}{b} \bra{b} \\
    &= \sum_{ab} ab \braket{a}{b} \ket{a} \bra{b}.
\end{align}
Since $\set{\ket a \mid \forall a}$ form an orthonormal basis defined by the eigenspace
- and $\set{\ket b \mid \forall b}$ describes the same basis -
$\braket{a}{b} = \delta_{ab}$
\footnote{$\delta_{ij}$ is defined in the paragraph that follows
    Nielsen and Chuang's Equation (2.16)
},
\begin{align}
    A^2 &= \sum_{ab} ab \delta_{ab} \ket{a} \bra{b} \\
    &= \sum_a a^2 \ketbra a a \\
    &= I.
\end{align}
Due to the completeness relation $\sum_a \ketbra a a = I$,
it is possible to conclude that $a = \pm 1$.

Compute the value of $exp(iAx)$ using the definition of operator functions
Nielsen and Chuang's Section 2.1.8, and Euler's Formula.
\begin{align}
    exp(iAx) &= \sum_a exp(iax) \ketbra a a \\
    &= \sum_a cos(ax) \ketbra a a + i\ sin(ax) \ketbra a a.
\end{align}
Recall that $a = \pm 1$.
From trigonometry, it is known $cos(x) = cos(-x)$.
Also, $sin(-x) = - sin(x)$.
Thus,
\begin{align}
    exp(iAx) &= \sum_a cos(x) \ketbra a a + i\ sin(x) a \ketbra a a \\
    &= cos(x) I + i\ sin(x) A.
\end{align}

Using this to verify Equations (4.4) to (4.6) is straightforward,
since $X^2 = Y^2 = Z^2 = I$.
For $R_x(\theta)$,
\begin{align}
    e^{i \theta X / 2} &= cos( - \frac \theta 2) I + i\ sin( - \frac \theta 2) X \\
    &= cos \frac \theta 2 I - i\ sin \frac \theta 2 X \\
    &= \left[ \begin{matrix} cos \frac \theta 2 & 0 \\ 0 & cos \frac \theta 2 \end{matrix} \right] - i
        \left[ \begin{matrix} 0 & sin \frac \theta 2 \\ sin \frac \theta 2 & 0 \end{matrix} \right] \\
    &= \left[ \begin{matrix} cos \frac \theta 2 & - i\ sin \frac \theta 2 \\
        - i\ sin \frac \theta 2 & cos \frac \theta 2 \end{matrix} \right].
\end{align}
For $R_y(\theta)$,
\begin{align}
    e^{i \theta Y / 2} &= cos( - \frac \theta 2) I + i\ sin( - \frac \theta 2) Y \\
    &= cos \frac \theta 2 I - i\ sin \frac \theta 2 Y \\
    &= \left[ \begin{matrix} cos \frac \theta 2 & 0 \\ 0 & cos \frac \theta 2 \end{matrix} \right] - i
        \left[ \begin{matrix} 0 & -i\ sin \frac \theta 2 \\ i\ sin \frac \theta 2 & 0 \end{matrix} \right] \\
    &= \left[ \begin{matrix} cos \frac \theta 2 & - sin \frac \theta 2 \\
        sin \frac \theta 2 & cos \frac \theta 2 \end{matrix} \right].
\end{align}
For $R_z(\theta)$,
\begin{align}
    e^{i \theta Z / 2} &= cos( - \frac \theta 2) I + i\ sin( - \frac \theta 2) Z \\
    &= cos \frac \theta 2 I - i\ sin \frac \theta 2 Z \\
    &= \left[ \begin{matrix} cos \frac \theta 2 & 0 \\ 0 & cos \frac \theta 2 \end{matrix} \right] - i
        \left[ \begin{matrix} sin \frac \theta 2 & 0 \\ 0 & - sin \frac \theta 2\end{matrix} \right] \\
    &= \left[ \begin{matrix} cos \frac \theta 2 - i\ sin \frac \theta 2 & 0 \\
        0 & cos \frac \theta 2 + i sin \frac \theta 2 \end{matrix} \right] \\
    &= \left[ \begin{matrix} cos(- \frac \theta 2) + i\ sin(- \frac \theta 2) & 0 \\
        0 & cos \frac \theta 2 + i sin \frac \theta 2 \end{matrix} \right] \\
    &= \left[ \begin{matrix} e^{-i \theta/2} & 0 \\ 0 & e^{i \theta/2} \end{matrix} \right]
\end{align}