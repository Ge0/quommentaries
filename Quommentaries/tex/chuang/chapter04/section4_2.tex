\subsection{Section 4.2}
\subsubsection{Bloch Vector}

At this point, it is strongly recommended that the reader understands the contents of
\hyperref[nielsen-and-chuang-qubit-representation-in-a-bloch-sphere]{
    Section \ref{nielsen-and-chuang-qubit-representation-in-a-bloch-sphere}
} and the sections mentioned therein.
Also, in order to properly understand the Bloch Vector,
refer back to Figure 1.3 in the book.

The given vector can be easily understood by interpreting a Bloch Sphere
as two unit circles with an overlapping axis.
For instance, the first circle is determined by the $x$ and $y$ axes
(the ``equatorial line circle''),
while the second circle is determined by $z$ and any
appropriate axis in the ``equatorial line''
(this axis depends on the point's position).

The $z$ position is the easiest one to understand.
It is simply the projection of the point ($\ket \psi$) along the $z$ axis ($\braket z \psi$).
Since it does not change depending on the azimuthal angle $\varphi$.

The positions described by both $x$ and $y$ axes depend on the projection
of the point $\ket \psi$ on the ``equatorial circle'' (name it $\ket{\psi_{xy}}$).
Consider the $x$ axis, if $z = 0$, then the projection of $\ket{\psi_{xy}}$
along $x$ would equal $cos(\varphi)$.
Note that $x \neq cos(\varphi)$ otherwise.
This happens because $\ket{\psi_{xy}}$ depends on $sin(\theta)$.
Thus, $\braket x \psi = cos(\varphi) sin(\theta)$.
Analogously, $\braket y \psi = sin(\varphi) sin(\theta)$.

In conclusion, the Bloch Vector is described by
$(x, y, z) = (cos\varphi sin\theta, sin\varphi sin\theta, cos\theta)$.

%%%%%%%%%%%%%%%%%%%%%%%%%%%%%%%%%%%%%%%%%%%%%%%%%%%%%%%%%%%%%%%%%%%%%%%%%%%%%%%%%%%%%%%%%%%%%%%%%%%%%%%%

\subsubsection{Exercise 4.1}
It is recommended to answer \hyperref[sec:nielsen-and-chuang-exercise-2-11]{Exercise 2.11} beforehand.
Additionally, the reader should attempt to understand the
\hyperref[nielsen-and-chuang-qubit-representation-in-a-bloch-sphere]{Bloch Sphere Equation}.
Throughout this exercise, the reader may constantly refer back to the Bloch Sphere Equation,
\begin{align}
    \ket \psi = \cos \frac \theta 2 \ket0 + e^{i \varphi} sin \frac \theta 2 \ket1 .
\end{align}

\begin{itemize}
    \item The eigenvectors of $X$ are $\ket+$ and $\ket -$;
    \begin{itemize}
        \item $\ket +$;
        \begin{itemize}
            \item $\ket+ = \frac{\ket0 + \ket1}{\sqrt2}$. Hence,
                $cos \frac \theta 2 \ket0 = \frac{1}{\sqrt2} \ket0$. Thus,
                $\theta = \frac \pi 2$;
            \item $\ket+ = \frac{\ket0 + \ket1}{\sqrt2}$. Since
                $\theta = \frac \pi 2$,
                $e^{i \varphi} sin \frac \pi 4 \ket1 = \frac{1}{\sqrt2} \ket1$. Thus,
                $e^{i \varphi} = 1$, and $\varphi = 0$;
            \item Substituting the values of $\theta$ and $\varphi$ in the Bloch Vector formula,
                $\ket+ = (cos\varphi sin\theta, sin\varphi sin\theta, cos\theta) = (1, 0, 0)$;
        \end{itemize}
        
        \item $\ket -$;
        \begin{itemize}
            \item $\ket- = \frac{\ket0 - \ket1}{\sqrt2}$. Hence,
                $cos \frac \theta 2 \ket0 = \frac{1}{\sqrt2} \ket0$. Thus,
                $\theta = \frac \pi 2$;
            \item $\ket- = \frac{\ket0 - \ket1}{\sqrt2}$. Since
                $\theta = \frac \pi 2$,
                $e^{i \varphi} sin \frac \pi 4 \ket1 = - \frac{1}{\sqrt2} \ket1$. Thus,
                $e^{i \varphi} = -1$, and $\varphi = \pi$;
            \item Substituting the values of $\theta$ and $\varphi$ in the Bloch Vector formula,
                $\ket- = (cos\varphi sin\theta, sin\varphi sin\theta, cos\theta) = (-1, 0, 0)$;
        \end{itemize}
    \end{itemize}
    
    \item The eigenvectors of $Y$ are $\frac{\ket0 + i \ket1}{\sqrt2}$ and $\frac{\ket0 - i \ket1}{\sqrt2}$;
    \begin{itemize}
        \item $\frac{\ket0 + i \ket1}{\sqrt2}$;
        \begin{itemize}
            \item $cos \frac \theta 2 \ket0 = \frac{1}{\sqrt2} \ket0$. Thus,
                $\theta = \frac \pi 2$;
            \item Since $\theta = \frac \pi 2$,
                $e^{i \varphi} sin \frac \pi 4 \ket1 = \frac{i}{\sqrt2} \ket1$. Thus,
                $e^{i \varphi} = i$, and $\varphi = \frac \pi 2$;
            \item Substituting the values of $\theta$ and $\varphi$ in the Bloch Vector formula,
                $\ket+ = (cos\varphi sin\theta, sin\varphi sin\theta, cos\theta) = (0, 1, 0)$;
        \end{itemize}
        
        \item $\frac{\ket0 - i \ket1}{\sqrt2}$;
        \begin{itemize}
            \item $cos \frac \theta 2 \ket0 = \frac{1}{\sqrt2} \ket0$. Thus,
                $\theta = \frac \pi 2$;
            \item Since $\theta = \frac \pi 2$,
                $e^{i \varphi} sin \frac \pi 4 \ket1 = - \frac{i}{\sqrt2} \ket1$. Thus,
                $e^{i \varphi} = - i$, and $\varphi = \frac{3\pi}{2}$;
            \item Substituting the values of $\theta$ and $\varphi$ in the Bloch Vector formula,
                $\ket+ = (cos\varphi sin\theta, sin\varphi sin\theta, cos\theta) = (0, -1, 0)$;
        \end{itemize}
    \end{itemize}
    
    \item The eigenvectors of $Z$ are $\ket0$ and $\ket1$;
    \begin{itemize}
        \item $\ket0$;
        \begin{itemize}
            \item $cos \frac \theta 2 \ket0 = 1 \ket0$. Thus, $\theta = 0$;
            \item Since $\theta = 0$, $sin(0) = 0$. And
                $\varphi$ can assume any value in the $[0, 2\pi)$ range;
            \item Substituting the values of $\theta$ and $\varphi$ in the Bloch Vector formula,
                $\ket0 = (cos\varphi sin\theta, sin\varphi sin\theta, cos\theta) = (0, 0, 1)$
        \end{itemize}
        
        \item $\ket1$;
        \begin{itemize}
            \item $cos \frac \theta 2 \ket0 = 0 \ket0$. Thus, $\theta = \pi$;
            \item Thus, $\ket1 = e^{i \varphi} sin \frac \pi 2 \ket 1$.
                It is known that \hyperref[sec:noson-equation-4-5]{
                    multiplying a state by any complex number does not change it}.
                Hence, $e^{i \varphi} sin \frac \pi 2 \ket1 =
                    e^{-i \varphi} e^{i \varphi} sin \frac \pi 2 \ket1 =
                    sin \frac \pi 2 \ket1$.
                In conclusion, the value of $\varphi$ is negligible,
                and can assume any value in the $[0, 2\pi)$ range;
            \item Substituting the values of $\theta$ and $\varphi$ in the Bloch Vector formula,
                $\ket1 = (cos\varphi sin\theta, sin\varphi sin\theta, cos\theta) = (0, 0, -1)$.
        \end{itemize}
    \end{itemize}
\end{itemize}

%%%%%%%%%%%%%%%%%%%%%%%%%%%%%%%%%%%%%%%%%%%%%%%%%%%%%%%%%%%%%%%%%%%%%%%%%%%%%%%%%%%%%%%%%%%%%%%%%%%%%%%%

\subsubsection{Exercise 4.2}
\label{sec:nielsen-and-chuang-exercise-4-2}
In Nielsen and Chuang's Section 2.1.8, operator functions are discussed.
Thus, $A$ has spectral decomposition, and
\begin{align}
    A^2 &= \sum_{ab} a \ketbra{a}{a} b \ketbra{b}{b} \\
    &= \sum_{ab} ab \ket{a} \braket{a}{b} \bra{b} \\
    &= \sum_{ab} ab \braket{a}{b} \ket{a} \bra{b}.
\end{align}
Since $\set{\ket a \mid \forall a}$ form an orthonormal basis defined by the eigenspace
- and $\set{\ket b \mid \forall b}$ describes the same basis -
$\braket{a}{b} = \delta_{ab}$
\footnote{$\delta_{ij}$ is defined in the paragraph that follows
    Nielsen and Chuang's Equation (2.16)
},
\begin{align}
    A^2 &= \sum_{ab} ab \delta_{ab} \ket{a} \bra{b} \\
    &= \sum_a a^2 \ketbra a a \\
    &= I.
\end{align}
Due to the completeness relation $\sum_a \ketbra a a = I$,
it is possible to conclude that $a = \pm 1$.

Compute the value of $exp(iAx)$ using the definition of operator functions
Nielsen and Chuang's Section 2.1.8, and Euler's Formula.
\begin{align}
    exp(iAx) &= \sum_a exp(iax) \ketbra a a \\
    &= \sum_a cos(ax) \ketbra a a + i\ sin(ax) \ketbra a a.
\end{align}
Recall that $a = \pm 1$.
From trigonometry, it is known $cos(x) = cos(-x)$.
Also, $sin(-x) = - sin(x)$.
Thus,
\begin{align}
    exp(iAx) &= \sum_a cos(x) \ketbra a a + i\ sin(x) a \ketbra a a \\
    &= cos(x) I + i\ sin(x) A.
\end{align}

Using this to verify Equations (4.4) to (4.6) is straightforward,
since $X^2 = Y^2 = Z^2 = I$.
For $R_x(\theta)$,
\begin{align}
    e^{-i \theta X / 2} &= cos( - \frac \theta 2) I + i\ sin( - \frac \theta 2) X \\
    &= cos \frac \theta 2 I - i\ sin \frac \theta 2 X \\
    &= \left[ \begin{matrix} cos \frac \theta 2 & 0 \\ 0 & cos \frac \theta 2 \end{matrix} \right] - i
        \left[ \begin{matrix} 0 & sin \frac \theta 2 \\ sin \frac \theta 2 & 0 \end{matrix} \right] \\
    &= \left[ \begin{matrix} cos \frac \theta 2 & - i\ sin \frac \theta 2 \\
        - i\ sin \frac \theta 2 & cos \frac \theta 2 \end{matrix} \right].
\end{align}
For $R_y(\theta)$,
\begin{align}
    e^{-i \theta Y / 2} &= cos( - \frac \theta 2) I + i\ sin( - \frac \theta 2) Y \\
    &= cos \frac \theta 2 I - i\ sin \frac \theta 2 Y \\
    &= \left[ \begin{matrix} cos \frac \theta 2 & 0 \\ 0 & cos \frac \theta 2 \end{matrix} \right] - i
        \left[ \begin{matrix} 0 & -i\ sin \frac \theta 2 \\ i\ sin \frac \theta 2 & 0 \end{matrix} \right] \\
    &= \left[ \begin{matrix} cos \frac \theta 2 & - sin \frac \theta 2 \\
        sin \frac \theta 2 & cos \frac \theta 2 \end{matrix} \right].
\end{align}
For $R_z(\theta)$,
\begin{align}
    e^{-i \theta Z / 2} &= cos( - \frac \theta 2) I + i\ sin( - \frac \theta 2) Z \\
    &= cos \frac \theta 2 I - i\ sin \frac \theta 2 Z \\
    &= \left[ \begin{matrix} cos \frac \theta 2 & 0 \\ 0 & cos \frac \theta 2 \end{matrix} \right] - i
        \left[ \begin{matrix} sin \frac \theta 2 & 0 \\ 0 & - sin \frac \theta 2\end{matrix} \right] \\
    &= \left[ \begin{matrix} cos \frac \theta 2 - i\ sin \frac \theta 2 & 0 \\
        0 & cos \frac \theta 2 + i sin \frac \theta 2 \end{matrix} \right] \\
    &= \left[ \begin{matrix} cos(- \frac \theta 2) + i\ sin(- \frac \theta 2) & 0 \\
        0 & cos \frac \theta 2 + i sin \frac \theta 2 \end{matrix} \right] \\
    &= \left[ \begin{matrix} e^{-i \theta/2} & 0 \\ 0 & e^{i \theta/2} \end{matrix} \right]
\end{align}

%%%%%%%%%%%%%%%%%%%%%%%%%%%%%%%%%%%%%%%%%%%%%%%%%%%%%%%%%%%%%%%%%%%%%%%%%%%%%%%%%%%%%%%%%%%%%%%%%%%%%%%%

\subsubsection{Exercise 4.3}
Compute $\pi/4$,
\begin{align}
    R_z(\pi/4) &= cos \frac{\pi}{8} I - sin \frac{\pi}{8} Z \\
    &= \left[ \begin{matrix} e^{-i \pi/8} & 0 \\ 0 & e^{i \pi/8} \end{matrix} \right].
\end{align}
Note that this almost matches $T$ - refer back to Equation (4.3).
Thus,
\begin{align}
    T = e^{i \pi/8} R_z(\pi/4).
\end{align}

%%%%%%%%%%%%%%%%%%%%%%%%%%%%%%%%%%%%%%%%%%%%%%%%%%%%%%%%%%%%%%%%%%%%%%%%%%%%%%%%%%%%%%%%%%%%%%%%%%%%%%%%

\subsubsection{Exercise 4.4}

It is desired to write $H$ in terms of a product of $R_x$, $R_z$, and $e^{i\varphi}$.
It may be tempting to assign in the range $[0, \pi]$,
but the requested solution would not be obtained.

First, find the corresponding rotation operator for $H$.
This is possible since $H^2 = I$
(refer back to \hyperref[sec:nielsen-and-chuang-exercise-4-2]{Exercise 4.2}).
Thus,
\begin{align}
    R_h(\theta) &= e^{-i \theta H / 2} \\
    &= cos \frac\theta2 I - i\ sin \frac\theta2 H.
\end{align}
Note that
\begin{align}
    R_h(\pi) &= -i H.
\end{align}
And, multiplying both sides by $e^{i \frac \pi 2}$,
\begin{align}
    \label{eq:nielsen-and-chuang-exercise-4-4-e-rh}
    e^{i \frac \pi 2} R_h(\pi) &= e^{i \frac \pi 2} \cdot (-i H) \\
    &= i (-i H) \\
    \label{eq:nielsen-and-chuang-exercise-4-4-h}
    &= H.
\end{align}

Now, express $R_h$ in terms of $R_x$, $R_z$, and $e^{i\psi}$,
according to the equation $R_h(\theta) = R_x(\theta) R_z(\theta) e^{i\psi}$.
Recall that $H = \frac{X + Z}{\sqrt 2}$.
Also, suppose $\psi = \theta \phi / 2$ for some $\phi$.
Then,
\begin{align}
    R_h(\theta) &= R_x(\theta) R_z(\theta) e^{i\psi} \\
    e^{-i \theta (\frac{X + Z}{\sqrt 2}) / 2} &=
        e^{-i \theta X / 2}\ e^{-i \theta (X + Z) / 2}\ e^{i \theta \phi / 2} \\
    &= e^{-i \theta (X + Z - \phi) / 2}.
\end{align}
Thus,
\begin{align}
    \frac{X + Z}{\sqrt 2} = X + Z - \phi \\
    \phi = X + Z - \frac{X + Z}{\sqrt 2} \\
    \phi = \frac{\sqrt2 - 1}{\sqrt 2} (X + Z).
\end{align}

Henceforth, $\psi = \theta \frac{\sqrt2 - 1}{\sqrt 2} (X + Z) / 2$, and
\begin{align}
    R_h(\theta) = R_x(\theta) R_z(\theta) e^{i \theta \frac{\sqrt2 - 1}{\sqrt 2} (X + Z) / 2}.
\end{align}

Merging this result with Equations
\ref{eq:nielsen-and-chuang-exercise-4-4-e-rh} to \ref{eq:nielsen-and-chuang-exercise-4-4-h},
\begin{align}
    H &= R_h(\pi) e^{i \frac \pi 2} \\
    &= R_x(\pi) R_z(\pi) e^{i \pi \frac{\sqrt2 - 1}{\sqrt 2} (X + Z) / 2} e^{i \frac \pi 2} \\
    &= R_x(\pi) R_z(\pi) e^{i \pi \frac{\sqrt2 - 1}{\sqrt 2} (X + Z) / 2\ +\ i \frac \pi 2}.
\end{align}
Thus, the requested value for $\varphi$ in $e^{i \varphi}$ is
\begin{align}
    \varphi = \pi \frac{\sqrt2 - 1}{\sqrt 2} (X + Z) / 2\ +\ \frac \pi 2.
\end{align}

%%%%%%%%%%%%%%%%%%%%%%%%%%%%%%%%%%%%%%%%%%%%%%%%%%%%%%%%%%%%%%%%%%%%%%%%%%%%%%%%%%%%%%%%%%%%%%%%%%%%%%%%
\subsubsection{Exercise 4.5}
A proof analogous to $(\hat{n} - \vec{\sigma})^2 = I$ can be found in the beginning of
\hyperref[sec:nielsen-and-chuang-problem-2-3]{Problem 2.3's solution}
(until
\hyperref[eq:nielsen-and-chuang-problem-2-3-q-qdagger-equals-i]{
    Equation \ref{eq:nielsen-and-chuang-problem-2-3-q-qdagger-equals-i}
}).
To prove $(\hat{n} - \vec{\sigma})^2 = I$, simply rename the variables.

Notwithstanding, proving Equation 4.8 is also straightforward.
Just substitute $(\hat{n} - \vec{\sigma})^2 = I$ for $A$ in Equation 4.7,
which was proved in \hyperref[sec:nielsen-and-chuang-exercise-4-2]{Exercise 4.2}.
Thus, using \hyperref[sec:nielsen-and-chuang-equation-2-116]{Equation 2.116},
\begin{align}
    R_{\hat{n}} &\equiv exp(-i \theta \hat{n} \cdot \vec{\sigma}/2) \\
    &= cos \frac \theta 2 I - i\ sin \frac \theta 2 (\hat{n} \cdot \vec{\sigma}) \\
    &= cos \frac \theta 2 I - i\ sin \frac \theta 2 (n_x X + n_y Y + n_z Z).
\end{align}
%%%%%%%%%%%%%%%%%%%%%%%%%%%%%%%%%%%%%%%%%%%%%%%%%%%%%%%%%%%%%%%%%%%%%%%%%%%%%%%%%%%%%%%%%%%%%%%%%%%%%%%%
\subsection{Exercise 4.6}
The original idea was to use $R_x(\theta)$, $R_y(\theta)$ and $R_z(\theta)$
to construct a basis to represent $R_{\hat n}(\theta)$.
Similar to how $\set{x, y, z}$ is a basis for $\R^3$.
However, I was neither able to handle the Math;
nor to come up with a formal proof which would satisfy me.
\emph{\textbf{If the reader manages to solve it this way, please contact me!}}
The solution hereby detailed is based on the idea proposed by
\href{https://scholar.google.com.br/citations?user=AzhOiFAAAAAJ&hl}
{Jalil Moqadam}. Thus, I thank him.

The idea is to rotate a state $\ket\psi$ --
with corresponding Bloch Vector $\vec{r} = [r_1, r_2, r_3]^\dagger$ --
by $\alpha$ degrees about the $\hat n$ vector --
$n = [n_1, n_2, n_3]^\dagger$.
Thus obtaining another stade $\ket \varphi$ with corresponding
Bloch Vector $\vec{r'} = [r_1', r_2', r_3']^\dagger$.
\href{https://www.youtube.com/watch?v=DtL_giO-EB8#t=4m3.5s}{In other words},
\begin{align}
    R_{\hat n}(\alpha) \ket \psi = \ket \varphi .
\end{align}
However, the information of the Bloch Vectors is implicit in a state.
On the other hand, if density operator notation is used,
the information about the Bloch Vector is explicitly stated
(refer to \todo{Exercise 2.72}),
\begin{align}
    \ketbra{\psi}{\psi} = \rho = \frac{I + \vec{r} \cdot \vec{\sigma}}{2}.
\end{align}
Thus, applying the rotation,
\begin{align}
    R_{\hat n}(\alpha) \rho R_{\hat n}(\alpha)^\dagger =
    R_{\hat n}(\alpha) \ketbra{\psi}{\psi} R_{\hat n}(\alpha)^\dagger =
    \ketbra{\varphi}{\varphi}  =
    \rho' =
    \frac{I + \vec{r'}\cdot\vec{\sigma}}{2} .
\end{align}
Hence, it is possible to compare the obtained $\vec{r'}$ with
the result of
\begin{align}
    R \vec r = \vec{r'},
    \label{eq:nielsen-and-chuang-exercise-4-6-assert}
\end{align}
where $R$ is the
\href{https://en.wikipedia.org/wiki/Rotation_matrix#Rotation_matrix_from_axis_and_angle}
{rotation matrix from axis and angle}.

By computing $R_{\hat n}(\alpha) \rho R_{\hat n}(\alpha)^\dagger$,
\begin{align}
    R_{\hat n}(\alpha) \rho R_{\hat n}(\alpha)^\dagger &= 
        \exp(-i \alpha \hat{n}\cdot\vec\sigma)
        \frac{I + \vec r \cdot \vec{\sigma}}{2}
        \exp(-i \alpha \hat{n}\cdot\vec\sigma)^\dagger \\
    &= \exp(-i \alpha \hat{n}\cdot\vec\sigma)
        \frac{I + \vec r \cdot \vec{\sigma}}{2}
        \exp(i \alpha \hat{n}\cdot\vec\sigma) \\
    &= \frac{1}{2} \left(
        e^{-i \alpha \hat{n}\cdot\vec\sigma}I
        e^{i \alpha \hat{n}\cdot\vec\sigma}
        +
        e^{-i \alpha \hat{n}\cdot\vec\sigma}
        \vec{r} \cdot \vec{\sigma}
        e^{i \alpha \hat{n}\cdot\vec\sigma}
    \right) \\
    &= \frac{1}{2} \left(
        \ I + \left(
            \cos \left( \frac{\alpha}{2} \right) I -
            i \sin \left( \frac{\alpha}{2}\right)
            \hat{n} \cdot \vec{\sigma}
        \right)
        \vec{r}\cdot\vec{\sigma}
        \left(
            \cos \left( \frac{\alpha}{2} \right) I +
            i \sin \left( \frac{\alpha}{2} \right)
            \hat{n} \cdot \vec{\sigma}
        \right)
    \ \right) \\
    &= \frac{1}{2} (\ 
        I + \cos^2 \frac{\alpha}{2} I \vec{r}\cdot\vec{\sigma}I +
        i \cos\frac{\alpha}{2}\sin\frac{\alpha}{2} I \vec{r} \cdot\vec{\sigma}
            \ \hat{n}\cdot\vec{\sigma} -
        i \sin\frac{\alpha}{2}\cos\frac{\alpha}{2} \hat{n}\cdot\vec{\sigma}
            \ \vec{r}\cdot\vec{\sigma} I -  \nonumber \\
        &\quad i^2 \sin^2\frac{\alpha}{2} \hat{n}\cdot\vec{\sigma}\ 
            \vec{r}\cdot\vec{\sigma}\ \hat{n}\cdot\vec{\sigma}
    \ )
    \\
    &= \frac{1}{2} (\ 
        I + \cos^2 \frac{\alpha}{2} \vec{r}\cdot\vec{\sigma} +
        i \cos\frac{\alpha}{2}\sin\frac{\alpha}{2} \vec{r} \cdot\vec{\sigma}
            \ \hat{n}\cdot\vec{\sigma} -
        i \sin\frac{\alpha}{2}\cos\frac{\alpha}{2} \hat{n}\cdot\vec{\sigma}
            \ \vec{r}\cdot\vec{\sigma} +  \nonumber \\
        &\quad \sin^2\frac{\alpha}{2} \hat{n}\cdot\vec{\sigma}\ 
            \vec{r}\cdot\vec{\sigma}\ \hat{n}\cdot\vec{\sigma}
    \ )
    \label{eq:nielsen-and-chuang-exercise-4-6-split}
\end{align}
is obtained.

It is desirable to isolate the common $\sigma_i$ factor,
i.e. to rewrite the previous equation such that
\begin{align}
    R_{\hat n}(\alpha) \rho R_{\hat n}(\alpha)^\dagger =
    \frac{I + \vec{r}\cdot\vec{\sigma}}{2} =
    \frac{I + \sum_{i = 1}^3 r_i \sigma_i}{2}.
\end{align}
Thus, for the sake of clarity, Equation \ref{eq:nielsen-and-chuang-exercise-4-6-split}
will be computed separately for each part as follows
\begin{align}
    R_{\hat n}(\alpha) \rho R_{\hat n}(\alpha)^\dagger = 
    \frac{I + a + b + c}{2}.
\end{align}
Where
\begin{align}
    a = \cos^2 \frac{\alpha}{2} \vec{r}\cdot\vec{\sigma},
    \label{eq:nielsen-and-chuang-exercise-4-6-a}
\end{align}
\begin{align}
    b = i \cos\frac{\alpha}{2}\sin\frac{\alpha}{2} \vec{r} \cdot\vec{\sigma}
        \ \hat{n}\cdot\vec{\sigma} -
        i \sin\frac{\alpha}{2}\cos\frac{\alpha}{2} \hat{n}\cdot\vec{\sigma}
        \ \vec{r}\cdot\vec{\sigma},
    \label{eq:nielsen-and-chuang-exercise-4-6-b}
\end{align}
and
\begin{align}
    c = \sin^2\frac{\alpha}{2} \hat{n}\cdot\vec{\sigma}\
        \vec{r}\cdot\vec{\sigma}\ \hat{n}\cdot\vec{\sigma} .
    \label{eq:nielsen-and-chuang-exercise-4-6-c}
\end{align}

\begin{enumerate}[label=\alph*)]
    \item Equation \ref{eq:nielsen-and-chuang-exercise-4-6-a} is already
    in the desired format, since
    \begin{align}
        a = \cos^2 \left(\frac{\alpha}{2}\right) \vec{r}\cdot\vec{\sigma} =
        \cos^2 \left(\frac{\alpha}{2}\right) \sum_{i = 1}^3 r_i \sigma_i .
    \end{align}
    \item Equation \ref{eq:nielsen-and-chuang-exercise-4-6-b} can be
    rewritten as follows
    \begin{align}
        b &= i \cos\frac{\alpha}{2}\sin\frac{\alpha}{2} \vec{r} \cdot\vec{\sigma}
            \ \hat{n}\cdot\vec{\sigma} -
            i \sin\frac{\alpha}{2}\cos\frac{\alpha}{2} \hat{n}\cdot\vec{\sigma}
            \ \vec{r}\cdot\vec{\sigma} \\
        &= i \cos\frac{\alpha}{2}\sin\frac{\alpha}{2}
            \sum_{i = 1}^3 \sum_{j = 1}^3 r_i \sigma_i\ n_j \sigma_j
            -
            i \sin\frac{\alpha}{2}\cos\frac{\alpha}{2}
            \sum_{j = 1}^3\sum_{i = 1}^3 n_j \sigma_j\ r_i \sigma_i \\
        &= i \cos\frac{\alpha}{2}\sin\frac{\alpha}{2}
            \sum_{i = 1}^3 \sum_{j = 1}^3 r_i n_j\ \sigma_i \sigma_j
            -
            i \sin\frac{\alpha}{2}\cos\frac{\alpha}{2}
            \sum_{j = 1}^3\sum_{i = 1}^3 n_j r_i\ \sigma_j \sigma_i.
    \end{align}
    Whenever $j = i$,
    \begin{align}
        i \cos\frac{\alpha}{2}\sin\frac{\alpha}{2}
        \sum_{i = 1}^3 r_i n_i\ \sigma_i \sigma_i
        -
        i \sin\frac{\alpha}{2}\cos\frac{\alpha}{2}
        \sum_{j = 1}^3 n_i r_i\ \sigma_i \sigma_i
        = 0.
    \end{align}
    Hence,
    \begin{align}
        b = i \cos\frac{\alpha}{2}\sin\frac{\alpha}{2}
            \sum_{i = 1}^3 \sum_{j \neq i} r_i n_j\ \sigma_i \sigma_j
            -
            i \sin\frac{\alpha}{2}\cos\frac{\alpha}{2}
            \sum_{j = 1}^3\sum_{j \neq i} n_j r_i\ \sigma_j \sigma_i .
    \end{align}
    Since $\sigma_j\sigma_i = - \sigma_i \sigma_j$ if $i \neq j$ -- i.e.,
    $YX = -XY$, $ZX = -XZ$, and
    $Z\href{https://www.youtube.com/watch?v=LdpMpfp-J_I}{Y = -YZ}$ --,
    \begin{align}
        b &= i \cos\frac{\alpha}{2}\sin\frac{\alpha}{2}
            \sum_{i = 1}^3 \sum_{j \neq i} r_i n_j\ \sigma_i \sigma_j
            +
            i \sin\frac{\alpha}{2}\cos\frac{\alpha}{2}
            \sum_{j = 1}^3\sum_{j \neq i} n_j r_i\ \sigma_i \sigma_j \\
        &= 2 i \cos\frac{\alpha}{2}\sin\frac{\alpha}{2}
            \sum_{i = 1}^3 \sum_{j \neq i} r_i n_j\ \sigma_i \sigma_j \\
        &= 2 i \cos\frac{\alpha}{2}\sin\frac{\alpha}{2}
            \sum_{i = 1}^3 \sum_{j > i} (r_i n_j - r_j n_i) \sigma_i \sigma_j .
    \end{align}
    Nevertheless, it is possible to write the product of two distinct
    Pauli Matrices in terms of another Pauli Matrix. Namely,
    \begin{itemize}
        \item $\sigma_1 \sigma_2 = i \sigma_3$ and
            $\sigma_2 \sigma_1 = -i \sigma_3$;
        \item $\sigma_1 \sigma_3 = -i \sigma_2$ and
            $\sigma_3 \sigma_1 = i \sigma_2$;
        \item $\sigma_2 \sigma_3 = i \sigma_1$ and
            $\sigma_3 \sigma_2 = -i \sigma_1$.    
    \end{itemize}
    Thus, it is possible to rewrite $b$ with the aid of the function
    \begin{align}
        f(x) = (x \mod 3) + 1
    \end{align}
    as follows
    \begin{align}
        b = - 2 \cos \frac{\alpha}{2} \sin \frac{\alpha}{2}
            \sum_{i = 1}^3 (r_{f(i)}n_{f(i + 1)} - r_{f(i+1)}n_{f(i)}) \sigma_i ,
    \end{align}
    which is in the desired format.
    
    \item Equation \ref{eq:nielsen-and-chuang-exercise-4-6-c} can be rewritten
    as follows
    \begin{align}
        c &= \sin^2\frac{\alpha}{2} \hat{n}\cdot\vec{\sigma}\
            \vec{r}\cdot\vec{\sigma}\ \hat{n}\cdot\vec{\sigma} \\
        &= \sin^2\frac{\alpha}{2} \left(
            \sum_{i = 1}^3\sum_{j = 1}^3\sum_{k = 1}^3
            n_i r_j n_k\ \sigma_j \sigma_k \sigma_k
        \right) \\
        &= \sin^2\frac{\alpha}{2} \left( \sum_{i = 1}^3 n_i \sigma_i \left(
            \sum_{j = 1}^3\sum_{k=1}^3 r_j n_k\ \sigma_j \sigma_k
        \right)\right) \\
        &= \sin^2\frac{\alpha}{2} \left( \sum_{i = 1}^3 n_i \sigma_i \left(
            \sum_{j = 1}^3 r_j n_j I + \sum_{j = 1}^3\sum_{k\neq j} r_j n_k\ \sigma_j \sigma_k
        \right)\right) \\
        &= \sin^2 \frac{\alpha}{2} \left(
            \sum_{i = 1}^3\sum_{j = 1}^3 n_i r_j n_j \sigma_i +
            \sum_{i=1}^3\sum_{j=1}^3\sum_{k \neq j}
                n_i r_j n_k \sigma_i \sigma_j \sigma_k
        \right) \\
        &= \sin^2 \frac{\alpha}{2} \left(
            \sum_{i = 1}^3 \left( n_i^2 r_i \sigma_i +
            \sum_{j \neq i} n_i r_j n_j \sigma_i +
            \sum_{k \neq i} n_i r_i n_k\ I\sigma_k +
            \sum_{j \neq i}\sum_{k \neq j} n_i r_j n_k\ \sigma_i \sigma_j \sigma_k
        \right) \right) \\
        &= \sin^2 \frac{\alpha}{2} \left(
            \sum_{i = 1}^3 \left( n_i^2 r_i \sigma_i +
            \sum_{j \neq i} n_i r_j n_j \sigma_i +
            \sum_{j \neq i} n_i r_i n_j\ \sigma_j +
            \sum_{j \neq i}\sum_{k \neq j} n_i r_j n_k\ \sigma_i \sigma_j \sigma_k
        \right) \right) \\
        &= \sin^2 \frac{\alpha}{2} \left(
            \sum_{i = 1}^3 \left( n_i^2 r_i \sigma_i +
            \sum_{j \neq i} \left(
            n_i n_j (r_j \sigma_i + r_i \sigma_j) +
            \sum_{k \neq j} n_i r_j n_k\ \sigma_i \sigma_j \sigma_k
        \right) \right) \right) \\
        &= \sin^2 \frac{\alpha}{2} \left(
            \sum_{i = 1}^3 \left( n_i^2 r_i \sigma_i +
            \sum_{j \neq i} \left(
            n_i n_j (r_j \sigma_i + r_i \sigma_j) -
            n_i^2 r_j \sigma_j +
            \sum_{k \neq j, k \neq i} n_i r_j n_k\ \sigma_i \sigma_j \sigma_k
        \right) \right) \right) .
    \end{align}
    The last summation equals 0.
    To see this, note that,
    \begin{align}
        \sum_{i = 1}^3 \sum_{j \neq i} \sum_{k \neq i, k \neq j}
        n_i r_j n_k\ \sigma_i \sigma_j \sigma_k
        &= \sum_{j = 1}^3 \sum_{i \neq j} \sum_{k \neq i, k \neq j}
        n_i r_j n_k\ \sigma_i \sigma_j \sigma_k \\
        &= - \sum_{j = 1}^3 \sum_{i \neq j} \sum_{k \neq i, k \neq j}
        n_i r_j n_k\ \sigma_i \sigma_k \sigma_j \\
        &= \sum_{j = 1}^3 \sum_{i \neq j} \sum_{k \neq i, k \neq j}
        n_i r_j n_k\ \sigma_k \sigma_i \sigma_j \\
        &= - \sum_{j = 1}^3 \sum_{i \neq j} \sum_{k \neq i, k \neq j}
        n_i r_j n_k\ \sigma_k \sigma_j \sigma_i .
    \end{align}
    Since for a fixed value of $j$
    there are only two possible permutations,
    \begin{align}
        \sum_{j = 1}^3 \sum_{i \neq j} \sum_{k \neq i, k \neq j}
            n_i r_j n_k\ \sigma_i \sigma_j \sigma_k
            &= \sum_{j = 1}^3 \left( n_i r_j n_k\ \sigma_i \sigma_j \sigma_k +
            n_k r_j n_i\ \sigma_k \sigma_j \sigma_i \right) \\
        &= \sum_{j = 1}^3 \left( n_i r_j n_k\ \sigma_i \sigma_j \sigma_k -
            n_k r_j n_i\ \sigma_i \sigma_j \sigma_k \right) \\
        &= \sum_{j = 1}^3 0 \\
        &= 0.
    \end{align}
    Hence,
    \begin{align}
        c &= \sin^2 \frac{\alpha}{2} \left(
            \sum_{i = 1}^3 \left( n_i^2 r_i \sigma_i +
            \sum_{j \neq i} \left(
            n_i n_j (r_j \sigma_i + r_i \sigma_j) -
            n_i^2 r_j \sigma_j
        \right) \right) \right) \\
        &= \sin^2 \frac{\alpha}{2} \left(
            \sum_{i = 1}^3 \left(
                n_i^2 r_i \sigma_i + \sum_{j \neq i} n_i n_j r_j \sigma_i
            \right) +
            \sum_{i = 1}^3 \sum_{j \neq i} \left(
                n_i n_j r_i \sigma_j - n_i^2 r_j \sigma_j
            \right)
        \right) .
    \end{align}
    Since addition and multiplication are commutative,
    the previous equation can be rearranged as follows
    (basically, what is being done here is similar to rewriting the set
    of tuples $\set{(1,2), (1,3), (2,1), (2,3), (3,1), (3,2)}$ as
    $\set{(2,1), (3,1), (1, 2), (3, 2), (1,3), (2,3)}$, which is the same set),
    \begin{align}
        c &= \sin^2 \frac{\alpha}{2} \left(
            \sum_{i = 1}^3 \left(
                n_i^2 r_i \sigma_i + \sum_{j \neq i} n_i n_j r_j \sigma_i
            \right) +
            \sum_{i = 1}^3 \sum_{j \neq i} \left(
                n_j n_i r_j \sigma_i - n_j^2 r_i \sigma_i
            \right)
        \right) \\
        &= \sin^2 \frac{\alpha}{2} \left(
            \sum_{i=1}^3 \left( n_i^2 r_i + \sum_{j \neq i} \left(
                n_i n_j r_j + n_j n_i r_j - n_j^2 r_i
            \right) \right) \sigma_i
        \right) \\
        &= \sin^2 \frac{\alpha}{2} \left(
            \sum_{i=1}^3 \left( n_i^2 r_i + \sum_{j \neq i} \left(
                2 n_j n_i r_j - n_j^2 r_i
            \right) \right) \sigma_i
        \right),
    \end{align}
    which is in the desired format.
    
\end{enumerate}

Concatenating these results,
Equation \ref{eq:nielsen-and-chuang-exercise-4-6-split} can be rewritten as follows,
\begin{align}
    R_{\hat n}(\alpha) \rho R_{\hat n}(\alpha)^\dagger &= \frac{1}{2} \Bigg(
        \ I + \sum_{i = 1}^3 \bigg( \Big( r_i \cos^2\frac{\alpha}{2}
        - 2 \cos \frac{\alpha}{2} \sin \frac{\alpha}{2}
        (r_{f(i)}n_{f(i + 1)} - r_{f(i+1)}n_{f(i)})\ +
        \nonumber \\
        &\quad  \sin^2\frac{\alpha}{2} \big(
            n_i^2r_i + \sum_{j \neq i} (2 r_j n_i n_j - r_i n_j^2)
        \big)
    \Big)\ \sigma_i \bigg) \Bigg) .
\end{align}
Using the trigonometric identities
\begin{itemize}
    \item $\cos^2 \frac{\alpha}{2} = \frac{1}{2} (1 + \cos\alpha)$;
    \item $\sin^2 \frac{\alpha}{2} = \frac{1}{2} (1 - \cos\alpha)$;
    \item $\cos\frac{\alpha}{2} \sin\frac{\alpha}{2} = \frac{\sin\alpha}{2}$;
\end{itemize}
\begin{align}
    R_{\hat n}(\alpha) \rho R_{\hat n}(\alpha)^\dagger &= \frac{1}{2} \Bigg(
        \ I + \sum_{i = 1}^3 \bigg( \Big( \frac{r_i}{2} (1 + \cos\alpha)
        - \sin\alpha (r_{f(i)}n_{f(i + 1)} - r_{f(i+1)}n_{f(i)})\ +
        \nonumber \\
        &\quad \frac{1}{2} (1 - \cos\alpha) \big(
            n_i^2r_i + \sum_{j \neq i} (2 r_j n_i n_j - r_i n_j^2)
        \big)
    \Big)\ \sigma_i \bigg) \Bigg) .
\end{align}
Let
\begin{align}
    d = \sin\alpha (r_{f(i)}n_{f(i + 1)} - r_{f(i+1)}n_{f(i)});
\end{align}
Then,
\begin{align}
    R_{\hat n}(\alpha) \rho R_{\hat n}(\alpha)^\dagger &= \frac{1}{2} \Bigg(
        \ I + \sum_{i = 1}^3 \bigg( \Big( \frac{r_i}{2} (1 + \cos\alpha) +
        \frac{1}{2} (1 - \cos\alpha) \big(
            n_i^2r_i + \sum_{j \neq i} (2 r_j n_i n_j - r_i n_j^2)
        \big) - d
    \Big)\ \sigma_i \bigg) \Bigg) \\
    &= \frac{1}{2} \Bigg(
        \ I + \sum_{i = 1}^3 \bigg( \frac{1}{2} \Big( r_i (1 + \cos\alpha) +
        (1 - \cos\alpha) \big(
            n_i^2r_i + \sum_{j \neq i} (2 r_j n_i n_j - r_i n_j^2)
        \big)
    \Big)\ \sigma_i -d \sigma_i \bigg) \Bigg) .
\end{align}
Let
\begin{align}
    \hat d = \frac{1}{2} \Big( r_i (1 + \cos\alpha) +
        (1 - \cos\alpha) \big(
            n_i^2r_i + \sum_{j \neq i} (2 r_j n_i n_j - r_i n_j^2)
        \big)
    \Big) .
\end{align}
Then,
\begin{align}
    R_{\hat n}(\alpha) \rho R_{\hat n}(\alpha)^\dagger =
        \frac{1}{2} \left(I + \sum_{i=1}^3 (\hat{d}\sigma_i - d\sigma_i) \right).
\end{align}
Computing $\hat d$,
\begin{align}
    \hat{d} &= \frac{1}{2} \Big( r_i (1 + \cos\alpha) +
        (1 - \cos\alpha) \big(
            n_i^2r_i + \sum_{j \neq i} (2 r_j n_i n_j - r_i n_j^2)
        \big) \Big) \\
    &= \frac{1}{2} \Big(
        n_i^2r_i(1 - \cos\alpha) +
        r_i + r_i\cos\alpha +
        \sum_{j \neq i} (2 r_j n_i n_j - r_i n_j^2 -
            2 \cos(\alpha) r_j n_i n_j + \cos(\alpha) r_i n_j^2
        ) \Big) \\
    &= \frac{1}{2} \Big(
        n_i^2r_i(1 - \cos\alpha) +
        r_i \left(1 - \sum_{j \neq i} n_j^2\right) +
        r_i \cos\alpha \left(1 + \sum_{j \neq i} n_j^2 \right) +
        \sum_{j \neq i} (
            2 r_j n_i n_j - 2 \cos(\alpha) r_j n_i n_j
        ) \Big) .
\end{align}
Since $\hat n$ is a real unit vector,
$1 - n_i^2 = \sum_{j \neq i} n_j^2$.
Hence,
\begin{align}
    \hat{d} &= \frac{1}{2} \Big(
        n_i^2r_i(1 - \cos\alpha) +
        r_i n_i^2 +
        r_i \cos\alpha \left(2 - n_i^2 \right) +
        \sum_{j \neq i} r_j n_i n_j ( 2  - 2 \cos\alpha ) \Big) \\
    &= \frac{1}{2} \Big(
        n_i^2r_i(2 - \cos\alpha) +
        2 r_i \cos\alpha - r_i n_i^2 \cos\alpha +
        \sum_{j \neq i} r_j n_i n_j ( 2  - 2 \cos\alpha ) \Big) \\
    &= \frac{1}{2} \Big(
        n_i^2r_i(2 - 2 \cos\alpha) +
        2 r_i \cos\alpha +
        \sum_{j \neq i} r_j n_i n_j ( 2  - 2 \cos\alpha ) \Big) \\
    &= n_i^2r_i(1 - \cos\alpha) +
        r_i \cos\alpha +
        \sum_{j \neq i} r_j n_i n_j ( 1  - \cos\alpha ) .
\end{align}
In summary,
\begin{align}
    R_{\hat n}(\alpha) \rho R_{\hat n}(\alpha)^\dagger &= \frac{1}{2} \Bigg(
        I + \sum_{i = 1}^3 \bigg( \Big(
            n_i^2r_i(1 - \cos\alpha) +
            r_i \cos\alpha +
            \sum_{j \neq i} r_j n_i n_j \big(
                 1 - \cos(\alpha)
            \big)\ - \nonumber \\
            &\quad
            \sin\alpha (r_{f(i)}n_{f(i + 1)} - r_{f(i+1)}n_{f(i)})
        \Big) \sigma_i \bigg)
    \Bigg)
    \\
    &= \frac{1}{2} \Bigg(
        I + \sum_{i = 1}^3 \bigg( \Big(
            n_i^2r_i(1 - \cos\alpha) +
            r_i \cos\alpha +
            \sum_{j \neq i} r_j n_i n_j \big(
                 1 - \cos(\alpha)
            \big)\ - \nonumber \\
            &\quad
            \sin\alpha (r_{f(i+1)}n_{f(i)} - r_{f(i)}n_{f(i + 1)})
        \Big) \sigma_i \bigg)
    \Bigg).
\end{align}
In conclusion,
the obtained Bloch Vector $r' = (r'_1, r'_2, r'_2)^\dagger$ is described --
for each $r'_i$, $1 \leq i \leq 3$ -- as follows,
\begin{align}
    r_i' = n_i^2r_i(1 - \cos\alpha) +
            r_i \cos\alpha +
            \sum_{j \neq i} r_j n_i n_j \big(
                1 - \cos\alpha
            \big) +
            \sin\alpha (r_{f(i+1)}n_{f(i)} - r_{f(i)}n_{f(i + 1)}) .
    \label{eq:nielsen-and-chuang-exercise-4-6-final-result}
\end{align}

To assert this result, compute Equation \ref{eq:nielsen-and-chuang-exercise-4-6-assert}
and verify that the results match.
That is, verify that each $r_i'$ computed using
Equation \ref{eq:nielsen-and-chuang-exercise-4-6-final-result} matches
the values $r_1'$, $r_2'$ and $r_3'$ obtained through
\begin{align}
    \left[ \begin{matrix}
        r_1' \\ r_2' \\ r_3'
    \end{matrix} \right] =
    \left[ \begin{matrix}
        n_1^2 (1 - \cos\alpha) + \cos\alpha &
        n_1 n_2 (1 - \cos\alpha) - n_3 \sin\alpha &
        n_1 n_3 (1 - \cos\alpha) + n_2 \sin\alpha
        \\
        n_1 n_2 (1 - \cos\alpha) + n_3 \sin\alpha &
        n_2^2 (1 - \cos\alpha) + \cos\alpha &
        n_2 n_3 (1 - \cos\alpha) - n_1 \sin\alpha
        \\
        n_1 n_3 (1 - \cos\alpha) - n_2 \sin\alpha &
        n_2 n_3 (1 - \cos\alpha) + n_1 \sin\alpha &
        n_3^2 (1 - \cos\alpha) + \cos\alpha
    \end{matrix} \right]
    \left[ \begin{matrix}
        r_1 \\ r_2 \\ r_3
    \end{matrix} \right].
\end{align}

As the book states,
the ``mysterious factor 2'' appears because of the
representation of a state in the Bloch Sphere
(refer to Section
\ref{nielsen-and-chuang-qubit-representation-in-a-bloch-sphere}).