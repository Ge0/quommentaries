\subsection{Section 1.2}
\subsubsection{Qubit representation in a Bloch Sphere}
\label{nielsen-and-chuang-qubit-representation-in-a-bloch-sphere}

The Bloch Sphere's equations explanation is not given by the book.

\begin{align}
\ket{\psi} = 
    cos \frac{\theta}{2}\ket{0} + e^{i \varphi} sin \frac{\theta}{2} \ket{1} .
    \label{eq:nielsen-and-chuang-sec-1-2-bloch-sphere}
\end{align}

However, \href{https://github.com/victoragnez}{Agnez} came up with a simple explanation using spherical coordinates.
Its details can be found in
\href{https://bits2qubits.blogspot.com/2018/09/representacao-de-qubits-na-esfera-de.html}{Bits2Qubits blog}
(the post is in Portuguese).

The details of this equation are not necessary until Chapter 4.
At this point, however, it is sufficient for the reader to know that
a Qubit can be represented as a point in a sphere.
A detailed explanation of the Bloch Sphere equations can be found in Sections
\hyperref[sec:noson-bloch-sphere-eq-5-80-to-5-88]{ \ref{sec:noson-bloch-sphere-eq-5-80-to-5-88}} and
\hyperref[sec:noson-exercise-5-4-4]{\ref{sec:noson-exercise-5-4-4}}.
However, it requires information that will be mentioned throughout
\hyperref[sec:nielsen-and-chuang-chapter-2]{Nielsen and Chuang's book's Chapter 2}
or in Noson's book's Chapters 4 and 5.

After comprehending Sections
\hyperref[sec:noson-bloch-sphere-eq-5-80-to-5-88]{ \ref{sec:noson-bloch-sphere-eq-5-80-to-5-88}} and
\hyperref[sec:noson-exercise-5-4-4]{\ref{sec:noson-exercise-5-4-4}},
the following Equation is obtained.
\begin{align}
    \ket\psi = cos\theta \ket0 + e^{i\phi} sin\theta \ket1,
\end{align}
where $\theta \in [0, \frac \pi 2]$, and $\phi \in [0, 2\pi)$.
Nielsen and Chuang's book uses $\theta \in [0, \pi]$.
Thus obtaining \hyperref[eq:nielsen-and-chuang-sec-1-2-bloch-sphere]{
    Equation \ref{eq:nielsen-and-chuang-sec-1-2-bloch-sphere}
}.

The book's Equation (1.3) can be obtained by mutiplying \hyperref[eq:nielsen-and-chuang-sec-1-2-bloch-sphere]{
    Equation \ref{eq:nielsen-and-chuang-sec-1-2-bloch-sphere}
} by $e^{i\gamma}$.
    \footnote{Actually, \hyperref[eq:nielsen-and-chuang-sec-1-2-bloch-sphere]{
        Equation \ref{eq:nielsen-and-chuang-sec-1-2-bloch-sphere}
        }
    was originally obtained \emph{from} Equation (1.3) by multiplying it by $e^{-i\gamma}$.
    }
Since it does not change the state,
as explained in \hyperref[sec:noson-equation-4-5]{Section \ref{sec:noson-equation-4-5}}.