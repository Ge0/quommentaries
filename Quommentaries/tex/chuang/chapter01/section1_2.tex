\subsection{Section 1.2}
\subsubsection{Qubit representation in a Bloch Sphere}

The Bloch Sphere's equations explanation is not given by the book.

\begin{align}
\ket{\psi} = 
    cos \frac{\theta}{2}\ket{0} + e^{i \varphi} sin \frac{\theta}{2} \ket{1} .
    \label{eq:nielsen-and-chuang-sec-1-2-bloch-sphere}
\end{align}

However, \href{https://github.com/victoragnez}{Agnez} came up with a simple explanation using spherical coordinates.
Its details can be found in
\href{https://bits2qubits.blogspot.com/2018/09/representacao-de-qubits-na-esfera-de.html}{Bits2Qubits blog}
(the post is in Portuguese).

The details of this equation are not necessary until Chapter 4.
At this point, however, it is sufficient for the reader to know that
a Qubit can be represented as a point in a sphere.
A detailed explanation of the Bloch Sphere equations can be found in Sections
\hyperref[sec:noson-bloch-sphere-eq-5-80-to-5-88]{ \ref{sec:noson-bloch-sphere-eq-5-80-to-5-88}} and
\hyperref[sec:noson-exercise-5-4-4]{\ref{sec:noson-exercise-5-4-4}}.
However, it requires information that will be mentioned throughout
\hyperref[sec:nielsen-and-chuang-chapter-2]{Nielsen and Chuang's book's Chapter 2}
or in Noson's book's Chapters 4 and 5.

After comprehending Sections
\hyperref[sec:noson-bloch-sphere-eq-5-80-to-5-88]{ \ref{sec:noson-bloch-sphere-eq-5-80-to-5-88}} and
\hyperref[sec:noson-exercise-5-4-4]{\ref{sec:noson-exercise-5-4-4}},
the following Equation is obtained.
\begin{align}
    \ket\psi = cos\theta \ket0 + e^{i\phi} sin\theta \ket1,
\end{align}
where $\theta \in [0, \frac \pi 2]$, and $\phi \in [0, 2\pi)$.
Nielsen and Chuang's book uses $\theta \in [0, \pi]$.
Thus obtaining \hyperref[eq:nielsen-and-chuang-sec-1-2-bloch-sphere]{
    Equation \ref{eq:nielsen-and-chuang-sec-1-2-bloch-sphere}
}.

The book's Equation (1.3) can be obtained by mutiplying \hyperref[eq:nielsen-and-chuang-sec-1-2-bloch-sphere]{
    Equation \ref{eq:nielsen-and-chuang-sec-1-2-bloch-sphere}
} by $e^{i\gamma}$.
    \footnote{Actually, \hyperref[eq:nielsen-and-chuang-sec-1-2-bloch-sphere]{
        Equation \ref{eq:nielsen-and-chuang-sec-1-2-bloch-sphere}
        }
    was originally obtained \emph{from} Equation (1.3) by multiplying it by $e^{-i\gamma}$.
    }
Since it does not change the state,
as explained in \hyperref[sec:noson-equation-4-5]{Section \ref{sec:noson-equation-4-5}}.

\begin{comment}
    Note that any Qubit may be written as
    \begin{align}
        \ket{\psi} = \alpha \ket0 + \beta \ket1 ,
    \end{align}
    where $\alpha, \beta \in \mathbb{C}$.
    Recall that any complex number can be written in the polar form.
    That is, if $z \in \mathbb{C}$ ($z = a + bi$, where $a, b \in \mathbb{R}$),
    then $z = r(cos\theta + i\ sin\theta$),
    where $\theta \in [0, 2\pi)$,
    and $r \in \mathbb{R}$ and it is the norm (length) of $z$
    ($|z| = r = \sqrt{a^2 + b^2}$).
    This representation is useful because it simplifies complex numbers multiplications and exponentiation.
    Additionally, it gives a geometrical interpretation for complex numbers:
    they can be represented as a point in a plane.
    For instance, $z = 1 + i = \sqrt2 (cos \frac{\pi}{2} + i\ sin \frac{\pi}{2})$,
    
    Therefore, considering a Qubit ($\ket\psi = \alpha \ket0 + \beta \ket1$),
    its complex values $\alpha$ and $\beta$ can be expressed in two different planes
    (since they correspond to different vectors).
    For instance, in \hyperref[fig:nielsen-and-chuang-sec-1-2-plane-intersec]{
        Figure \ref{fig:nielsen-and-chuang-sec-1-2-plane-intersec}
    },
    
    Note, however, that $|\alpha|^2 + |\beta|^2 = 1$.
    Therefore, the possible values for $\alpha$ and $\beta$ form a
    sphere with radius $1$ in the three-dimensional space (the \textbf{Bloch Sphere},
    illustrated in \hyperref[fig:nielsen-and-chuang-sec-1-2-bloch-sphere]{
        Figure \ref{fig:nielsen-and-chuang-sec-1-2-bloch-sphere}
    }).
    In addition, any point in a sphere can be written using spherical coordinates,
    e.g. Earth's latitude and longitude.
    
    Until Chapter 4, it is sufficient for the reader to comprehend that
    
    describes that any Qubit has two degrees of freedom - $\theta$ and $\varphi$ 
    (which may be interpreted as latitude and longitude).
\end{comment}