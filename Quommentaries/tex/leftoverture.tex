\section{Leftoverture}
This repository is dedicated to solve exercises and comment on Quantum Computing. Most of the discussion is based on \href{https://www.cambridge.org/core/books/quantum-computation-and-quantum-information/01E10196D0A682A6AEFFEA52D53BE9AE}{Nielsen And Chuang's book "Quantum Computation and Quantum Information"}.
In addition, \href{https://books.google.com.br/books?hl=pt-BR&lr=&id=8jwVDAAAQBAJ&oi=fnd&pg=PR5&dq=an+introduction+quantum+computation+mosca&ots=1EgvgtQL_A&sig=YqURWZlJOHdatZHyly-cNPLhdxM#v=onepage&q=an\%20introduction\%20quantum\%20computation\%20mosca&f=false}{Kaye, Laflamme and Mosca's "An Introduction to Quantum Computing"}
is used as a complementary book, as well as \href{http://www.cambridge.org/gb/academic/subjects/computer-science/cryptography-cryptology-and-coding/quantum-computing-computer-scientists?format=HB}{Yanofsky and Mannucci's "Quantum Computing for Computer Scientists"}
- recommended by \href{http://vitorgreati.me/}{Greati}.

\subsection{Objective}
Although Nielsen and Chuang's book is very famous, some equations may be solved too quickly. This may discourage the reader to continue the studies if the basic concepts were not mastered. One of the objectives of this repository is to support those who are studying Quantum Computing and Quantum Information by explaining some of these equations step-by-step.

In addition, the exercises present in the book may not be trivial for beginners. Hence, this repository attempts to help the students by showing a detailed solution or, at least, a sketch.

\subsection{Disclaimer}
This repository is being constructed by an \textbf{undergaduate student}. Henceforth, the notes, commentaries and exercises are \textbf{suscetible to errors}. Please, \textbf{do not hesitate to give feedback} (\href{mailto:gustavowl@lcc.ufrn.br}{gustavowl@lcc.ufrn.br}).

\pagebreak